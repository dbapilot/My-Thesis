% Define colors
\definecolor{codegreen}{rgb}{0,0.6,0}  % ✅ Green for comments
\definecolor{codegray}{rgb}{0.5,0.5,0.5}  % ✅ Gray for numbers
\definecolor{codepurple}{rgb}{0.58,0,0.82}  % ✅ Purple for strings
\definecolor{backcolour}{rgb}{0.95,0.95,0.92}  % ✅ Light gray background
\definecolor{bordercolor}{rgb}{0.7,0.7,0.7}  % ✅ Left border color (gray)
\definecolor{codeblue}{rgb}{0,0,0.8}  % ✅ Blue for SQL keywords

% ✅ Define SQL language with correct comment handling
\lstdefinelanguage{MySQL}{
    keywords={SELECT, FROM, WHERE, JOIN, ON, INNER, OUTER, LEFT, RIGHT, FULL, GROUP, BY, ORDER, ASC, DESC, AS, COUNT, SUM, AVG, MAX, MIN, DISTINCT, INSERT, INTO, VALUES, UPDATE, SET, DELETE, CREATE, TABLE, PRIMARY, FOREIGN, KEY, DEFAULT, NULL, NOT, CHECK, CONSTRAINT, INDEX, VIEW, MATERIALIZED, PROCEDURE, FUNCTION, TRIGGER, DATABASE, ALTER, DROP, EXEC, IF, EXISTS, UNION, ALL, CASE, WHEN, THEN, ELSE, END, CAST, CONVERT, LIKE, IN, BETWEEN, AND, OR, HAVING, LIMIT, OFFSET,EXEC,sp_add_jobstep,sp_refreshview,N,sp_add_schedule,sp_attach_schedule,dbo,Sp_add_jobserver,go},
    sensitive=false,
    morestring=[b]',  % ✅ Strings in single quotes
    morestring=[b]",  % ✅ Strings in double quotes
    morecomment=[l][\color{codegreen}]{--}  % ✅ Ensures full line comment in green
}

\lstdefinestyle{sqlstyle}{
    alsoletter={.,_}, % Treat dots and underscores as part of identifiers
    showstringspaces=false, % Prevent special space character
    backgroundcolor=\color{backcolour},   
    commentstyle=\color{codegreen},  % ✅ Comments in green
    keywordstyle=\bfseries\color{codeblue},  % ✅ SQL Keywords in Blue & Bold
    numberstyle=\scriptsize\color{codegray},  % ✅ Row numbers in gray
    stringstyle=\color{codepurple},  % ✅ Strings in purple
    basicstyle=\ttfamily\footnotesize,
    breaklines=true,
    captionpos=b,
    numbers=left,      % ✅ Enables row numbers on the left
    stepnumber=1,      % ✅ Row numbers increment by 1
    firstnumber=1,     % ✅ Starts numbering at 1
    numbersep=8pt,     % ✅ Increases space between numbers and SQL code
    xleftmargin=3em,   % ✅ Ensures space inside the left border
    frame=single,      % ✅ Keeps a single border (left-aligned)
    framesep=5pt,      % ✅ Ensures space inside the frame
    rulesepcolor=\color{bordercolor},  % ✅ Matches row numbers with left border
    rulecolor=\color{bordercolor},  % ✅ Sets left border color
    language=MySQL  % ✅ Uses SQL keyword highlighting
}
         \begin{lstlisting}[style=sqlstyle, caption={Maintenance and Refresh Strategies}, label=lst:Maintenance_and_Refresh_Strategies]

     ---1. Incremental refresh 
     
    CREATE TRIGGER trg_UpdateClaims
ON Claims
AFTER INSERT, UPDATE, DELETE
AS
BEGIN
    -- Refresh materialized view logic here
    EXEC sp_refreshview 'TotalClaimsByPatient';
END;
GO

--This trigger ensures that the TotalClaimsByPatient materialized view stays up to date. It automatically refreshes the view whenever records in the Claims table are inserted, updated, or deleted by calling sp_refreshview.



     ---2. Example of creating a job to refresh a materialized view automatically every hour

USE msdb;
GO

-- Create the job
EXEC dbo.sp_add_job
    @job_name = N'RefreshMaterializedViewJob',
    @enabled = 1,
    @description = N'Job to refresh the materialized view every hour.';

-- Add a job step
EXEC sp_add_jobstep
    @job_name = N'RefreshMaterializedViewJob',
    @step_name = N'RefreshViews',
    @subsystem = N'TSQL',
    @command = 'EXEC sp_refreshview ''TotalClaimsByPatient'';
               EXEC sp_refreshview ''TotalTreatmentCostByProvider'';
               EXEC sp_refreshview ''MonthlyClaimsByProvider'';',
    @retry_attempts = 3,
    @retry_interval = 5;

-- Create a schedule for the job
EXEC sp_add_schedule
    @schedule_name = N'HourlySchedule',
    @freq_type = 4, -- Daily
    @freq_interval = 1, -- Every day
    @freq_subday_type = 8, -- Hourly
    @freq_subday_interval = 1, -- Every 1 hour
    @active_start_time = 000000; -- Start time (midnight)

-- Attach the schedule to the job
EXEC sp_attach_schedule
    @job_name = N'RefreshMaterializedViewJob',
    @schedule_name = N'HourlySchedule';

-- Assign the job to the SQL Server Agent service
EXEC dbo.Sp_add_jobserver
     @job_name = N'RefreshMaterializedViewJob';

GO



      ---3. Manual Refresh
EXEC sp_refreshview 'TotalClaimsByPatient';
EXEC sp_refreshview 'TotalTreatmentCostByProvider';
EXEC sp_refreshview 'MonthlyClaimsByProvider';

--The EXEC sp_refreshview statements refresh the metadata for each specified view. This ensures that the views are up to date with any changes made to the underlying tables (like added columns or altered data types), helping to prevent potential schema-related errors during future queries or operations.
        \end{lstlisting}