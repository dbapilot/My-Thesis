% Define colors
\definecolor{codegreen}{rgb}{0,0.6,0}  % ✅ Green for comments
\definecolor{codegray}{rgb}{0.5,0.5,0.5}  % ✅ Gray for numbers
\definecolor{codepurple}{rgb}{0.58,0,0.82}  % ✅ Purple for strings
\definecolor{backcolour}{rgb}{0.95,0.95,0.92}  % ✅ Light gray background
\definecolor{bordercolor}{rgb}{0.7,0.7,0.7}  % ✅ Left border color (gray)
\definecolor{codeblue}{rgb}{0,0,0.8}  % ✅ Blue for SQL keywords

% ✅ Define SQL language with correct comment handling
\lstdefinelanguage{MySQL}{
    keywords={SELECT, FROM, WHERE, JOIN, ON, INNER, OUTER, LEFT, RIGHT, FULL, GROUP, BY, ORDER, ASC, DESC, AS, COUNT, SUM, AVG, MAX, MIN, DISTINCT, INSERT, INTO, VALUES, UPDATE, SET, DELETE, CREATE, TABLE, PRIMARY, FOREIGN, KEY, DEFAULT, NULL, NOT, CHECK, CONSTRAINT, INDEX, VIEW, MATERIALIZED, PROCEDURE, FUNCTION, TRIGGER, DATABASE, ALTER, DROP, EXEC, IF, EXISTS, UNION, ALL, CASE, WHEN, THEN, ELSE, END, CAST, CONVERT, LIKE, IN, BETWEEN, AND, OR, HAVING, LIMIT, OFFSET},
    sensitive=false,
    morestring=[b]',  % ✅ Strings in single quotes
    morestring=[b]",  % ✅ Strings in double quotes
    morecomment=[l][\color{codegreen}]{--}  % ✅ Ensures full line comment in green
}

\lstdefinestyle{sqlstyle}{
    backgroundcolor=\color{backcolour},   
    commentstyle=\color{codegreen},  % ✅ Comments in green
    keywordstyle=\bfseries\color{codeblue},  % ✅ SQL Keywords in Blue & Bold
    numberstyle=\scriptsize\color{codegray},  % ✅ Row numbers in gray
    stringstyle=\color{codepurple},  % ✅ Strings in purple
    basicstyle=\ttfamily\footnotesize,
    breaklines=true,
    captionpos=b,
    numbers=left,      % ✅ Enables row numbers on the left
    stepnumber=1,      % ✅ Row numbers increment by 1
    firstnumber=1,     % ✅ Starts numbering at 1
    numbersep=8pt,     % ✅ Increases space between numbers and SQL code
    xleftmargin=3em,   % ✅ Ensures space inside the left border
    frame=single,      % ✅ Keeps a single border (left-aligned)
    framesep=5pt,      % ✅ Ensures space inside the frame
    rulesepcolor=\color{bordercolor},  % ✅ Matches row numbers with left border
    rulecolor=\color{bordercolor},  % ✅ Sets left border color
    language=MySQL  % ✅ Uses SQL keyword highlighting
}
  
\begin{lstlisting}[style=sqlstyle, caption={Materialized view creation},]
--Materialized View for Query 1: Total Sales by Customer

CREATE VIEW totalsalesbycustomer
WITH schemabinding
AS
  SELECT customerid,
         Sum(totalamount) AS TotalSales
  FROM   dbo.orders
  GROUP  BY customerid;

go

CREATE UNIQUE CLUSTERED INDEX ix_totalsalesbycustomer
  ON totalsalesbycustomer (customerid);

go 

-- This view calculates the total sales amount for each customer by aggregating the TotalAmount column from the Orders table.

--Materialized View for Query 2: Total Quantity by Product


CREATE VIEW totalquantitybyproduct
WITH schemabinding
AS
  SELECT productid,
         Sum(quantity) AS TotalQuantity
  FROM   dbo.orderdetails
  GROUP  BY productid;

go

CREATE UNIQUE CLUSTERED INDEX ix_totalquantitybyproduct
  ON totalquantitybyproduct (productid);

go 

-- This view calculates the total quantity sold for each product by aggregating the Quantity column from the OrderDetails table.

--Materialized View for Query 3: Monthly Sales

CREATE VIEW monthlysales
WITH schemabinding
AS
  SELECT Year(orderdate)  AS SalesYear,
         Month(orderdate) AS SalesMonth,
         Sum(totalamount) AS MonthlySales
  FROM   dbo.orders
  GROUP  BY Year(orderdate),
            Month(orderdate);

go

CREATE UNIQUE CLUSTERED INDEX ix_monthlysales
  ON monthlysales (salesyear, salesmonth);

go 

-- This view calculates the total sales amount for each month by aggregating the TotalAmount column from the Orders table, grouped by year and month.
\end{lstlisting}