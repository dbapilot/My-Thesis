% Define colors
\definecolor{codegreen}{rgb}{0,0.6,0}  % ✅ Green for comments
\definecolor{codegray}{rgb}{0.5,0.5,0.5}  % ✅ Gray for numbers
\definecolor{codepurple}{rgb}{0.58,0,0.82}  % ✅ Purple for strings
\definecolor{backcolour}{rgb}{0.95,0.95,0.92}  % ✅ Light gray background
\definecolor{bordercolor}{rgb}{0.7,0.7,0.7}  % ✅ Left border color (gray)
\definecolor{codeblue}{rgb}{0,0,0.8}  % ✅ Blue for SQL keywords

\definecolor{commentcolor}{RGB}{0, 128, 0}  % Green for comments
\definecolor{titlecolor}{RGB}{0, 0, 255}    % Blue for titles
% ✅ Define SQL language with correct comment handling
\lstdefinelanguage{MySQL}{
    keywords={SELECT, FROM, WHERE, JOIN, ON, INNER, OUTER, LEFT, RIGHT, FULL, GROUP, BY, ORDER, ASC, DESC, AS, COUNT, SUM, AVG, MAX, MIN, DISTINCT, INSERT, INTO, VALUES, UPDATE, SET, DELETE, CREATE, TABLE, PRIMARY, FOREIGN, KEY, DEFAULT, NULL, NOT, CHECK, CONSTRAINT, INDEX, VIEW, MATERIALIZED, PROCEDURE, FUNCTION, TRIGGER, DATABASE, ALTER, DROP, EXEC, IF, EXISTS, UNION, ALL, CASE, WHEN, THEN, ELSE, END, CAST, CONVERT, LIKE, IN, BETWEEN, AND, OR, HAVING, LIMIT, OFFSET},
    sensitive=false,
    morestring=[b]',  % ✅ Strings in single quotes
    morestring=[b]",  % ✅ Strings in double quotes
    morecomment=[l][\color{codegreen}]{--}  % ✅ Ensures full line comment in green
}

\lstdefinestyle{sqlstyle}{
    backgroundcolor=\color{backcolour},   
    commentstyle=\color{codegreen},  % ✅ Comments in green
    keywordstyle=\bfseries\color{codeblue},  % ✅ SQL Keywords in Blue & Bold
    numberstyle=\scriptsize\color{codegray},  % ✅ Row numbers in gray
    stringstyle=\color{codepurple},  % ✅ Strings in purple
    basicstyle=\ttfamily\footnotesize,
    breaklines=true,
    captionpos=b,
    title=\color{titlecolor}\textbf{Table Creation Script}, % Title in blue
    numbers=left,      % ✅ Enables row numbers on the left
    stepnumber=1,      % ✅ Row numbers increment by 1
    firstnumber=1,     % ✅ Starts numbering at 1
    numbersep=8pt,     % ✅ Increases space between numbers and SQL code
    xleftmargin=3em,   % ✅ Ensures space inside the left border
    frame=single,      % ✅ Keeps a single border (left-aligned)
    framesep=5pt,      % ✅ Ensures space inside the frame
    rulesepcolor=\color{bordercolor},  % ✅ Matches row numbers with left border
    rulecolor=\color{bordercolor},  % ✅ Sets left border color
    language=MySQL  % ✅ Uses SQL keyword highlighting
}
  
\begin{lstlisting}[style=sqlstyle, caption={Materialized view creation}, label=lst:MV_creation]
--Materialized View for Query 1: Total Claims By Patient
CREATE VIEW TotalClaimsByPatient
WITH SCHEMABINDING
AS
SELECT PatientID, COUNT_BIG(*) AS TotalClaims
FROM dbo.Claims
GROUP BY PatientID;
GO

CREATE UNIQUE CLUSTERED INDEX IX_TotalClaimsByPatient
ON TotalClaimsByPatient (PatientID);
GO

-- This view calculates the total number of claims submitted by each patient. It groups the data by PatientID and uses COUNT_BIG(*) to count all related claims.



--Materialized View for Query 2: TotalTreatmentCostByProvider


CREATE VIEW TotalTreatmentCostByProvider
WITH SCHEMABINDING
AS
SELECT c.InsuranceProviderID, SUM(t.Cost) AS TotalCost
FROM dbo.Treatments t
JOIN dbo.Claims c ON t.ClaimID = c.ClaimID
GROUP BY c.InsuranceProviderID;
GO

CREATE UNIQUE CLUSTERED INDEX IX_TotalTreatmentCostByProvider
ON TotalTreatmentCostByProvider (InsuranceProviderID);
GO

-- This view computes the total treatment cost for each insurance provider. It joins the Treatments and Claims tables and sums up the Cost field, grouped by InsuranceProviderID.

--Materialized View for Query 3: MonthlyClaimsByProvider

CREATE VIEW MonthlyClaimsByProvider
WITH SCHEMABINDING
AS
SELECT 
    c.InsuranceProviderID, 
    YEAR(c.ClaimDate) AS ClaimYear, 
    MONTH(c.ClaimDate) AS ClaimMonth, 
    COUNT_BIG(*) AS TotalClaims
FROM dbo.Claims c
GROUP BY c.InsuranceProviderID, YEAR(c.ClaimDate), MONTH(c.ClaimDate);
GO

CREATE UNIQUE CLUSTERED INDEX IX_MonthlyClaimsByProvider
ON MonthlyClaimsByProvider (InsuranceProviderID, ClaimYear, ClaimMonth);
GO

-- This view summarizes the number of claims submitted per insurance provider for each month. It extracts the year and month from ClaimDate and counts claims grouped by provider, year, and month.
\end{lstlisting}