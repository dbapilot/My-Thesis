% Define colors
\definecolor{codegreen}{rgb}{0,0.6,0}  % ✅ Green for comments
\definecolor{codegray}{rgb}{0.5,0.5,0.5}  % ✅ Gray for numbers
\definecolor{codepurple}{rgb}{0.58,0,0.82}  % ✅ Purple for strings
\definecolor{backcolour}{rgb}{0.95,0.95,0.92}  % ✅ Light gray background
\definecolor{bordercolor}{rgb}{0.7,0.7,0.7}  % ✅ Left border color (gray)
\definecolor{codeblue}{rgb}{0,0,0.8}  % ✅ Blue for SQL keywords

\definecolor{commentcolor}{RGB}{0, 128, 0}  % Green for comments
\definecolor{titlecolor}{RGB}{0, 0, 255}    % Blue for titles
% ✅ Define SQL language with correct comment handling
\lstdefinelanguage{MySQL}{
    keywords={SELECT, FROM, WHERE, JOIN, ON, INNER, OUTER, LEFT, RIGHT, FULL, GROUP, BY, ORDER, ASC, DESC, AS, COUNT, SUM, AVG, MAX, MIN, DISTINCT, INSERT, INTO, VALUES, UPDATE, SET, DELETE, CREATE, TABLE, PRIMARY, FOREIGN, KEY, DEFAULT, NULL, NOT, CHECK, CONSTRAINT, INDEX, VIEW, MATERIALIZED, PROCEDURE, FUNCTION, TRIGGER, DATABASE, ALTER, DROP, EXEC, IF, EXISTS, UNION, ALL, CASE, WHEN, THEN, ELSE, END, CAST, CONVERT, LIKE, IN, BETWEEN, AND, OR, HAVING, LIMIT, OFFSET},
    sensitive=false,
    morestring=[b]',  % ✅ Strings in single quotes
    morestring=[b]",  % ✅ Strings in double quotes
    morecomment=[l][\color{codegreen}]{--}  % ✅ Ensures full line comment in green
}

\lstdefinestyle{sqlstyle}{
    backgroundcolor=\color{backcolour},   
    commentstyle=\color{codegreen},  % ✅ Comments in green
    keywordstyle=\bfseries\color{codeblue},  % ✅ SQL Keywords in Blue & Bold
    numberstyle=\scriptsize\color{codegray},  % ✅ Row numbers in gray
    stringstyle=\color{codepurple},  % ✅ Strings in purple
    basicstyle=\ttfamily\footnotesize,
    breaklines=true,
    captionpos=b,
    title=\color{titlecolor}\textbf{Table Creation Script}, % Title in blue
    numbers=left,      % ✅ Enables row numbers on the left
    stepnumber=1,      % ✅ Row numbers increment by 1
    firstnumber=1,     % ✅ Starts numbering at 1
    numbersep=8pt,     % ✅ Increases space between numbers and SQL code
    xleftmargin=3em,   % ✅ Ensures space inside the left border
    frame=single,      % ✅ Keeps a single border (left-aligned)
    framesep=5pt,      % ✅ Ensures space inside the frame
    rulesepcolor=\color{bordercolor},  % ✅ Matches row numbers with left border
    rulecolor=\color{bordercolor},  % ✅ Sets left border color
    language=MySQL  % ✅ Uses SQL keyword highlighting
}
  
\begin{lstlisting}[style=sqlstyle, caption={Materialized view creation}, label=lst:Analysis]
USE healthInsuranceDB;
GO

-- Clear caches for accurate testing (development only)
DBCC DROPCLEANBUFFERS;
DBCC FREEPROCCACHE;
GO

PRINT '===========================================================';
PRINT 'PERFORMANCE COMPARISON: DIRECT QUERIES vs MATERIALIZED VIEWS';
PRINT '===========================================================';
GO

-- Enable time statistics
SET STATISTICS TIME ON;
GO

-- ===========================================================
-- TEST CASE 1: Total Claims By Patient
-- ===========================================================
PRINT CHAR(10) + '1. TOTAL CLAIMS BY PATIENT' + CHAR(10) + REPLICATE('-', 40);

PRINT 'DIRECT QUERY:';
SELECT 
    p.PatientID,
    p.FirstName,
    p.LastName,
    COUNT(c.ClaimID) AS TotalClaims
FROM Patients p
JOIN Claims c ON p.PatientID = c.PatientID
GROUP BY p.PatientID, p.FirstName, p.LastName;

PRINT CHAR(10) + 'MATERIALIZED VIEW:';
SELECT * FROM TotalClaimsByPatient;
GO

-- ===========================================================
-- TEST CASE 2: Total Treatment Cost By Provider
-- ===========================================================
PRINT CHAR(10) + '2. TOTAL TREATMENT COST BY PROVIDER' + CHAR(10) + REPLICATE('-', 40);

PRINT 'DIRECT QUERY:';
SELECT 
    ip.ProviderName,
    SUM(t.Cost) AS TotalTreatmentCost,
    COUNT(t.TreatmentID) AS TreatmentCount
FROM Treatments t
JOIN Claims c ON t.ClaimID = c.ClaimID
JOIN InsuranceProviders ip ON c.InsuranceProviderID = ip.InsuranceProviderID
GROUP BY ip.ProviderName;

PRINT CHAR(10) + 'MATERIALIZED VIEW:';
SELECT * FROM TotalTreatmentCostByProvider;
GO

-- ===========================================================
-- TEST CASE 3: Monthly Claims By Provider
-- ===========================================================
PRINT CHAR(10) + '3. MONTHLY CLAIMS BY PROVIDER' + CHAR(10) + REPLICATE('-', 40);

PRINT 'DIRECT QUERY:';
SELECT 
    YEAR(c.ClaimDate) AS ClaimYear,
    MONTH(c.ClaimDate) AS ClaimMonth,
    ip.ProviderName,
    COUNT(c.ClaimID) AS ClaimCount,
    SUM(c.ClaimAmount) AS TotalClaimAmount
FROM Claims c
JOIN InsuranceProviders ip ON c.InsuranceProviderID = ip.InsuranceProviderID
GROUP BY YEAR(c.ClaimDate), MONTH(c.ClaimDate), ip.ProviderName;

PRINT CHAR(10) + 'MATERIALIZED VIEW:';
SELECT * FROM MonthlyClaimsByProvider;
GO

-- Disable statistics
SET STATISTICS TIME OFF;
GO

PRINT CHAR(10) + '===========================================================';
PRINT 'SUMMARY OF RESULTS';
PRINT '===========================================================';
PRINT 'Compare the CPU time and elapsed time values in Messages tab for:';
PRINT '1. Total Claims By Patient';
PRINT '2. Total Treatment Cost By Provider';
PRINT '3. Monthly Claims By Provider';
PRINT '';
PRINT 'Note: Materialized views will show significantly lower times';
PRINT '===========================================================';
\end{lstlisting}