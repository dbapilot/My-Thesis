% Define colors
\definecolor{codegreen}{rgb}{0,0.6,0}  % Green for comments
\definecolor{codegray}{rgb}{0.5,0.5,0.5}  % Gray for numbers
\definecolor{codepurple}{rgb}{0.58,0,0.82}  % Purple for strings
\definecolor{backcolour}{rgb}{0.95,0.95,0.92}  % Light gray background
\definecolor{bordercolor}{rgb}{0.7,0.7,0.7}  % Left border color (gray)
\definecolor{codeblue}{rgb}{0,0,0.8}  % Blue for SQL keywords

\lstdefinelanguage{MySQL}{
    keywords={SELECT, FROM, WHERE, JOIN, ON, INNER, OUTER, LEFT, RIGHT, FULL, GROUP, BY, ORDER, ASC, DESC, AS, COUNT, SUM, AVG, MAX, MIN, DISTINCT, INSERT, INTO, VALUES, UPDATE, SET, DELETE, CREATE, TABLE, PRIMARY, FOREIGN, KEY, DEFAULT, NULL, NOT, CHECK, CONSTRAINT, INDEX, VIEW, MATERIALIZED, PROCEDURE, FUNCTION, TRIGGER, DATABASE, ALTER, DROP, EXEC, IF, EXISTS, UNION, ALL, CASE, WHEN, THEN, ELSE, END, CAST, CONVERT, LIKE, IN, BETWEEN, AND, OR, HAVING, LIMIT, OFFSET},
    sensitive=false,
    morestring=[b]',  % String in single quotes
    morestring=[b]"   % String in double quotes
}

\lstdefinestyle{sqlstyle}{
    backgroundcolor=\color{backcolour},   
    commentstyle=\color{codegreen},  % Comments in green
    keywordstyle=\bfseries\color{codeblue},  % ✅ SQL Keywords in Blue & Bold
    numberstyle=\scriptsize\color{codegray},  % Row numbers in gray
    stringstyle=\color{codepurple},  % Strings in purple
    basicstyle=\ttfamily\footnotesize,
    breaklines=true,
    captionpos=b,
    numbers=left,      % ✅ Enables row numbers on the left
    stepnumber=1,      % ✅ Row numbers increment by 1
    firstnumber=1,     % ✅ Starts numbering at 1
    numbersep=8pt,     % ✅ Increases space between numbers and SQL code
    xleftmargin=3em,   % ✅ Ensures space inside the left border
    frame=single,      % ✅ Keeps a single border (left-aligned)
    framesep=5pt,      % ✅ Ensures space inside the frame
    rulesepcolor=\color{bordercolor},  % ✅ Matches row numbers with left border
    rulecolor=\color{bordercolor},  % ✅ Sets left border color
    language=MySQL  % ✅ Uses SQL keyword highlighting
}

\begin{lstlisting}[style=sqlstyle, caption={Materialized views with aggregates}]
 
CREATE materialized VIEW salessummary
AS SELECT productid,
          region,
          categoryid,
          SUM(salesamount) AS TotalSales,
          Avg(salesamount) AS AvgSales,
          Count(*)         AS NumberOfSales
   FROM   sales
   GROUP  BY productid,
             region,
             categoryid; 

\end{lstlisting}