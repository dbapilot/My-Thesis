% % % % % % % % % % % % % % % % % % % % % % % % % % % % %
% % LaTeX template for bachelor's and master's theses % %

\documentclass[language=en,degree=bachelor, 12pt, book]{./other_pages/mfithesis} % options: language=[de|en], degree=[bachelor|master], [natbib|biblatex], [bibtex|biber]
%\newtheorem{definition}{Definition}[Section]
\usepackage[utf8]{inputenc}
\usepackage{appendix}
\usepackage{microtype}
\bibliographystyle{achemso}
\usepackage[utf8]{inputenc} % Ensures the document can use Unicode characters
%index
\usepackage{imakeidx}
\usepackage{etoolbox}
\usepackage{imakeidx}   % Best indexing package
%\usepackage{idxlayout}  % Ensures better formatting
\makeindex[title=Index, columns=2, options=-s mystyle.ist]
\usepackage[totoc, initsep=15pt, font=small, unbalanced=true]{idxlayout}
%End Index
%\usepackage[T1]{fontenc}    % Output font encoding for international characters
\usepackage[dvipsnames]{xcolor}
\usepackage{csquotes}
\usepackage{lipsum} % This package generates filler text for the example
\usepackage{fncychap}
\usepackage{amsmath}
\usepackage{float}
%\usepackage{chapterbib}

%\usepackage{geometry} % Adjust page margins
%\geometry{a4paper, margin=1in} % Set margins to fit A4 size
\usepackage{natbib}
\usepackage{pgfplots}
\pgfplotsset{compat=1.18}
\usepackage[linguistics]{forest}
\usepackage{rotating}
\usepackage{listings}
\usepackage{xcolor}
\usepackage[normalem]{ulem}
\usepackage[caption=false]{subfig}
\usepackage{url} 
\usepackage[hyphens]{xurl}
\usepackage[hidelinks,bookmarksopen,bookmarksnumbered,breaklinks]{hyperref}
\usepackage{natbib}
%\bibliographystyle{plainnat}
\usepackage{float}
\restylefloat{figure}
\usepackage{placeins}
\usepackage{listings}
\usepackage{graphicx}
%%-Chapter title--%
\usepackage{fancyhdr}
  \fancyhf{}
  \pagestyle{fancy}
  \fancyhead[C]{{\nouppercase\leftmark\nonumber}}
  \fancyfoot[C]{\thepage}
  \renewcommand{\headrulewidth}{0pt}
\fancypagestyle{plain}{

}
%%
\usepackage{scrlayer}
  % Optional for coloring code
%Enumerations
\usepackage{enumerate}
\usepackage{placeins}
\usepackage{enumitem}
\usepackage{acronym}
\usepackage{glossaries}
\usepackage{booktabs, tabularx}
\usepackage{amsmath}
\usepackage{amsthm}
\theoremstyle{definition}
\newcommand{\definition}[1]{\vspace{20pt}\noindent\textit{#1}\vspace{20pt}}


%\newtheorem{definition}{Definitions}
%Hyperlinks

\hypersetup{
    colorlinks=true,
    linkcolor=black ,
    filecolor=magenta,      
    urlcolor=teal,
    pdftitle={Overleaf Example},
    pdfpagemode=FullScreen,
    breaklinks=true,
    colorlinks=true,
    linkcolor=black,
    citecolor=green,
    }
\urlstyle{same}

%-----Python code------
\usepackage{listings}
\usepackage{tcolorbox}
\usepackage{minted}
\setminted{
    bgcolor=white,
    fontsize=\footnotesize,
    frame=single,
    rulecolor=black,
    tabsize=2, 
    breaklines=true,
    escapeinside=||,
    numbers=left, % Add line numbers
    linenos, % Enable line numbers
    numbersep=5pt, % Space between line numbers and code
}
    
\usepackage{caption} % For better captions
\lstdefinestyle{pythonstyle}{
    language=Python,
    basicstyle=\ttfamily\footnotesize,
    keywordstyle=\color{blue},
    stringstyle=\color{red},
    commentstyle=\color{green!50!black},
    backgroundcolor=\color{gray!10},
    numbers=left,
    numberstyle=\tiny\color{gray},
    stepnumber=1,
    breaklines=true,
    frame=single,
    breaklines=true,
    captionpos=b,
    tabsize=4,
    morekeywords={self, None},
    breakatwhitespace=true, % Break lines at whitespace
    showstringspaces=false, % Do not display spaces in strings
    xleftmargin=15pt, % Left margin for better alignment
    xrightmargin=15pt % Right margin for better alignment
}


\lstdefinestyle{sql}{
    language=SQL,
    basicstyle=\ttfamily\footnotesize,
    keywordstyle=\color{blue},
    stringstyle=\color{green!60!black},
    commentstyle=\color{red!60!black},
    frame=lines,
    numbers=left,
    numberstyle=\tiny\color{gray},
    breaklines=true
}
% ----python----
%New colors defined below
\definecolor{codegreen}{rgb}{0,0.6,0}
\definecolor{codegray}{rgb}{0.5,0.5,0.5}
\definecolor{codepurple}{rgb}{0.58,0,0.82}
\definecolor{backcolour}{rgb}{0.95,0.95,0.92}

\usepackage{listings}
%-------
\usepackage[table, dvipsnames]{xcolor}  % Include all the options you need
\usepackage{amsthm}
\usepackage{float}
%Hyperlinks
%\theoremstyle{definition}
%\lstset{style=mystyle}
\usepackage[colorlinks=true, urlcolor=blue, linkcolor=blue, citecolor=blue]{hyperref}
\usepackage{caption}
\captionsetup[lstlisting]{labelfont=bf, labelsep=colon}
\usepackage{array}
\usepackage[table,xcdraw]{xcolor} % For table colors
\usepackage{booktabs} % For professional table styling

\begin{document}
  \author{Md Nazmul Islam}
 \title{Optimizing Query Performance in a Database System}
 \matrno{Matriculation Number: 3072770}
 \course{Computer Engineering, ISE, Vert. Software Engineering }

 \advisor{Dr. Claudia Gotzes}
 \coadvisor{Dr. Robert Martin}
 \externaladvisor{Björn Renner} % optional

 \date{Date of the thesis' completion}
 \logo{logoUDE.pdf}

 \maketitle % generate the title page\\ 

 \declaration
 \newpage
\thispagestyle{empty} % Optional: to not have a header or footer on this page

\begin{center}
    \Large\textbf{Non-Disclosure Agreement}
\end{center}
\noindent This bachelor thesis contains business-secret and confidential data of opta data Finance GmbH. The dissemination, duplication, or publication of the work, as well as the use and announcement of its contents without written permission of opta data Finance GmbH, \textbf{\textit{is not allowed}}. It is made available to members of the examination board only for assessment.\vspace{4cm}

                   \begin{center}
              \color{teal}
                        \large{“Information learned is more valuable than information given.”}\\
                      \color{black} 
                                                 {— Al Mualim, video game Assassin’s Creed (2007)}  
                   \end{center}
 \begin{center}
    \Large\textbf{Acknowledgment}
\end{center}\vspace{1cm}
\normalsize
I am deeply grateful and would like to thank Dr. Claudia Gotzes for her invaluable guidance, immense knowledge, motivation, support and feedback throughout the entire process of researching and writing this bachelor’s thesis. I also offer my appreciation to Prof. Dr. Robert Martin for agreeing to be my co-supervisor.\\
I would also like to show my gratitude to my team at opta data Finance GmbH, particularly Mr. Björn Renner, for his enthusiastic encouragement and valuable instructions as my supervisor and my entire team as well.\\
My deepest gratitude goes to my beloved parents, whose unwavering support and
encouragement have inspired me throughout my academic journey. Thank you for making this academic journey a fulfilling and rewarding experience.

 \newpage
   \tableofcontents{}
   \listoffigures{}
   \listoftables{}
   
\begin{center}
    \fancyhead[]{}\Large\textbf{Abstract}
    \cfoot{\thepage} % add page number in center footer
\end{center}

\normalsize
This thesis emphasizes the improvement of query effectiveness through the discussion of an array of metrics that reflect different aspects of performance, which is essential for improving execution speed and flexibility in large-scale information management, particularly in database systems. This study points to boosting framework query throughput (which refers to the number of queries or transactions processed by the database in each period) and decreasing execution times (query latency). Additionally, the study emphasizes resource utilization,  including CPU,\footnote{CPU, a central processing unit, also called a central processor, main processor, is the most important processor in a given computer.} Memory and I/O during query execution time.\vspace{.4cm}

This thesis also offers query optimization techniques like \hyperref[term:materialized_views]{Materialized views \footnote{Materialized view explain later part on this paper chapter 2.6.}} that can be used to reduce the time to select optimized queries related to views in database systems. Materialized views have been found to be very effective at speeding up queries and are increasingly becoming effective and being supported by commercial databases. Materialized views can provide massive improvements in query processing time, especially for aggregate queries over large tables. To take advantage of this potential, the query optimizer must know how and when to exploit materialized views. This paper presents a fast and scalable algorithm for determining whether part or all of a query can be computed from materialized views and describes how it can be incorporated in transformation-based optimizers.\vspace{.4cm}

The selection of materialized views is one of the most important decisions in designing a data warehouse for optimal efficiency. A suitable set of materialized views minimizes the total cost associated with their maintenance and storage, making it the key component in data warehousing. To solve this problem, the use of the Particle Swarm Optimization (PSO) algorithm is proposed, which helps to select MVs to accelerate workloads efficiently. This paper gives the output of a PSO-based selection algorithm, indicating its effectiveness in improving query performance.\vspace{.4cm}



\noindent \textbf{Keywords:} Database, SQL, ETL, Materialized views, DML, API, Indexing, Latency, Throughput, DDL, Resource utilization, ACID, Performance Enhancement, SPJ, Particle swarm algorithm(PSO).

 


   %\documentclass{article}
%\usepackage{geometry} % To adjust margins if needed to fit the list on one page
%\geometry{left=2cm,right=2cm,top=2cm,bottom=2cm} % Adjust margins as needed

% \begin{document}
\thispagestyle{empty}
\begin{center}
    \huge\textbf{List of Abbreviations}
\end{center}  \vspace{.4cm}

\noindent % This ensures the minipage spans the full text width

\begin{minipage}{\textwidth}
\begin{tabular}{ll}
    \textbf{Abbreviation} & \textbf{Meaning}\\
    SQL & Structured Query Language\\
    ID3 & Iterative  Dichotomizer 3\\
    EKV & Elektronischer Kostenvoranschlag (Electronic cost estimate)\\
    DML & Data Manipulation Language \\
    DDL & Data Definition Language \\
    SPJ & Select-Project-Join \\
    DBMS & Database Management Systems \\
    MV & {Materialized Views} \\
    SPM & Sorted, Projected, and Materialized \\
    QPS & Queries Per Second \\
    TPS & Transaction Per Second \\
    ETL & Extract, Transform, and Load \\
    ACID & Atomicity, Consistency, Isolation, Durability \\
    RAM & Random Access Memory \\
    PSO & Particle Swarm Optimization \\

% Add more abbreviations as needed
\end{tabular}
\end{minipage}

% \end{document}

    
   %\chapter{Introduction and Motivation}
\section{Introduction}
This chapter outlines the motivation behind the thesis and objectives of critical issues faced by enterprises surrounding query performance in modern database systems.
\subsection{Motivation}
The motivation for this thesis is driven by the urgent need to enhance query performance in database systems, which are a core part of many enterprise-level and cloud-based applications. SQL queries are essential for data manipulation and analysis, but they can also be a source of irritation and inefficiency if they are not properly optimized. Despite significant advancements in database technologies, enterprises still face issues such as high latency, poor load balancing, and inefficient data retrieval mechanisms, especially under complex query loads. Addressing these challenges not only improves the user experience but also boosts the overall efficiency of data-driven decision-making processes.\vspace{.4cm}

The collaboration with opta data Gruppe, a company that leverages over 50 years of experience in the healthcare sector to provide tailored digital, financial, and operational solutions for health providers and organizations, provides a unique opportunity to tackle these challenges in a real-world context. Opta data Gruppe has extensive expertise in managing large-scale databases and offers valuable insights and access to proprietary datasets and infrastructure. This partnership will enable the practical application of theoretical concepts and the evaluation of proposed optimizations in a living environment, ensuring that the research outcomes are both scientifically robust and industrially relevant. \vspace{.4cm}

This thesis report seeks to investigate how advanced query optimization techniques could enhance database performance and efficiency within the company's operations. Materialized views are an excellent way to improve the efficiency of complex queries by pre-computing and storing frequently used results over existing methods. Unlike traditional query optimization techniques, implementing materialized views might significantly lead to faster query response times, reduced server load, and improved overall system efficiency for our business, particularly in reporting and analytics, where aggregations and joins across large datasets are typical. Furthermore, by proactively selecting and controlling materialized views based on query frequency and complexity, this approach may boost performance, maximize resource utilization, and ensure a more responsive system, resulting in cost savings and increased productivity compared to conventional optimization methods. This strategy could be especially effective for accelerating decision-making procedures that rely on quick access to critical data insights.\vspace{.4cm}

By leveraging opta data Gruppe resources and industry experience, this thesis aims to develop a deep understanding of the limitations present in current query optimization strategies used in database systems. Design and test innovative approaches (a sophisticated algorithm solution) capable of overcoming the prevalent hurdles of databases. It intends to significantly improve search capabilities by combining the PSO algorithm's selection, crossover, and mutation operators to form decisions that quickly determine the most efficient execution plans for queries that can predict and adapt to changing data patterns while remaining relevant and up-to-date without incurring excessive overhead. In an operational environment, investigate the effects of these improvements on query performance, such as decreased latency and increased throughput. \vspace{.4cm}


%%Recognizing the limitations of traditional query optimizers in distributed settings, this research is driven by the ambition to develop a sophisticated algorithmic solution capable of overcoming the prevalent hurdles of distributed databases. By proposing a unique optimizer architecture and deploying the Iterative Dichotomizer 3 (ID3) algorithm as a query optimizer, we aim to significantly improve search capability. This involves the fusion of selection, crossover, and mutation operators from genetic algorithms to formulate decision trees, which can swiftly deduce the most efficient execution plans for queries.

%Moreover, the thesis aims to innovate beyond conventional methods by introducing a caching strategy that mitigates the time cost associated with processing numerous queries. Through rigorous testing and comparative experiments, the research evaluates the execution costs and convergence speeds of Top-k query plans, highlighting the effectiveness of the proposed solutions in achieving superior query efficiency, albeit with a trade-off in execution time.

%The thesis also ventures into the realm of artificial intelligence by presenting "Neo" (Neural Optimizer), a groundbreaking learning-based query optimizer that capitalizes on deep neural networks. This innovative approach to generating query execution plans represents a leap forward in query optimization, underscoring the transformative potential of machine learning in the field of database management.






\subsection{Problem Statement}
The idea of using materialized views for the benefit of improved query processing has been proposed in the literature for more than a decade \cite{Blakeley1986EfficientlyUM}. The presence of the right materialized views can significantly improve performance, particularly for decision support applications. However, to realize this potential, a reasonable selection of materialized views is crucial \cite{agrawal2000automated}.\vspace{.4cm}

Earlier approaches to query optimization encountered several significant challenges that impacted their effectiveness. One of the main issues was scalability; as the number of materialized views grew, the algorithms often struggled to keep up, leading to delays in processing times in large databases. Additionally, maintaining these views became tricky in dynamic environments where data changes frequently, creating a frustrating trade-off between having up-to-date information and achieving optimal performance. The interactions between materialized views and indices were often less than ideal, resulting in inefficiencies and redundant computations that further hampered overall efficiency. These challenges underscored the urgent need for more flexible and integrated optimization strategies that could enhance both practicality and performance in dynamic settings.\vspace{.4cm}

A database management system(DBMS) is a crucial software component that enables efficient creation, updating, deleting, and retrieving of data stored in databases. In the era of big data, databases have become a cornerstone for handling vast amounts of information across multiple locations. As the backbone of large-scale online applications, such as social media platforms, e-commerce sites, and cloud services, the ability of these systems to efficiently process queries is paramount. The performance of databases directly influences the responsiveness of applications and, by extension, user experience and business operations \cite{4}. Especially in database systems, one of the most important factors related to large databases is query optimization and response time, as well as on-time access to information, which are the basic requirements of successful business applications. A data warehouse implements many materialized views to process a predefined set of queries with speed efficiently. For any database, quick response time and accuracy are very important factors in considering its success \cite{karde2010selection}.\vspace{.4cm}

A necessary condition for the success of a data warehouse is to provide accurate and timely consolidated information for the decision-makers, along with fast query response times. For this purpose, a common method used in practice is the use of higher information and the best concept of response time, whereby a query gets answered quickly. One of the most important decisions in designing data Warehouses is selecting views to materialize for the purpose of efficiently supporting the decision-making. The view selection problem is defined as selecting a set of derived views to materialize that minimizes the sum of total query response time and maintains the selected views. Thus, the goal is to select a good set of views that minimizes the overall query response time and also maintains the selected views. The decision ``What is the best set of views to materialize?'' is to be made based on the system workload, a sequence of queries and updates that exemplifies the typical load on the system. The criterion can be very simple, for example, one that simply minimizes the overall execution time of workload queries.\vspace{.4cm}

In relational databases, a view is a function from a set of based tables to a derived table; the function is recomputed every time the view is referenced. A materialized view, on the other hand, acts like a cache: a duplicate of the represented data that can be accessed efficiently. Therefore, it is evident that the use of materialized views incorporating not just traditional simple SELECT PROJECT JOIN operators but also complex online analytical processing operators contribute significantly to improving online analytical process (OLAP) query performance. Materialized views are used in data warehousing, replication servers, recording systems, and data visualization and mobile systems. In some cases, it can be more advantageous to materialize the view than to have to compute the base tables each time the view is queried. Each time a change is made to the base tables to which the view refers, the materialized view is refreshed. It can be quite costly to rematerialize this view every time a change may affect one of the base tables. So, it is ideal to propagate the changes incrementally; the materialized view should be refreshed for incremental changes to base tables \cite{Data_warehousing,efficient_incremental,rashid2009role}.\vspace{.4cm}

Databases serve many purposes, but they present unique challenges in query performance optimization. These systems need to handle not only large volumes of data but also spatially dispersed data. Such dispersion requires complex coordination and communication among different nodes, which can lead to latency and potential bottlenecks that can degrade query performance. Additionally, network variability and the unpredictability common in a database environment further complicate the effective execution of queries.

\subsection{Overview of Opta data Gruppe}
%\normalsize
\subsubsection{Opta data Stiftung \& Co. KG }
Opta data Stiftung \& Co. KG acts as the umbrella organization of the opta data Group, which brings together relevant areas for the entire company. This also includes working Students and trainees, although they work and can be trained in various specialist areas. Opta data Holding is a company that specializes in billing, IT, and healthcare services.

\subsubsection{Opta data Finance GmbH  }
Opta data Finance GmbH (referred to as odFIN) is part of opta data Holding and is one of the leading companies in the healthcare billing sector. With a wide range of products, odFIN offers various solutions for service providers and payers in the healthcare sector. As an innovator, the company is actively driving digitalization in the healthcare sector and occupies a leading position in the telemetry infrastructure.

\subsubsection{Business Area egeko }
The egeko division at odFIN develops and supports software systems for electronic approval procedures in the healthcare sector. With a focus on electronic cost estimates (eKV), the egeko division offers innovative solutions with the software of the same name for service providers in the medical aid and care sector, among others. The company facilitates processes between service providers and health insurance companies by supporting the entire service provision process.\vspace{.4cm}

The division's software development is made up of a total of three scrum teams and two additional Kanban teams. The scrum teams focus on development with a focus on central services, aids, care, and master data and patient transport. DevOps and quality assurance provide support.\vspace{.4cm} 

As a part of egeko, team DevOps is a cross-functional group that combines software development (Dev) and IT operations (Ops) roles, aiming to create a more efficient and integrated approach to building, testing, and releasing software. Key characteristics of a DevOps team include automating everything: deployment, infrastructure, test, build, scale, promoting fault tolerance, server monitoring, and static code errors that do not directly reduce software quality from the customer's point of view. However, error prevention blocks the CI/CD pipeline's agile approach. A platform can be migrated anywhere and at any time. Further, CIPs are used to make everyday life easier, avoid click tasks, and create mutual trust in each other and old or new technologies. Active knowledge transfer, definition of consulting measures, support and introducing people to new technologies No fear of new technologies, first evaluate then judge. Create shared visions and responsibility, and realize wishes. A start-to-finish responsibility, everyone is involved can get involved and strengthen collaboration.\vspace{.4cm}

The egeko software is used by third-party customers. To ensure that customer requests are fulfilled and faults are rectified, there is a team responsible for customer support. The customer support team has a ticket database that records messages and faults from customers.\vspace{.4cm}

I have finished my internship in the DevOps team in the egeko division. The DevOps team is part of the development of egeko. I have been assigned the task of continuing the optimization of egeko's database infrastructure, which has given better results for the existing server in Opta Data Finance GmbH. I must do the following Tasks: Monitoring database performance, server health checks, conducting regular performance tuning, and optimizing queries for maximum efficiency. Manage and optimize healthcare databases to ensure the availability and reliability of critical organizational website actual and future ETL processes to ensure optimization and best practices. Develop and implement data security policies, procedures, and best practices to protect sensitive healthcare information. The area of my key activities included the following tasks:\vspace{.4cm}
\begin{itemize}
    \item Optimization of database infrastructure.
    \item Database design, management, planning, and tuning.
    \item Performance analysis, monitoring, and alerting.
    \item Query optimization and maintained security management.
    \item Development and implementation of backup strategies.
    \item Disaster recovery and documentation.
\end{itemize}

I used most of Microsoft SQL Server Management Studio (SSMS), VSCode, Grafana (a multi-platform open-source analytics and interactive visualization web application), mRemoteng, and MSSQL as programming languages.\vspace{.4cm}

The issues that opta data Gruppe faces in managing healthcare data are closely related to this thesis, as fragmentation of data information across numerous systems makes it difficult to achieve a unified view of records. As the volume of data increases and changes frequently, the time taken to retrieve required information can significantly impact user experience. Users expect quick and up-to-date access to data, and any delays can be frustrating and decrease productivity. Slow response times can cause bottlenecks, blocking other queries and further slowing down overall performance. Optimizing query performance by prioritizing query response time opta data Gruppe directly contributes to improved performance and user satisfaction.

\subsection{Goal of this Thesis}
\normalsize
This thesis conducts a thorough review of the literature on cost estimation (Query performance ) within relational databases, a critical component of the query optimization process. For this systematic review to be effective, precise objectives must be established. These principal aims are outlined in the study \cite{CostEstimation}. The following key points are the goal of this work: 
\begin{itemize}
  \item To provide an overview of the materialized view, an efficient query optimizer in databases with the details of the approaches.
  \item To provide a comprehensive and efficient solution.
  \item To find the limitations of the approaches.
  \item To understand the scope for further research, which contributes towards building
better query Performance. %\cite{CostEstimation}.
\end{itemize}
\subsection{Structure of the Thesis }
The following sections focus on the objectives of the systematic literature. In section 2, we will provide a background of materialized views on databases. It includes an overview of query processing, query optimization, and cost estimation.\vspace{.4cm}

In Section 3, the Methodology section, we will discuss the systematic approach to investigate and implement materialized views for query optimization in a database system, including the specific steps, tools, and techniques used to demonstrate the effectiveness of materialized views.\vspace{.4cm}

At the end of the thesis, we will discuss limitations and future work related to this topic that can be conducted.











    %\chapter{Background}
\section{Background \& Theoretical Basics }

In this chapter, we provide high-level explanations of the most important aspects and comprehensive understanding of the foundational theories and principles that make query optimization and materialized views effective. The concrete basics needed are explained the respective section along the thesis.

%some basics of the Database particularly distributed database.Then an overview of materialized view as query processing technique. later,we will discuss various methods and algorithms to about materialized view and Finally we discuss related work.
\subsection{ Database System}

\begin{definition}
A database is an organized collection of data or a type of data store based on the use of a database management system(DBMS), the software that enables the creation, modification and the database itself to capture and analyze the data.\end{definition}\vspace{.4cm}
A database in SQL Server is made up of a collection of tables that stores a specific set of structured data. A table contains a collection of rows, also refereed to as records or tuples and columns also referred to as attributes. Each column in the table is designed to store a certain type of information, for example: dates, names, dollar and numbers.\cite{williamdassafmsft-2024} Scalability, security, high availability, Network latency, fault tolerance these are the key characteristics of databases. There are two most distinct types of databases: Relational and NoSQL database.

\begin{itemize}
    \item \textbf{Relational databases(RDBMS) :} The relational database model came about 1969-70 as a solution for dealing with the variety of custom designed DBMSs that were used, prior to 1969. These databases store the data in structured tables with rows and columns and use for SQL for querying. They are known for their robustness, flexibility, and support for ACID properties. Example: Microsoft SQL server, MySQL, PostgreSQL, Oracle database.\cite{editor-2024,foote-2023}
    
    \item \textbf{NoSQL databases:} NoSQL, also referred to as "not only SQL", emerged in the late 2000s, is an approach to database design that enables the storage and querying of data outside the traditional structures found in relational database. It offers flexibility and scalability, making them particularly useful for handling large amounts of unstructured data in real time applications. They support horizontal scaling allowing for increased storage and processing capabilities as data volume grow. Example: MongoDB, Redis, CouchDB.\cite{ibm-2024,justacademy_nosql_characteristics}
\end{itemize}

\subsection{Query Processing }
A query is a request sent to a database for data retrieval. Specific conditions are passed in a query to match and retrieve relevant data. SQL (Structured Query Language) is used to write these queries to extract information from relational databases. The query process involves translating high-level queries into low-level expressions suitable for the file system, optimizing the query, and executing it to obtain the result. The steps involved in the execution of a query are as follows:\cite{wwwnaukricom-no-date}\\
\begin{figure}[h]
    \centering
    \includegraphics[width=0.5\textwidth]{Figure/Flow of QueryProcessing.jpg}
    \caption{The flow of query processing in DBMS.}
    \label{fig:my_image}
\end{figure}
\begin{enumerate}
\item \textbf{Parser:} Query parsing is the first step in query processing. In this steps, database performs the following checks- Syntax,semantic and shared pool check,after converting the query into relational algebra.\cite{wwwnaukricom-no-date}
    \begin{enumerate}
        \item \textbf{Syntax check:} A query is checked for syntax error.It concludes syntactic validity. Example:\\
        \input{SQL/Query1}
          Here error of wrong spelling of FROM is given by this check.
        \item \textbf{Semantic check:} It checks whether the statement is meaningful or not. Example: Query contains a table name which doesn't exist\\
        \input{SQL/Query2}
        A syntactically correct statement can fail a semantic check, as shown in the following example of a query of a nonexistent table.\cite{Oracle}
        \item \textbf{Shared pool check:} During the parse, the database performs a shared pool check to determine whether it can skip (hash code )resource-intensive steps of statement processing. Every query posses a hash code..
        
\definecolor{dkgreen}{rgb}{0,0.6,0}
\definecolor{gray}{rgb}{0.5,0.5,0.5}
\definecolor{mauve}{rgb}{0.58,0,0.82}
\lstset{language=SQL,
  basicstyle={\small\ttfamily},
  belowskip=3mm,
  breakatwhitespace=true,
  breaklines=true,
  classoffset=0,
  columns=flexible,
  commentstyle=\color{dkgreen},
  framexleftmargin=0.25em,
  frameshape={}{yy}{}{}, %To remove to vertical lines on left, set `frameshape={}{}{}{}`
  keywordstyle=\color{blue},
  numbers=none, %If you want line numbers, set `numbers=left`
  numberstyle=\tiny\color{gray},
  showstringspaces=false,
  stringstyle=\color{mauve},
  tabsize=3,
  xleftmargin =1em
}
         \begin{lstlisting}
ALTER SESSION SET OPTIMIZER_MODE=ALL_ROWS;
ALTER SYSTEM FLUSH SHARED_POOL;               # optimizer environment 1
SELECT * FROM sh.sales;

ALTER SESSION SET OPTIMIZER_MODE=FIRST_ROWS;  # optimizer environment 2
SELECT * FROM sh.sales;

ALTER SESSION SET SQL_TRACE=true;             # optimizer environment 3
SELECT * FROM sh.sales;

        \end{lstlisting}
        In this example, the same \textbf{SELECT} statement is executed in three different optimizer environments. As a result database creates three separate shared SQL areas for these statements and force a hard parse of each statement.\cite{Oracle}
    \end{enumerate}.\\
\begin{figure}[h]
    \centering
    \includegraphics[width=0.5\textwidth]{Figure/Query processing.jpg}
    \caption{The flow of query processing in DBMS}
    \label{fig:my_image}
\end{figure}
    
    \item \textbf{Optimizer}: After parsing the query, the DBMS starts to find the most efficient way of executing the provided query. The factors for the query follow some optimization process. During the optimization stage, at least one complex parsing of one unique DML statement must be done. The database never optimizes DDL unless it includes a DML component, such as a sub-query, which needs optimization. Such operations are used for selecting data, inserting something, updating, etc. After everything is completed, then the evaluation step is made. in this step, the result is returned by the DBMS. This result is displayed to you in an appropriate format.
    \item \textbf{Result:} After getting the best execution plan, the DBMS starts the execution of the optimized query and perform the operation on data including selecting the data, inserting something, updating the data.\\
    Once everything is completed, DBMS returns the result after the evaluation step. This result is shown to you in a suitable format.\cite{Query,QueryProcessing,Oracle}
\end{enumerate}

\subsection{Query Optimization }When we write code, we aim for optimal logic in terms of both space and time complexity. Similarly, when we write database queries, we want them to be optimal in terms of their execution time and resource utilization's. Query optimization is  a crucial aspect of DBMS that seeks the most efficient way to execute a given query by considering a variety of query execution strategies. It minimizing the total cost or the total response time for the execution of a query.It is one of the factors that affect the application performance.\vspace{.4cm}

The result of a query is generated by processing the rows in a database in a way that yields the requested information. Since database structure are complex,in most cases and especially for not very simple queries, the needed data for a query can be collected from a database by accessing it  in different ways through different data structure and in different orders\cite{selinger-1979}. Each different way typically requires different processing time. Processing time of the same query may have large variance, from a fraction of a second to hours, depending on the chosen method. The purpose of a query optimization is to find the way to process a given query in minimum time, the large possible variance in time justifies performing query optimization, though finding the exact optimal query plan among all possibilities, is typically very complex, time consuming by itself may be too costly, and often practically impossible. Thus query optimization typically tries to approximate the optimum by comparing several common-sense alternatives to provide in a reasonable time.\vspace{.4cm}

There is a trade off between the amount of time spent figuring out the best query plan and the quality of the choice, the optimizer may not choose the best answer on its own. Different qualities of database management systems have different ways of balancing these two. Cost based query optimizes  evaluate the resource footprints of various query plans and use this as the basis for plan selection.These assign an estimated cost to each possible query plan and choose the plan with the smallest cost. Cost are used to estimate the runtime cost of evaluating the query, in terms of number of I/O operations required, cpu path length, amount of disk buffer space, disk storage service time, and interconnect usage between units of parallelism and other factors determined from the data directory. The set of query plans examined is formed by examining the possible access paths e.g. primary index, secondary index access, full file scan and various relational table join techniques e.g Merge join, Has join, Product join. the search space can become  quite large depending on the complexity of the SQL query. There are two types of optimization. These consist of logical optimization-which generates a sequence of relational algebra to solve the query-and physical optimization-which is used to determine the means of carrying out each operation.\cite{dremio-2024}


\subsubsection{Database query performance Metrics}
To effectively measure the performance of SQL queries, the following metrics are playing vital role. But most relevant metrics may vary depending on the specific database system and application requirements.
Here is a summery of key metrics used to evaluate and enhance query performance.\cite{chwesewicz-2024}
\begin{itemize}
    \item\textbf{Query execution time}: Total query duration to execute from start to finish. Measured in seconds or milliseconds. Lower execution time indicates better performance.
    \item\textbf{Query throughput}: It measures the number of queries or operation a database can handle per unit time, typically expressed as transaction per second (TPS) and queries per second(QPS).
    \item\textbf{Resource utilization}: Measures the percentage of time the CPU is occupied processing database operations. High CPU or memory usages can indicate heavy processing load or un-optimized queries.
\end{itemize}
\subsubsection{Key optimization techniques:}
 \begin{itemize}
     \item \textbf{Indexing}: Indexing is a crucial technique for query optimization in databases. It's named indexing because of how an index works in a book. An index is a structure that holds the field the index is sorting and a pointer from each record to their corresponding record in the original table where data is actually stored.\cite{tomar-2021,atlassian-no-date}
     \item \textbf{Query rewriting}: It is a technique that used in query optimization to transform a given database query into an equivalent form that executes more efficiently. It is one of the initial phases of query processing where original query is parsed and translated into an internal representation. This method particularly useful for complex queries, including those queries that have many sub queries or many joins.\cite{pitoura-2009,unknown-IBM-25-2024}
     \item \textbf{Partitioning}: This optimization method is specially  effective for large database. It involves splitting a large table or index into smaller segments to make the query more manageable pieces. Each partition acts as a separate entity that can be managed independently.\cite{planck-2024} 
     \item \textbf{Materialized view}: Materialized views are a powerful tool for query optimization, which we will focus on. 
 \end{itemize}
 
 \subsection{Reasons for Using Materialized Views for Query Optimization}:
\begin{itemize}
    \item\textbf{Precomputed result:} Materialized views store the results of a query that allows subsequent queries to access these precomputed results directly rather than recalculating them from the base tables. These results are updated periodically or on demand based on the underlying data changes.\cite{khan-2023,Risingwave-no-date}
    \item\textbf{Reduced Query complexity:} The main reason for creating materialized views is to improve query performance. It stores a snapshot of the data, that reduces the need for intricate query design, as we get the accesses precomputed results directly.\cite{Risingwave-no-date,Databricks-no-date}
    \item\textbf{Efficient use of resources:} As we  get the data from precomputed data and don't need to run full query every time materialized views decrease the computational load on database servers. It leads faster query response time and improved overall system performance and required fewer resources.\cite{google-no-date, khan-2023}
    
\end{itemize}\vspace{.4cm}

The motivation for using materialized views is to improve performance but the overhead associated with materialized view management can become a significant system management problem. The common materialized view management activities include: identifying which materialized view to create, indexing the materialized view; ensuring that all materialized views and materialized view indexes are refreshed properly each time the database is updated; checking which materialized views have been used; determining how effective each materialized view has been on workload performance; measuring the space being used by materialized views; determining which existing materialized views should be dropped,  archiving old detail and materialized view data that is no longer useful.\cite{Ashadevi2008CostEA,1363763}

\subsubsection{Challenges in query optimization} Query optimization is a crucial aspect of database management, aiming to improve the performance and efficiency of SQL queries.
\begin{itemize}
    \item\textbf{Data Fragmentation and Localization}: Dealing with how data is partitioned and distributed across multiple nodes. Data can be fragmented both horizontally and vertically and spread across nodes. Balancing between local execution and data transfer is also challenging. 
    \item\textbf{Query Decomposition and Allocation}: It refers to the process of breaking down a query into sub queries and assign them to different nodes.Challenging to choose best sub queries and node.
    \item\textbf{Complexity of Query Execution Plans}: Query optimization involves various techniques. Mastering these techniques can be challenging to find the most efficient one, especially for complex or multi-table queries.
    \item\textbf{Dependency on Indexes}: Proper indexing improves query performance but failing to create or maintain supporting indexes can result in inefficient query executions. Managing and updating indexes as data grows and changes can become a challenge.\\
    \cite{team-2020,etutorials-03-2024,editor-ijmter-2015}
\end{itemize}
\subsubsection*{Optimization Goals:}

\begin{itemize}
    \item Minimize response time
    \item Minimize resource consumption
    \item Minimize time to first tuple
    \item Maximize throughput
\end{itemize}\vspace{.4cm}

Expressed during optimization as a cost function. The common choice is to minimize response time within given resource limitations.



\subsection{What is a view in SQL:}
Views are perspective on a database. A view provides data of a database to a client and simultaneously prevents client access to the original database tables. When discussing views, the database tables always referred to as the view tables. View tables can reproduce the data of a database, they can provide a specific selection of data or they can compute new data out of the base table contents. Basically any analytical operation that can be derived from base table data can also be represented in a view table.\vspace{.4cm}

A view in SQL is a a virtual table that is generated by an SQL query. It does not store data physically but retrieves it from the underlying base table.Views are compiled at runtime, and they simplify the presentation of data from one or more tables without modifying the original data. It can be made over one or more database tables. Generally, we put those column in view that we need to query again and again. Once we created a view, we can make index, trigger on the view and query the view as table. A view may act as a filter on certain tables being referenced in the view \cite{chauhan-2024,Rohan_Vats-2024}

\subsubsection{Types of Views:}

There are two types of view in SQL server
\begin{itemize}
    \item \textbf{System defined view }: The system defined views are predefined views that already exists in the  master database of SQL server, such as tempdb, master and temp. Each of the database has its own properties and function. These system views will be automatically attached to any user-defined database.It will expose the metadata of the database and they can be used to get all possible information about the instance of SQL server or database objects, columns and contains. There are three types of System defined views, Information Schema, Catalog View, and Dynamic Management View. \cite{chauhan-2024}
    \item \textbf{User defined view }: This are the types of views that are defined by the user. User defined these view to meet their specific requirements.It can also divide into three types such as simple,complex and materialized views.\cite{javapoint-author-2024}
\end{itemize}
   

   \section{Methodology}\vspace{.4cm}
This chapter will focus on the materialized view selection processes, implementation strategies, and best practices for creating and maintaining MV. It ends with highlights of the performance improvements, such as query execution time and scalable systems, that make materialized views important in modern database systems.

 \subsection{How Materialized Views Work on MSSQL:} Not every database supports materialized views, and those that do each handle them a little differently, especially when it comes to the approach to view maintenance \cite{hattemer-2020}. Microsoft SQL Server supports materialized views. Still, they are called "indexed views" because a materialized view may be indexed in multiple ways, and a materialization step is a matter of creating an index on a regular view. A view is materialized by creating a unique clustered index on an existing view. Uniqueness implies that the view output must contain a unique key. An indexable view must be defined by a single-level SQL statement containing selections, inner joins, and optional group-by \cite{goldstein-2001}.\vspace{0.8cm}

 To create materialized views in mssql, we must follow a step-by-step creation procedure:
 \begin{enumerate}
     

     \item\textbf{Identify the query or queries:} The most vital step in materialized view creation is detecting the exact query or queries that will form the view. Hence, they must be fine-tuned to ensure that the required data is fetched optimally and the performance of view generation is optimized. Queries should consider the complexity of determination, the size of their result sets, and the frequency of data updates \cite{castordoc2023}.
     
      \item\textbf{ Write the base query :} Begin by writing the base  SQL query that defines the materialized views. It should contain all necessary data, join, aggregation, or other complex operations that need to be optimized. \vspace{0.4cm}
      \input{SQL/Base_query}
      
      \item\textbf{ Create the materialized view :} After writing the query, proceed to create a materialized view (indexed view) in SQL server using the `` with schemabinding'' clause starting with schema and view name. Schema binding options bind the view to the schema of the underlying tables, preventing changes to the base tables from affecting the view's definition \cite{risingwave2024}.\vspace{0.4cm}

      \input{SQL/MV_on_mssql}

      \item\textbf{ Create a unique clustered index:} This step is very crucial for the SQL server to materialize as it organizes and stores the sorted data of the materialized view based on the indexed column and transforms it into an MV. \vspace{0.4cm}
      
      
\definecolor{dkgreen}{rgb}{0,0.6,0}
\definecolor{gray}{rgb}{0.5,0.5,0.5}
\definecolor{mauve}{rgb}{0.58,0,0.82}
\lstset{language=SQL,
  basicstyle={\small\ttfamily},
  belowskip=3mm,
  breakatwhitespace=true,
  breaklines=true,
  classoffset=0,
  columns=flexible,
  commentstyle=\color{dkgreen},
  framexleftmargin=0.25em,
  frameshape={}{yy}{}{}, %To remove to vertical lines on left, set `frameshape={}{}{}{}`
  keywordstyle=\color{blue},
  numbers=none, %If you want line numbers, set `numbers=left`
  numberstyle=\tiny\color{gray},
  showstringspaces=false,
  stringstyle=\color{mauve},
  tabsize=3,
  xleftmargin =1em
}
         \begin{lstlisting}
CREATE UNIQUE CLUSTERED INDEX index_name
ON schema_name.view_name(column1, column2, ...);
        \end{lstlisting} 

This statement materializes the view and stores the result in a clustered index.
      
      \item\textbf{ Configure refresh options:} Finally, it is crucial to refresh the materialized view periodically to ensure it remains up-to-date. This option needs to be configured to decide whether manual or automatic refreshes will be preferred. Manual refresh is updated at the user's discretion. On the other hand, automatic refresh occurs at scheduled intervals. The frequency of refreshing the materialized view depends on the rate of data updates and the requirements of the application \cite{castordoc2023}.\vspace{0.4cm}
      
      \item\textbf{ Querying the MV:} Once created, configured, and populated, an Mv can be queried just like a table based on cost and necessity.  \vspace{0.4cm}
      

\end{enumerate}

\subsection{Deciding When to Create a Materialized or a Regular View}There are some key factors to consider when deciding to create a materialized view or a regular view.

%\usepackage[a4paper, margin=1in]{geometry}
\begin{table}[h!]
  \centering
  \caption{Deciding between regular and materialized view}\vspace{.4cm}
  \label{tab:view-comparison}
  \resizebox{\textwidth}{!}{
    \begin{tabular}{|p{0.18\textwidth}|p{0.40\textwidth}|p{0.42\textwidth}|}
      \hline
      \textbf{Requirement} & \textbf{Regular view} & \textbf{Materialized view} \\
      \hline
      Complex Queries & Not ideal, recalculates each time, useful for ad-hoc queries & Suitable, precomputes results for faster querying \\
      \hline
      Real-time Data & Suitable for the most current data & Useful when data is accessed frequently and updated infrequently. \\
      \hline
      Performance & May slow down with complex queries & Faster performance with precomputed data. \\
      \hline
      Database Size & May slow down with large datasets & Suitable for large datasets as they reduce query latency \\
      \hline
    \end{tabular}}
\end{table}



 \subsection{Cost Model}
There is a cost associated with the jobs that maintain them. This section will discuss the cost model of Materialized view.\vspace{.4cm}

 \begin{enumerate}[label=\alph*)]
    \item \textbf{Computing cost :} Computing cost mostly depends on the workload and data size. Query processing and data transferring, there are refresh mechanisms to stay accurate. All these procedures are costly. Both the initial creation and subsequent refreshes of materialized views consume compute resources.
    
    \item \textbf{Maintenance cost:} These costs are associated with keeping Materialized views up-to-date and on service as base data changes. Every refresh or recalculation uses compute resources. Regular tuning, indexing, and monitoring to ensure the view optimally benefits query performance. The frequency of updates to base tables and the complexity of the materialized view impact these costs. 
    
    \item \textbf{Storage cost}: Materialized views require additional disk space, particularly for large databases. The gain in query performance should offset the storage cost of MV. Disabled views still incur storage costs, even if they're not maintained or used for query optimization. As data volume grows, the storage costs for MV can increase significantly. This aspect should be considered when designing a long-term strategy for MV. 
     
    \item \textbf{Usage cost:} Materialized views are great; however, there is a cost associated with the jobs that maintain them. We can calculate the cost using the following formula:\cite{10.1145/2206869.2206874}

    % Total Cost Equation

% Total Cost Equation

\begin{equation}
C_{total} = \sum_{i=1}^{n} \left( S_{query,i} - (C_{maint,i} + C_{storage,i} + C_{fresh,i}) \right)
\end{equation}

\subsection*{Where:}
\begin{itemize}
    \item $C_{total}$: Total cost after considering all savings and expenses.
    \item $S_{query,i}$: Query execution savings for the $i^{th}$ query.
    \item $C_{maint,i}$: Maintenance cost for the $i^{th}$ query.
    \item $C_{storage,i}$: Storage cost for the $i^{th}$ query.
    \item $C_{fresh,i}$: Freshness impact cost for the $i^{th}$ query.
    \item $i$: Index representing each query or instance.
    \item $n$: Total number of queries or instances.
\end{itemize}



  Suppose opta data Group runs a complex query with a materialized view, which takes 1 ms to execute, compared to 5 ms without MV. If it runs 100 times per day, the total savings would be:

  \[
(5 - 1) \times 100 = 400 \text{ ms, which is valued at } \$400 \text{ per day.}
\]

Costs involved:
\begin{itemize}
    \item Maintenance cost for refreshing = \$30
    \item Storage cost for keeping MV = \$20
    \item Estimated correction cost = \$10
\end{itemize}

Total saving calculation:

\[
\text{Total Saving} = 400 - (30 + 20 + 10) = \$340
\]

This means the net benefit for using the Materialized View (MV) is= \$340
  
\end{enumerate}




\subsection{ Static vs Dynamic View Selection } The selection process for materialized views in query optimization can be classified into static and dynamic. Both alternatives are designed to improve query performance but differ in timing, adaptivity, and resource management. Below is the static and dynamic selection process breakdown:
\begin{enumerate}
        \item \textbf{Static View Selection:} Before executing any query, the process of selecting the static view would involve the materialized views that were selected in the design or setup phase of a database. Whether or not to create and keep particular materialized views is dependent on what has been analyzed historically, workload patterns, and domain knowledge. Views are predefined and fixed until the database reconfiguration is done manually.\vspace{.4cm}
    
    \textbf{Key Characteristics:} In this case, the fixed view predetermines materialized views at the design stage, generally deriving the set predetermined from historical analysis of expected workloads. Once chosen, these remain set unless some manual alteration is applied. This is suitable for the predictable nature of stable workloads where queries are not so prone to changes over time, such as data warehouse environments or environments that perform online analytical processing (OLAP). By saving time spent on choices, static view selection becomes quite easy to implement and run. It has an inherent limitation because any considerable shift in the workload would entail manual intervention in the set of materialized views \cite{lohman2000selftuning,mamoulis2012survey,gupta2002selftuning}.
    \item \textbf{Dynamic View Selection:} Dynamic View Selection is the selection of dynamic materialized views at execution time or the adaptation of a set of materialized views to changing workload patterns. This is thus a continuous process consisting of monitoring query patterns and data modification in conjunction with cost-benefit considerations to decide whether to add a materialized view, maintain one, or remove it.\vspace{.4cm}
    
    \textbf{Key Characteristics:} On the contrary, dynamic view selection automatically selects, creates, and keeps materialized views updated according to the real-time analysis of workload. This monitoring approach continuously studies the pattern of incoming queries and applies a sophisticated algorithm to decide which view should either be created, dropped, or modified. Consequently, it is well-suited to environments with unpredictable or rapidly changing workloads, such as in transactional databases or mixed workloads requiring both operational and analytical capabilities. Dynamic nature is what makes this technique effective in response to workload shifts, with selected materialized views being optimized for current usage patterns. On the contrary, real-time adaptability comes with complexity in implementation and increased maintenance overhead because it requires continuous monitoring and decision-making processes. Dynamic view selection necessitates a lot of cost trade-offs for computation of maintaining and refreshing views against the benefits it would have accrued in processing query performance \cite{lohman2000selftuning,mamoulis2012survey,gupta2002selftuning}.
\end{enumerate}

\subsection{ Materialized View Management and Selection Approach}

A well-structured and systematic selection approach is needed to optimize query performance using materialized views in MSSQL. Below is a detailed overview of the materialized views selection approach for query optimization.

\subsubsection{Materialized View Selection Approach}
  
\begin{enumerate}[label=\alph*)]
    \item \textbf{Query identification:} This is the first step in selecting materialized views to identify which queries are most frequently executed and resource intensive. The prime queries involve complex joins, aggregation, or large datasets.
    
    \item \textbf{Analyze query patterns:} This step involves analysing execution logs to determine:
    
     \begin{itemize}
          \item How often specific queries are run.
          \item Resource consumption and average execution time.
          \item The types of operations involved (e.g, joins, aggregations)
      \end{itemize}
      This analysis helps to prioritize which queries should be materialized.
      
    \item \textbf{Storage management:} Access the storage requirements for maintaining materialized views, considering CPU time, I/O operations, and memory usage when determining which materialized views will provide the greatest performance benefit.
    
    \item \textbf{Refresh Policies:} Once queries are identified, design and  Establish refresh policies based on data volatility or requirements.
\end{enumerate}
\subsubsection{View Matching for Query Optimization }

View matching is a critical process in query optimization that determines whether a query or sub-expression can be computed from existing materialized views. Since the objective of query optimization is to minimize the computational cost of query execution by taking advantage of precomputed results, this is an essential component. The view matching typically involves the following steps:\vspace{.4cm}

  \begin{itemize}
      \item \textbf{Query Decomposition:} The entering query is reviewed and broken down into its constituent elements, which include projection, selection, aggregation, and joins. This stage aids in determining the information required by the query and the components available in materialized views \cite{theodoratos2000decomposition}.

      \item \textbf{Normalization and canonical form:} The incoming query and the materialized view definitions transform into a canonical form, facilitating a more straightforward comparison. This normalization aids in removing discrepancies in the expression of similar operations, thereby enhancing the efficiency of matching.

     \item \textbf{Checking coverage:} The attributes and conditions of the query are assessed against those present in the materialized view to ascertain if the view encompasses all required data. This procedure guarantees that the materialized view encompasses all necessary data as specified by the query. In instances where the query includes a WHERE condition, the materialized view must contain an equivalent condition or one that is more restrictive.
     
      \item \textbf{Predicate Subsumption:} The predicates within the materialized view are evaluated to ascertain whether they subsume the predicates of the incoming query. The selection conditions in the query must be either encompassed by or more stringent than those in the materialized view. If the materialized view's predicates are more general, it remains applicable to the query by implementing further filtering as required \cite{adali1996query}.

      \item \textbf{Query Rewriting:} Should the materialized view be determined to encompass the incoming query, the query is subsequently reformulated to utilize the materialized view rather than visiting the basis tables. This rewriting entails adjusting the original query to conform to the schema of the materialized view, sometimes incorporating supplementary filters or projections as necessary \cite{haldar2001query}.

         \item \textbf{Cost Evaluation:} If several materialized views correspond to the query, the query optimizer assesses the cost of utilizing each available view and chooses the one with the lowest predicted cost \cite{hulgeri2001cost}.
This stage guarantees the selection of the most efficient execution plan, optimizing aspects such as I/O, memory use, and processing duration.

  \end{itemize}
  

\subsubsection{Conditions that Must be Fulfilled to be Capable of Using Materialized Views:}
To effectively utilize the materialized views, certain conditions must be fulfilled. These ensure that the query optimizer can create, maintain, and effectively use the MV. Here are the key conditions that must be met:

\begin{enumerate}[label=\alph*)]
    \item \textbf{Stable base table:} The underlying table should have a table schema and not experience frequent changes. It can increase the maintenance cost. All columns referenced in the view must belong to base tables.
    
    \item \textbf{Syntax and structural requirements:} The view must be created using the schema binding options to ensure schema consistency. A unique clustered index is mandatory to materialize the view. Insert, update, and delete operations on base tables must not violate the constraints of the view.
    
    \item \textbf{Sufficient storage and permission:} Adequate storage and permission must be available for the user to accommodate the data stored in materialized views.
    \item \textbf{Effective refresh Strategies:} A clear strategy for refreshing materialized views is necessary to ensure data accuracy while avoiding excessive maintenance overhead.
    \item \textbf{Assessment:} It is essential to periodically evaluate materialized views (MVs) to keep them accurate and aligned with the evolving needs of database systems. Clear documentation should be maintained for each materialized view, detailing its purpose, structure, and refresh policies.
\end{enumerate}

% Three types of materialized views used to increase query performance and reduce response time are as follows:

%%\begin{enumerate}[label=\alph*)]
   % \item \textbf{Materialized view management task}
    %\item \textbf{Materialized view selection}
    %\item \textbf{Incremental Materialized View Maintenance}
%%\end{enumerate}

%\subsubsection{Multiple View Processing Plans }
%\subsubsection{Do All Required Rows Exist in the Views }
\subsubsection{How Particle Swarm Optimization Algorithm Works on MV}
%\subsubsection{Maintenance of Existing Materialized Views }


\textbf{......\textbf{Work on Progress}..........}       
 

   %\chapter{Implementaion}

\section{Implementations}
This section describes a step-by-step process for effectively deploying materialized views, and PSO algorithm based on the earlier theoretical foundations, to reduce query execution time, minimize computational overhead, and enhance overall performance. The implementation begins with an overview of the tools and technologies utilized in this project, providing essential context before delving into the optimization process.

\subsection{Used Software and Tools}
This project uses some software tools and technologies related to database management, query optimization, and technical documentation. Collectively, these tools support database management, query optimization, data analysis, and technical documentation throughout the research and development process.

\begin{enumerate}[label=(\roman*)]
\item\textbf{SQL Server Management Studio :} SQL Server Management Studio(SSMS) is actually an integrated environment that can maintain the SQL Server infrastructure. It is used to access, manage, configure, administer, and develop all components of SQL Server. In addition, it is used to manage the schema, tables, access to SQL Server, and materialized views of the database. Also, it monitors the performance of the query using execution plans and statistics, and it helps to debug SQL queries and optimize their execution. Here, MSSQL is used to host the database, create tables, and define materialized views, enabling query execution and performance measurement.


\item\textbf{{Visual Studio Code:}} Microsoft created this open-source integrated development environment (IDE) for web browsers, Linux, macOS, and Windows. It is used to write Python scripts for automation (e.g., measuring query performance), integrate MSSQL queries with Python using extensions, and debug Python and SQL scripts. VS Code is used as the primary Integrated Development Environment (IDE) for writing, debugging, and running Python scripts that implement the PSO algorithm and interact with the SQL Server database.

\item\textbf{Overleaf:} Overleaf is an open-source online, real-time collaborative LaTeX\footnote{LaTeX is a powerful typesetting system commonly used for academic and technical documents. It includes features designed for the production of technical and scientific documentation.} editor that simplifies the process of creating, editing, and collaborating on LaTeX documents. The whole project is written with the help of Overleaf.

\item\textbf{Microsoft SQL Server:} Microsoft SQL Server is a relational database that provides a wide range of features for storing, processing, and securing data. SQL Server hosts the database, stores the tables and materialized views, and executes the SQL queries, allowing the measurement and optimization of query performance using the PSO algorithm.


\item\textbf{Python:} Python is a high-level interpreted programming language known for its simplicity and readability. It was created by Guido van Rossum and released in 1991 \cite{martin2023stam,wijanarko2020prediksi} . It has become one of the most popular programming languages worldwide. Its object-oriented approach helps programmers to write logical and clear code for small and large projects. Python libraries (packages) effectively simplify many important processes such as analyzing and visualizing data, retrieving unstructured data from the web, image processing, building machine learning models, and textual information \cite{Samira_Gholizadeh2022}. Here, it has been used to implement the PSO algorithm to optimize query performance, connect to the database using libraries like pyodbc, measures, executes, and analyzes the database programmatically. Due to its simplicity, readability, extensive libraries, ease of use, and efficiency, it is one of the best choices for implementing PSO.

\item\textbf{pyodbc:} Pyodbc is an open-source Python module that makes accessing ODBC\footnote{Open Database Connectivity (ODBC) is a standardized application programming interface (API) for accessing databases.} databases simple. It implements the DB API\footnote{An Application Programming Interface (API) is a set of protocols, tools, and definitions that allow different software applications to communicate with each other.} 2.0 specification but is packed with even more Pythonic convenience. The pyodbc library connects Python to MSSQL, allowing programmatic execution of SQL queries and retrieval of results.

\item\textbf{pandas:} The pandas constitute an open-source data manipulation and analysis tool that is fast, powerful, flexible, and easy to use. It provides data structures like DataFrames and Series, which are particularly useful for working with structured data. It is built entirely on the Python programming language. Pandas is used to analyze query performance data, such as execution times and storage costs, for better insights.

\item\textbf{matplotlib:} Matplotlib is a comprehensive library for creating static, animated, and interactive visualizations in Python. \enquote{Matplotlib makes easy things easy and hard things possible} \cite{matplotlib}. It has been used to visualize query performance metrics and PSO convergence, making results easier to interpret.

\item\textbf{GitHub:} GitHub is an open-source version control system that is used for tracking the changes made in the files and for enabling a collaborative software world. GitHub is incredibly popular due to its flexibility, speed, and ability to support almost any workflow. It has become the world's largest source code host, with around 90 per cent of developers worldwide using it to create, store, manage, and share their code. GitHub is used to track changes in Python scripts, LaTeX documents, and SQL files, making it an essential tool for collaboration and version control during thesis development. The platform’s features, such as branching, pull requests, and issue tracking, enable efficient project management and seamless integration of feedback from advisors or peers. Additionally, GitHub’s public or private repository options allow me to share my work with the academic community while retaining control over access.

\end{enumerate}

\clearpage

\subsection{Practical Implementation }

\subsubsection{Database Creation}
Choosing a database or creating one if not available is the initial step of the implementation. For the demonstration purpose of this thesis only, one database called \texttt{HealthInsuranceDB} is created here. The database was populated with tables and sample data to simulate a real-world environment. Relationships among tables are created through primary keys and foreign keys to ensure data integrity and efficient querying, enabling the testing and optimization of queries using materialized views and the PSO algorithm.

The following SQL query checks if the database \texttt{HealthInsuranceDB} exists and creates it if it does not: \vspace{.4cm}


% Define SQL style
\lstdefinestyle{sql}{
    language=SQL,
    backgroundcolor=\color{white},
    basicstyle=\ttfamily\footnotesize,
    keywordstyle=\color{blue},
    commentstyle=\color{green!40!black},
    stringstyle=\color{red},
    showstringspaces=false,
    breaklines=true,
    frame=single,
    rulecolor=\color{black},
    tabsize=2,
    captionpos=b,
    aboveskip=10pt,
    belowskip=10pt
}


The following SQL query checks if the database \texttt{AccessAuditDB} exists and creates it if it does not:

\begin{lstlisting}[style=sql, caption={SQL Query to Create Database}, label={lst:sql-create-db}]
IF NOT EXISTS (SELECT * FROM sys.databases WHERE name = 'AccessAuditDB')
BEGIN
    CREATE DATABASE AccessAuditDB;
END
GO
USE AccessAuditDB;
\end{lstlisting}




\subsubsection{Table Creation} A database schema with four tables, namely \texttt{InsuranceProviders}, \texttt{Patients}, \texttt{Claims}, and \texttt{Treatments}, are created with the help of the following SQL query. These tables are intended to store information on insurance providers, patient details, insurance claims, and associated treatments respectively. Foreign key constraints ensure data integrity as they set up relationships between the tables, making this schema suitable for demonstrating materialized views and query optimization in a health insurance context. \vspace{.4cm}

% Define colors
\definecolor{codegreen}{rgb}{0,0.6,0}  % ✅ Green for comments
\definecolor{codegray}{rgb}{0.5,0.5,0.5}  % ✅ Gray for numbers
\definecolor{codepurple}{rgb}{0.58,0,0.82}  % ✅ Purple for strings
\definecolor{backcolour}{rgb}{0.95,0.95,0.92}  % ✅ Light gray background
\definecolor{bordercolor}{rgb}{0.7,0.7,0.7}  % ✅ Left border color (gray)
\definecolor{codeblue}{rgb}{0,0,0.8}  % ✅ Blue for SQL keywords

% ✅ Define SQL language with correct comment handling
\lstdefinelanguage{MySQL}{
    keywords={SELECT, FROM, WHERE, JOIN, ON, INNER, OUTER, LEFT, RIGHT, FULL, GROUP, BY, ORDER, ASC, DESC, AS, COUNT, SUM, AVG, MAX, MIN, DISTINCT, INSERT, INTO, VALUES, UPDATE, SET, DELETE, CREATE, TABLE, PRIMARY, FOREIGN, KEY, DEFAULT, NULL, NOT, CHECK, CONSTRAINT, INDEX, VIEW, MATERIALIZED, PROCEDURE, FUNCTION, TRIGGER, DATABASE, ALTER, DROP, EXEC, IF, EXISTS, UNION, ALL, CASE, WHEN, THEN, ELSE, END, CAST, CONVERT, LIKE, IN, BETWEEN, AND, OR, HAVING, LIMIT, OFFSET},
    sensitive=false,
    morestring=[b]',  % ✅ Strings in single quotes
    morestring=[b]",  % ✅ Strings in double quotes
    morecomment=[l][\color{codegreen}]{--}  % ✅ Ensures full line comment in green
}

\lstdefinestyle{sqlstyle}{
    backgroundcolor=\color{backcolour},   
    commentstyle=\color{codegreen},  % ✅ Comments in green
    keywordstyle=\bfseries\color{codeblue},  % ✅ SQL Keywords in Blue & Bold
    numberstyle=\scriptsize\color{codegray},  % ✅ Row numbers in gray
    stringstyle=\color{codepurple},  % ✅ Strings in purple
    basicstyle=\ttfamily\footnotesize,
    breaklines=true,
    captionpos=b,
    numbers=left,      % ✅ Enables row numbers on the left
    stepnumber=1,      % ✅ Row numbers increment by 1
    firstnumber=1,     % ✅ Starts numbering at 1
    numbersep=8pt,     % ✅ Increases space between numbers and SQL code
    xleftmargin=3em,   % ✅ Ensures space inside the left border
    frame=single,      % ✅ Keeps a single border (left-aligned)
    framesep=5pt,      % ✅ Ensures space inside the frame
    rulesepcolor=\color{bordercolor},  % ✅ Matches row numbers with left border
    rulecolor=\color{bordercolor},  % ✅ Sets left border color
    language=MySQL  % ✅ Uses SQL keyword highlighting
}

\begin{lstlisting}[style=sqlstyle, caption={SQL query to Create Database}]
-- Create InsuranceProviders Table
CREATE TABLE InsuranceProviders (
    InsuranceProviderID INT PRIMARY KEY IDENTITY(1,1),
    ProviderName NVARCHAR(100) NOT NULL,
    Address NVARCHAR(255),
    City NVARCHAR(100),
    State NVARCHAR(50),
    ZipCode NVARCHAR(20)
);

-- Create Patients Table
CREATE TABLE Patients (
    PatientID INT PRIMARY KEY IDENTITY(1,1),
    FirstName NVARCHAR(50) NOT NULL,
    LastName NVARCHAR(50) NOT NULL,
    DateOfBirth DATE NOT NULL,
    Gender CHAR(1) CHECK (Gender IN ('M', 'F', 'O')), -- M: Male, F: Female, O: Other
    Address NVARCHAR(255),
    City NVARCHAR(100),
    State NVARCHAR(50),
    ZipCode NVARCHAR(20),
    InsuranceProviderID INT FOREIGN KEY REFERENCES InsuranceProviders(InsuranceProviderID)
);

-- Create Claims Table
CREATE TABLE Claims (
    ClaimID INT PRIMARY KEY IDENTITY(1,1),
    PatientID INT FOREIGN KEY REFERENCES Patients(PatientID),
    InsuranceProviderID INT FOREIGN KEY REFERENCES InsuranceProviders(InsuranceProviderID),
    ClaimDate DATE NOT NULL,
    ClaimAmount DECIMAL(18, 2) NOT NULL,
    Status NVARCHAR(50) CHECK (Status IN ('Pending', 'Approved', 'Rejected'))
);

-- Create Treatments Table
CREATE TABLE Treatments (
    TreatmentID INT PRIMARY KEY IDENTITY(1,1),
    ClaimID INT FOREIGN KEY REFERENCES Claims(ClaimID),
    TreatmentDate DATE NOT NULL,
    TreatmentType NVARCHAR(100) NOT NULL,
    Cost DECIMAL(18, 2) NOT NULL
);
GO
\end{lstlisting}

\subsubsection{Inserting Random Data in Database}
One million patient records are automatically generated through a looped SQL script~\ref{lst:inserting_data}. Random values are assigned for date of birth (ranging from 0-100 years), gender (evenly distributed among M/F/O), and location data (address, city, state, and zip code).\vspace{.4cm}

 \input{SQL/inserting_data}

\subsubsection{Identify Complex Queries} This is a very important part of identifying the frequent queries that need to be optimized. It refers to the process of analyzing a database workload to pinpoint queries that are resource-intensive, frequently executed, or critical to performance. SQL Server Management Studio or direct SQL query can be used to analyze execution logs and identify the frequently executed and resource-intensive queries. Also, identify frequently used sub-queries, aggregation (Queries with GROUP BY, SUM, AVG)s, or frequent joins between tables. For example, the following query to identify long-running queries: \vspace{.4cm}

\input{SQL/Query_identify}

As shown in Listing~\ref{lst:IdentifyComplexQueries}, query retrieves the top 10 most time-consuming queries from SQL Server by analyzing the \(\texttt{sys.dm\_exec\_query\_stats}\) Dynamic Management Views(DMV)\footnote{DMVs are system defined views for database administrators and developers to troubleshoot performance issues, identify missing indexes, analyze query performance, and monitor resource usage efficiently.}. It calculates the total elapsed time in milliseconds, counts the number of executions, and fetches the query text using \(\texttt{sys.dm\_exec\_sql\_text}\). The results are sorted by \(\texttt{total\_elapsed\_time}\) in descending order to highlight the slowest queries for performance tuning and optimization. These queries were then used to create materialized views, which were optimized using the PSO algorithm to improve overall query performance.

\subsubsection{ Materialized views Creation (Indexed Views)}\label{Query_decomposition} Once the top slow queries are sorted out, MVs(In MSSQL Server Mvs are implemented as Indexed views ) are created for each of these queries to store their precomputed results.\vspace{.4cm}

  %\input{SQL/Create_MSSQL}
  % Define colors
\definecolor{codegreen}{rgb}{0,0.6,0}  % ✅ Green for comments
\definecolor{codegray}{rgb}{0.5,0.5,0.5}  % ✅ Gray for numbers
\definecolor{codepurple}{rgb}{0.58,0,0.82}  % ✅ Purple for strings
\definecolor{backcolour}{rgb}{0.95,0.95,0.92}  % ✅ Light gray background
\definecolor{bordercolor}{rgb}{0.7,0.7,0.7}  % ✅ Left border color (gray)
\definecolor{codeblue}{rgb}{0,0,0.8}  % ✅ Blue for SQL keywords

\definecolor{commentcolor}{RGB}{0, 128, 0}  % Green for comments
\definecolor{titlecolor}{RGB}{0, 0, 255}    % Blue for titles
% ✅ Define SQL language with correct comment handling
\lstdefinelanguage{MySQL}{
    keywords={SELECT, FROM, WHERE, JOIN, ON, INNER, OUTER, LEFT, RIGHT, FULL, GROUP, BY, ORDER, ASC, DESC, AS, COUNT, SUM, AVG, MAX, MIN, DISTINCT, INSERT, INTO, VALUES, UPDATE, SET, DELETE, CREATE, TABLE, PRIMARY, FOREIGN, KEY, DEFAULT, NULL, NOT, CHECK, CONSTRAINT, INDEX, VIEW, MATERIALIZED, PROCEDURE, FUNCTION, TRIGGER, DATABASE, ALTER, DROP, EXEC, IF, EXISTS, UNION, ALL, CASE, WHEN, THEN, ELSE, END, CAST, CONVERT, LIKE, IN, BETWEEN, AND, OR, HAVING, LIMIT, OFFSET},
    sensitive=false,
    morestring=[b]',  % ✅ Strings in single quotes
    morestring=[b]",  % ✅ Strings in double quotes
    morecomment=[l][\color{codegreen}]{--}  % ✅ Ensures full line comment in green
}

\lstdefinestyle{sqlstyle}{
    backgroundcolor=\color{backcolour},   
    commentstyle=\color{codegreen},  % ✅ Comments in green
    keywordstyle=\bfseries\color{codeblue},  % ✅ SQL Keywords in Blue & Bold
    numberstyle=\scriptsize\color{codegray},  % ✅ Row numbers in gray
    stringstyle=\color{codepurple},  % ✅ Strings in purple
    basicstyle=\ttfamily\footnotesize,
    breaklines=true,
    captionpos=b,
    title=\color{titlecolor}\textbf{Table Creation Script}, % Title in blue
    numbers=left,      % ✅ Enables row numbers on the left
    stepnumber=1,      % ✅ Row numbers increment by 1
    firstnumber=1,     % ✅ Starts numbering at 1
    numbersep=8pt,     % ✅ Increases space between numbers and SQL code
    xleftmargin=3em,   % ✅ Ensures space inside the left border
    frame=single,      % ✅ Keeps a single border (left-aligned)
    framesep=5pt,      % ✅ Ensures space inside the frame
    rulesepcolor=\color{bordercolor},  % ✅ Matches row numbers with left border
    rulecolor=\color{bordercolor},  % ✅ Sets left border color
    language=MySQL  % ✅ Uses SQL keyword highlighting
}
  
\begin{lstlisting}[style=sqlstyle, caption={Materialized view creation}, label=lst:MV_creation]
--Materialized View for Query 1: Total Claims By Patient
CREATE VIEW TotalClaimsByPatient
WITH SCHEMABINDING
AS
SELECT PatientID, COUNT_BIG(*) AS TotalClaims
FROM dbo.Claims
GROUP BY PatientID;
GO

CREATE UNIQUE CLUSTERED INDEX IX_TotalClaimsByPatient
ON TotalClaimsByPatient (PatientID);
GO

-- This view calculates the total number of claims submitted by each patient. It groups the data by PatientID and uses COUNT_BIG(*) to count all related claims.



--Materialized View for Query 2: TotalTreatmentCostByProvider


CREATE VIEW TotalTreatmentCostByProvider
WITH SCHEMABINDING
AS
SELECT c.InsuranceProviderID, SUM(t.Cost) AS TotalCost
FROM dbo.Treatments t
JOIN dbo.Claims c ON t.ClaimID = c.ClaimID
GROUP BY c.InsuranceProviderID;
GO

CREATE UNIQUE CLUSTERED INDEX IX_TotalTreatmentCostByProvider
ON TotalTreatmentCostByProvider (InsuranceProviderID);
GO

-- This view computes the total treatment cost for each insurance provider. It joins the Treatments and Claims tables and sums up the Cost field, grouped by InsuranceProviderID.

--Materialized View for Query 3: MonthlyClaimsByProvider

CREATE VIEW MonthlyClaimsByProvider
WITH SCHEMABINDING
AS
SELECT 
    c.InsuranceProviderID, 
    YEAR(c.ClaimDate) AS ClaimYear, 
    MONTH(c.ClaimDate) AS ClaimMonth, 
    COUNT_BIG(*) AS TotalClaims
FROM dbo.Claims c
GROUP BY c.InsuranceProviderID, YEAR(c.ClaimDate), MONTH(c.ClaimDate);
GO

CREATE UNIQUE CLUSTERED INDEX IX_MonthlyClaimsByProvider
ON MonthlyClaimsByProvider (InsuranceProviderID, ClaimYear, ClaimMonth);
GO

-- This view summarizes the number of claims submitted per insurance provider for each month. It extracts the year and month from ClaimDate and counts claims grouped by provider, year, and month.
\end{lstlisting}\vspace{.4cm} 



These materialized/indexed views in listing ~\ref{lst:MV_creation} are designed to optimize query performance in a health insurance database by precomputing and storing aggregated results. The \texttt{TotalClaimsByPatient} view demonstrates the retrieval of claim counts per patient, while the \texttt{TotalTreatmentCostByProvider} view offers a rapid summary of treatment costs by the insurance provider. The \texttt{MonthlyClaimsByProvider} view breaks down monthly claims in detail, enabling efficient analysis of claim trends over time. By materializing these views with unique clustered indexes, the database ensures faster query execution and reduced computational costs on data with frequent access.



  \subsubsection{View Maintenance and Refresh Strategies}\label{View_maintainance} Incremental, Manual or automatic refresh strategies can be set up according to the requirements query to create a scheduled job with the help of script as in the listing ~\ref{lst:Maintenance_and_Refresh_Strategies}.\vspace{.4cm}

% Define colors
\definecolor{codegreen}{rgb}{0,0.6,0}  % ✅ Green for comments
\definecolor{codegray}{rgb}{0.5,0.5,0.5}  % ✅ Gray for numbers
\definecolor{codepurple}{rgb}{0.58,0,0.82}  % ✅ Purple for strings
\definecolor{backcolour}{rgb}{0.95,0.95,0.92}  % ✅ Light gray background
\definecolor{bordercolor}{rgb}{0.7,0.7,0.7}  % ✅ Left border color (gray)
\definecolor{codeblue}{rgb}{0,0,0.8}  % ✅ Blue for SQL keywords

% ✅ Define SQL language with correct comment handling
\lstdefinelanguage{MySQL}{
    keywords={SELECT, FROM, WHERE, JOIN, ON, INNER, OUTER, LEFT, RIGHT, FULL, GROUP, BY, ORDER, ASC, DESC, AS, COUNT, SUM, AVG, MAX, MIN, DISTINCT, INSERT, INTO, VALUES, UPDATE, SET, DELETE, CREATE, TABLE, PRIMARY, FOREIGN, KEY, DEFAULT, NULL, NOT, CHECK, CONSTRAINT, INDEX, VIEW, MATERIALIZED, PROCEDURE, FUNCTION, TRIGGER, DATABASE, ALTER, DROP, EXEC, IF, EXISTS, UNION, ALL, CASE, WHEN, THEN, ELSE, END, CAST, CONVERT, LIKE, IN, BETWEEN, AND, OR, HAVING, LIMIT, OFFSET},
    sensitive=false,
    morestring=[b]',  % ✅ Strings in single quotes
    morestring=[b]",  % ✅ Strings in double quotes
    morecomment=[l][\color{codegreen}]{--}  % ✅ Ensures full line comment in green
}

\lstdefinestyle{sqlstyle}{
    backgroundcolor=\color{backcolour},   
    commentstyle=\color{codegreen},  % ✅ Comments in green
    keywordstyle=\bfseries\color{codeblue},  % ✅ SQL Keywords in Blue & Bold
    numberstyle=\scriptsize\color{codegray},  % ✅ Row numbers in gray
    stringstyle=\color{codepurple},  % ✅ Strings in purple
    basicstyle=\ttfamily\footnotesize,
    breaklines=true,
    captionpos=b,
    numbers=left,      % ✅ Enables row numbers on the left
    stepnumber=1,      % ✅ Row numbers increment by 1
    firstnumber=1,     % ✅ Starts numbering at 1
    numbersep=8pt,     % ✅ Increases space between numbers and SQL code
    xleftmargin=3em,   % ✅ Ensures space inside the left border
    frame=single,      % ✅ Keeps a single border (left-aligned)
    framesep=5pt,      % ✅ Ensures space inside the frame
    rulesepcolor=\color{bordercolor},  % ✅ Matches row numbers with left border
    rulecolor=\color{bordercolor},  % ✅ Sets left border color
    language=MySQL  % ✅ Uses SQL keyword highlighting
}
         \begin{lstlisting}[style=sqlstyle]
-- Example of creating a job to refresh a materialized view automatically every hour

USE msdb;
GO

-- Create the job
EXEC dbo.sp_add_job
    @job_name = N'RefreshMaterializedViewJob',
    @enabled = 1,
    @description = N'Job to refresh the materialized view every hour.';

-- Add a job step
EXEC sp_add_jobstep
    @job_name = N'RefreshMaterializedViewJob',
    @step_name = N'RefreshViewStep',
    @subsystem = N'TSQL',
    @command = N'EXEC RefreshMaterializedView;',
    @retry_attempts = 3,
    @retry_interval = 5;

-- Create a schedule for the job
EXEC sp_add_schedule
    @schedule_name = N'HourlySchedule',
    @freq_type = 4, -- Daily
    @freq_interval = 1, -- Every day
    @freq_subday_type = 8, -- Hourly
    @freq_subday_interval = 1, -- Every 1 hour
    @active_start_time = 000000; -- Start time (midnight)

-- Attach the schedule to the job
EXEC sp_attach_schedule
    @job_name = N'RefreshMaterializedViewJob',
    @schedule_name = N'HourlySchedule';

-- Assign the job to the SQL Server Agent service
EXEC dbo.Sp_add_jobserver
     @job_name = N'RefreshMaterializedViewJob';

go 

        \end{lstlisting}\vspace{.4cm}

As the database is created here only for demonstration purposes and all data are inserted once manually, therefore materialized views are refreshed manually. This approach is suitable for scenarios where immediate updates are required or when automated refresh mechanisms (like triggers or scheduled jobs) are not feasible. Manual refresh ensures that the views reflect the latest changes in the underlying data, providing accurate results for query processing.\vspace{.4cm}

\subsection{Automation with the PSO algorithm:} \label{Cost_evaluation}
 The following Python code illustrates the implementation of the PSO algorithm for multiple values in the database system. The goal is to minimize query execution time and CPU cost by selecting the most beneficial views to materialize. This section provides a detailed explanation of each function in the implementation.



%\section*{\textbf{Code example to selecting the optimal view using PSO algorithm.} \vspace{.4cm}}


\definecolor{codegreen}{rgb}{0,0.6,0}
\definecolor{codegray}{rgb}{0.5,0.5,0.5}
\definecolor{codepurple}{rgb}{0.58,0,0.82}
\definecolor{backcolour}{rgb}{0.95,0.95,0.92}

\lstdefinestyle{mystyle}{
    backgroundcolor=\color{backcolour},   
    commentstyle=\color{codegreen},
    keywordstyle=\color{magenta},
    numberstyle=\tiny\color{codegray},
    stringstyle=\color{codepurple},
    basicstyle=\ttfamily\footnotesize,
    breakatwhitespace=false,         
    breaklines=true,                 
    captionpos=b,                    
    keepspaces=true,                 
    numbers=left,                    
    numbersep=5pt,                  
    showspaces=false,                
    showstringspaces=false,
    showtabs=false,                  
    tabsize=2
}

\lstset{style=mystyle}



\subsection*{Python Code Explanation}

\subsubsection*{1. Importing Libraries}
The following Python libraries are imported for the script:
\begin{lstlisting}[language=Python]
import pyodbc  # A Python library for interacting with ODBC databases like MSSQL.
import random  # Used to generate random numbers for particle initialization.
import time    # Used to measure execution time.
import numpy as np  # Used for numerical computations.
import logging  # Used for logging information, warnings, and errors.
\end{lstlisting}\vspace{.4cm}
The \texttt{pyodbc} library is used to connect to and interact with the SQL Server database. The \texttt{random} library is used to generate random numbers for initializing particle positions and velocities in the Particle Swarm Optimization (PSO) algorithm. \texttt{Time} library is used to determine the execution time of queries. The \texttt{numpy} library is used for numerical calculations, such as computing the sigmoid function. The \texttt{logging} library is used to log messages, warnings, and errors during the execution of the script.

\subsubsection*{2. Configuring Logging}
The logging system is configured to display messages with a timestamp, log level, and message:
\begin{lstlisting}[language=Python]
logging.basicConfig(level=logging.INFO, format='%(asctime)s - %(levelname)s - %(message)s')
\end{lstlisting}\vspace{.4cm}

This configuration ensures that all log messages are displayed with a timestamp, log level (e.g., INFO, ERROR), and the message itself. This helps in tracking the execution flow and debugging issues.

\subsubsection*{3. Connection Parameters}
The connection parameters for the SQL Server database are defined:
\begin{lstlisting}[language=Python]
server = 'T915-TEST-DB'  # Server name
database = 'AccessAuditDB'  # Database name
driver_name = 'ODBC Driver 17 for SQL Server'  # ODBC driver
\end{lstlisting}\vspace{.4cm}

The \texttt{server} variable holds the name of the server, \texttt{database} specifies the database to connect to, and \texttt{driver\_name} specifies the ODBC driver to use for the connection. These parameters are used to establish a connection to the database.

\subsubsection*{4. Establishing a Database Connection}
The \texttt{create\_connection} function establishes a connection to the database:
\begin{lstlisting}[language=Python]
def create_connection():
    try:
        conn = pyodbc.connect(
            f'DRIVER={{{driver_name}}};'
            f'SERVER={server};'
            f'DATABASE={database};'
            'Trusted_Connection=yes;'  # Windows Authentication
        )
        logging.info("Connection established!")
        logging.info(f"Connected to database: {database} on server: {server}")
        return conn
    except pyodbc.Error as e:
        logging.error("Error connecting to SQL Server:", exc_info=True)
        return None
\end{lstlisting}\vspace{.4cm}

This function attempts to establish a connection to the SQL Server database using the \texttt{pyodbc.connect} method. If the connection is successful, it logs a success message and returns the connection object. If an error occurs, it logs the error and returns \texttt{None}.

\subsubsection*{5. Cost Function}
The \texttt{cost\_function} calculates the total cost of executing queries based on selected materialized views:
\begin{lstlisting}[language=Python]
def cost_function(selected_views, queries, conn):
    total_time = 0
    cpu_cost = 0

    if not any(selected_views):  # No views selected
        return float('inf'), 0, 0  # High cost for no selection

    cursor = conn.cursor()  # Create a new cursor
    for i, view in enumerate(selected_views):
        if view == 1:  # If view is selected
            start_time = time.time()
            try:
                logging.info(f"Executing query: {queries[i]}")
                cursor.execute(queries[i])
                cursor.fetchall()
                execution_time = time.time() - start_time
                total_time += execution_time

                # Estimate CPU cost
                query_complexity = queries[i].upper().count('JOIN') + 1
                cpu_cost += execution_time * query_complexity

            except pyodbc.Error as e:
                logging.error(f"Error executing query {i}: {queries[i]}", exc_info=True)
                return float('inf'), 0, 0  # High cost for query errors

    # Weighted cost function
    alpha, beta = 0.7, 0.3  # Weights for execution time and CPU cost
    total_cost = alpha * total_time + beta * cpu_cost
    return total_cost, total_time, cpu_cost
\end{lstlisting}\vspace{.4cm}

The \texttt{cost\_function} calculates the total cost of executing a set of queries based on the selected materialized views. It measures the execution time and estimates the CPU cost for each query. If no views are selected, it returns a high cost (\texttt{float('inf')}). The total cost is a weighted sum of execution time and CPU cost, where execution time contributes 70\% and CPU cost contributes 30\%.

\subsubsection*{6. Advanced PSO Algorithm}
The \texttt{pso} function implements the Particle Swarm Optimization (PSO) algorithm:
\begin{lstlisting}[language=Python]
def pso(num_particles, num_iterations, num_queries, queries, conn):
    # PSO parameters
    W_max = 0.9  # Maximum inertia weight
    W_min = 0.4  # Minimum inertia weight
    c1, c2 = 1.5, 1.5  # Cognitive and social factors
    v_max = 6.0  # Maximum velocity for clamping

    # Initialize particles
    particles = [{'position': [random.choice([0, 1]) for _ in range(num_queries)],
                 'velocity': [random.uniform(-1, 1) for _ in range(num_queries)],
                 'best_position': None,
                 'best_cost': float('inf')} for _ in range(num_particles)]

    global_best_position = None
    global_best_cost = float('inf')
    global_best_time = 0
    global_best_cpu_cost = 0

    # PSO main loop
    for iteration in range(num_iterations):
        logging.info(f"Iteration {iteration + 1} started.")
        
        # Dynamic inertia weight
        W = W_max - (W_max - W_min) * (iteration / num_iterations)

        for particle in particles:
            cost, execution_time, cpu_cost = cost_function(particle['position'], queries, conn)
            particle['cost'] = cost

            # Update personal best
            if cost < particle['best_cost']:
                particle['best_position'] = particle['position'][:]
                particle['best_cost'] = cost

            # Update global best
            if cost < global_best_cost:
                global_best_position = particle['position'][:]
                global_best_cost = cost
                global_best_time = execution_time
                global_best_cpu_cost = cpu_cost

        # Update velocity and position
        for particle in particles:
            for i in range(num_queries):
                r1, r2 = random.random(), random.random()
                # Update velocity
                particle['velocity'][i] = (W * particle['velocity'][i] +
                                          c1 * r1 * (particle['best_position'][i] - particle['position'][i]) +
                                          c2 * r2 * (global_best_position[i] - particle['position'][i]))
                # Clamp velocity
                particle['velocity'][i] = max(min(particle['velocity'][i], v_max), -v_max)
                # Update position using sigmoid function
                sigmoid = 1 / (1 + np.exp(-particle['velocity'][i]))
                particle['position'][i] = 1 if random.random() < sigmoid else 0

        # Log iteration results
        logging.info(f"Iteration {iteration + 1}: Best Cost = {global_best_cost:.4f}, Execution Time = {global_best_time:.4f}, CPU Cost = {global_best_cpu_cost:.4f}")

    return global_best_position, global_best_cost, global_best_time, global_best_cpu_cost
\end{lstlisting}\vspace{.4cm}

The \texttt{pso} function implements the Particle Swarm Optimization (PSO) algorithm. It initializes particles with random positions and velocities. The algorithm iteratively updates the particles' positions and velocities based on their personal best and the global best. The inertia weight (\texttt{W}) decreases over time to balance exploration and exploitation. The algorithm logs the best cost, execution time, and CPU cost for each iteration.

\subsubsection*{7. Main Function}
The \texttt{main} function orchestrates the execution of the script:
\begin{lstlisting}[language=Python]
def main():
    conn = create_connection()
    if not conn:
        return

    # List of queries
    queries = [
        "SELECT * FROM TotalSalesByCustomer",  
        "SELECT * FROM TotalQuantityByProduct",  
        "SELECT * FROM MonthlySales"  
    ]

    # PSO parameters
    num_particles = 5  # Number of particles
    num_iterations = 5  # Number of iterations
    num_queries = len(queries)  # Number of queries

    # Run PSO
    logging.info("Starting PSO algorithm...")
    logging.info(f"Number of particles: {num_particles}")
    logging.info(f"Number of iterations: {num_iterations}")
    best_position, best_cost, best_time, best_cpu_cost = pso(num_particles, num_iterations, num_queries, queries, conn)

    # Output optimal materialized views
    optimal_views = [queries[i] for i, view in enumerate(best_position) if view == 1]
    logging.info("Optimal Materialized Views:")
    for view in optimal_views:
        logging.info(f"- {view}")
    logging.info(f"Best Execution Time: {best_time:.4f}")
    logging.info(f"Best CPU Cost: {best_cpu_cost:.4f}")

    # Close connection
    conn.close()
    logging.info("Connection closed.")

if __name__ == "__main__":
    main()
\end{lstlisting}\vspace{.4cm}

The \texttt{main} function orchestrates the execution of the script. It establishes a database connection, defines the queries to optimize, and sets the PSO parameters. It then runs the PSO algorithm and logs the optimal materialized views, execution time, and CPU cost. Finally, it closes the database connection.

\subsubsection*{8. Execution}
The script is executed when run directly:
\begin{lstlisting}[language=Python]
if __name__ == "__main__":
    main()
\end{lstlisting}\vspace{.4cm}

This block ensures that the \texttt{main} function is executed only when the script is run directly, not when it is imported as a module.

\vspace{.4cm}

A sequence diagram in figure ~\ref{fig:Sequence_diagram} is included here to help better understand the workflow and interactions between the system's components. It provides a step-by-step visualisation of how this code operates, making it more understandable.
  
 % 
\begin{lstlisting}[style=pythonstyle, caption={Python script to automate optimal view.}, label={lst:pso_query_optimization}]

# 
import pyodbc # A python  library for interacting with ODBC databases like MSSQL that helps to manage database connection
import random #It helps to generate random numbers and choice used to initialize particle positions and velocities
import time

# Connection parameters
server = 'T915-TEST-DB'  # server name that used to get data 
database = 'AccessAuditDB'  # Database name
driver_name = 'ODBC Driver 17 for SQL Server'  # ODBC  driver from pyodbc.drivers()
# Uncomment and add these if using SQL Server Authentication
# username = 'm.islam'
# password = 'your_password'

try:
    # Establish connection
    conn = pyodbc.connect(
        f'DRIVER={{{driver_name}}};'
        f'SERVER={server};'
        f'DATABASE={database};'
        'Trusted_Connection=yes;'  # Indicates that Windows Authentication used for Authentication
        # Uncomment these lines for SQL Authentication
        # f'UID={username};'
        # f'PWD={password};'
    )
    cursor = conn.cursor()
    print("Connection established!")

    #  A list of Queries corresponding to materialized views
    queries = [
        "SELECT * FROM TotalSalesByCustomer;",
        "SELECT * FROM TotalQuantityByProduct;",
        "SELECT * FROM MonthlySales;"
    ]

    # Cost function: Measure total query execution time for a set of materialized views 
    def cost_function(selected_views):
        total_time = 0
        if not any(selected_views):  # No views selected
            return float('inf')  # High cost for no selection
        for i, view in enumerate(selected_views):
            if view == 1:  # If view is selected
                start_time = time.time()
                cursor.execute(queries[i]) # Executes the SQL query for the selected view
                cursor.fetchall()
                total_time += time.time() - start_time # Total query execution time 
        return total_time

    # PSO parameters
    num_particles = 5 # Number of particles in the swarm 
    num_iterations = 5 #The number of iterations the algorithm will run
    num_queries = len(queries) #The number of queries (materialized views) to optimize.
    W = 0.5  # Inertia weight Controls the impact of the previous velocity on the current velocity.

    c1, c2 = 1.5, 1.5 # Encourages particles to move toward the personal/global best position.

    # Initialize particles
    particles = [{'position': [random.choice([0, 1]) for _ in range(num_queries)], #Represents a particle's selected views (1 = selected, 0 = not selected).
                  'velocity': [random.uniform(-1, 1) for _ in range(num_queries)], #Represents the rate of change for each view selection.
                  'best_position': None,
                  'best_cost': float('inf')} for _ in range(num_particles)] #The lowest cost (execution time) encountered by the particle.

    global_best_position = None  
    global_best_cost = float('inf')

    # PSO algorithm
    for iteration in range(num_iterations):
        for particle in particles:
            # Evaluate cost
            cost = cost_function(particle['position'])
            print(f"Particle position: {particle['position']}, Cost: {cost:.4f}")

            # Update personal best
            if cost < particle['best_cost']:
                particle['best_position'] = particle['position'][:]
                particle['best_cost'] = cost

            # Update global best
            if cost < global_best_cost:
                global_best_position = particle['position'][:]
                global_best_cost = cost

            # Update velocity and position using PSO formula 
            for i in range(num_queries):
                r1, r2 = random.random(), random.random()
                particle['velocity'][i] = (W * particle['velocity'][i] +
                                           c1 * r1 * (particle['best_position'][i] - particle['position'][i]) +
                                           c2 * r2 * (global_best_position[i] - particle['position'][i]))
                particle['position'][i] = 1 if random.random() < abs(particle['velocity'][i]) else 0

        print(f"Iteration {iteration + 1}: Best Cost = {global_best_cost:.4f}")

    print("Optimal Materialized Views:", global_best_position)

except pyodbc.Error as e:
    print("Error connecting to SQL Server:", e) # Catches and displays database connection errors.


finally:
    if 'conn' in locals() and conn:
        conn.close()
        print("Connection closed.")  #Ensures the database connection is closed after the script execution.





\end{lstlisting} \vspace{.4cm}


\clearpage


% Define block styles


% Inserting the image
\begin{figure}[h]
    \centering
    \includegraphics[width=0.5\textwidth]{Figure/seq.diagram .png} % Replace 'example-image' with your image file name
    \caption{Sequence diagram of code}
    \label{fig:Sequence_diagram}
\end{figure}






\subsubsection{Output from PSO automation }  Upon execution for a specified number of iterations, the code records the best combination of materialized views that minimize the total cost. It makes a list of the iteration-wise data, including the best execution time, CPU cost and the corresponding views for each iteration. Finally, it prints the overall best view according to their parameters. Below is the sample of the output from the python code: \vspace{.4cm}



  

\begin{lstlisting}[style=pythonstyle, caption={Output from python code }, label={lst:pso_query_optimization}]]

2025-03-17 16:58:11,620 - INFO - Connection established!
2025-03-17 16:58:11,620 - INFO - Connected to database: AccessAuditDB on server: T915-TEST-DB
2025-03-17 16:58:11,620 - INFO - Starting PSO algorithm...
2025-03-17 16:58:11,620 - INFO - Number of particles: 5
2025-03-17 16:58:11,620 - INFO - Number of iterations: 5
2025-03-17 16:58:11,620 - INFO - Iteration 1 started.
2025-03-17 16:58:11,621 - INFO - Executing query: SELECT * FROM TotalQuantityByProduct
2025-03-17 16:58:11,741 - INFO - Executing query: SELECT * FROM MonthlySales
2025-03-17 16:58:11,887 - INFO - Executing query: SELECT * FROM MonthlySales
2025-03-17 16:58:12,016 - INFO - Executing query: SELECT * FROM TotalSalesByCustomer
2025-03-17 16:58:16,463 - INFO - Executing query: SELECT * FROM TotalQuantityByProduct
2025-03-17 16:58:17,029 - INFO - Executing query: SELECT * FROM MonthlySales
2025-03-17 16:58:17,531 - INFO - Executing query: SELECT * FROM TotalSalesByCustomer
2025-03-17 16:58:22,793 - INFO - Executing query: SELECT * FROM TotalQuantityByProduct
2025-03-17 16:58:22,916 - INFO - Executing query: SELECT * FROM MonthlySales
2025-03-17 16:58:23,083 - INFO - Executing query: SELECT * FROM TotalSalesByCustomer
2025-03-17 16:58:25,112 - INFO - Executing query: SELECT * FROM TotalQuantityByProduct
2025-03-17 16:58:25,197 - INFO - Executing query: SELECT * FROM MonthlySales
2025-03-17 16:58:25,306 - INFO - Iteration 1: Best Cost = 0.1292, Execution Time = 0.1292, CPU Cost = 0.1292
2025-03-17 16:58:25,306 - INFO - Iteration 2 started.
2025-03-17 16:58:25,306 - INFO - Executing query: SELECT * FROM MonthlySales
2025-03-17 16:58:25,406 - INFO - Executing query: SELECT * FROM TotalQuantityByProduct
2025-03-17 16:58:25,487 - INFO - Executing query: SELECT * FROM TotalSalesByCustomer
2025-03-17 16:58:27,227 - INFO - Executing query: SELECT * FROM MonthlySales
2025-03-17 16:58:27,378 - INFO - Iteration 2: Best Cost = 0.0813, Execution Time = 0.0813, CPU Cost = 0.0813
2025-03-17 16:58:27,378 - INFO - Iteration 3 started.
2025-03-17 16:58:27,378 - INFO - Executing query: SELECT * FROM TotalQuantityByProduct
2025-03-17 16:58:27,494 - INFO - Executing query: SELECT * FROM MonthlySales
2025-03-17 16:58:27,618 - INFO - Executing query: SELECT * FROM TotalSalesByCustomer
2025-03-17 16:58:28,910 - INFO - Executing query: SELECT * FROM MonthlySales
2025-03-17 16:58:28,998 - INFO - Executing query: SELECT * FROM TotalSalesByCustomer
2025-03-17 16:58:29,985 - INFO - Executing query: SELECT * FROM MonthlySales
2025-03-17 16:58:30,080 - INFO - Executing query: SELECT * FROM TotalSalesByCustomer
2025-03-17 16:58:32,420 - INFO - Executing query: SELECT * FROM TotalQuantityByProduct
2025-03-17 16:58:32,484 - INFO - Executing query: SELECT * FROM TotalSalesByCustomer
2025-03-17 16:58:35,111 - INFO - Executing query: SELECT * FROM TotalQuantityByProduct
2025-03-17 16:58:35,467 - INFO - Iteration 3: Best Cost = 0.0813, Execution Time = 0.0813, CPU Cost = 0.0813
2025-03-17 16:58:35,467 - INFO - Iteration 4 started.
2025-03-17 16:58:35,467 - INFO - Executing query: SELECT * FROM TotalSalesByCustomer
2025-03-17 16:58:40,846 - INFO - Executing query: SELECT * FROM TotalQuantityByProduct
2025-03-17 16:58:41,379 - INFO - Executing query: SELECT * FROM TotalSalesByCustomer
2025-03-17 16:58:43,526 - INFO - Executing query: SELECT * FROM TotalQuantityByProduct
2025-03-17 16:58:43,596 - INFO - Executing query: SELECT * FROM TotalSalesByCustomer
2025-03-17 16:58:45,083 - INFO - Executing query: SELECT * FROM TotalQuantityByProduct
2025-03-17 16:58:45,205 - INFO - Executing query: SELECT * FROM MonthlySales
2025-03-17 16:58:45,337 - INFO - Iteration 4: Best Cost = 0.0705, Execution Time = 0.0705, CPU Cost = 0.0705
2025-03-17 16:58:45,337 - INFO - Iteration 5 started.
2025-03-17 16:58:45,337 - INFO - Executing query: SELECT * FROM MonthlySales
2025-03-17 16:58:45,439 - INFO - Executing query: SELECT * FROM TotalSalesByCustomer
2025-03-17 16:58:46,835 - INFO - Executing query: SELECT * FROM TotalQuantityByProduct
2025-03-17 16:58:46,899 - INFO - Executing query: SELECT * FROM TotalQuantityByProduct
2025-03-17 16:58:46,958 - INFO - Executing query: SELECT * FROM MonthlySales
2025-03-17 16:58:47,045 - INFO - Executing query: SELECT * FROM TotalSalesByCustomer
2025-03-17 16:58:48,672 - INFO - Executing query: SELECT * FROM TotalQuantityByProduct
2025-03-17 16:58:48,739 - INFO - Executing query: SELECT * FROM TotalQuantityByProduct
2025-03-17 16:58:48,811 - INFO - Iteration 5: Best Cost = 0.0705, Execution Time = 0.0705, CPU Cost = 0.0705
2025-03-17 16:58:48,812 - INFO - Optimal Materialized Views:
2025-03-17 16:58:48,812 - INFO - - SELECT * FROM TotalQuantityByProduct
2025-03-17 16:58:48,812 - INFO - Best Execution Time: 0.0705
2025-03-17 16:58:48,812 - INFO - Best CPU Cost: 0.0705
2025-03-17 16:58:48,916 - INFO - Connection closed.

[Done] exited with code=0 in 38.476 seconds


\end{lstlisting} \vspace{.4cm}
  

  %\begin{lstlisting}[style=pythonstyle, label={lst:example} caption={Python Code Example}]

Connection established!
Particle position: [0, 1, 1], Cost: 0.1426
Particle position: [1, 0, 1], Cost: 1.0547
Particle position: [0, 0, 1], Cost: 0.0894
Particle position: [1, 1, 1], Cost: 1.0979
Particle position: [0, 0, 0], Cost: inf
Iteration 1: Best Cost = 0.0894
Particle position: [1, 0, 0], Cost: 0.9459
Particle position: [0, 1, 0], Cost: 0.0562
Particle position: [0, 0, 0], Cost: inf
Particle position: [1, 0, 0], Cost: 0.9560
Particle position: [0, 0, 1], Cost: 0.1090
Iteration 2: Best Cost = 0.0562
Particle position: [1, 1, 1], Cost: 1.0917
Particle position: [1, 0, 0], Cost: 0.9502
Particle position: [0, 1, 1], Cost: 0.1423
Particle position: [1, 0, 0], Cost: 0.9476
Particle position: [0, 1, 1], Cost: 0.1524
Iteration 3: Best Cost = 0.0562
Particle position: [1, 1, 1], Cost: 1.0997
Particle position: [1, 1, 0], Cost: 1.0002
Particle position: [0, 0, 0], Cost: inf
Particle position: [1, 1, 0], Cost: 1.0056
Particle position: [0, 0, 0], Cost: inf
Iteration 4: Best Cost = 0.0562
Particle position: [1, 1, 1], Cost: 1.0935
Particle position: [1, 0, 0], Cost: 0.9564
Particle position: [0, 1, 0], Cost: 0.0553
Particle position: [1, 0, 0], Cost: 0.9487
Particle position: [0, 1, 1], Cost: 0.1417
Iteration 5: Best Cost = 0.0553
Optimal Materialized Views: [0, 1, 0]

Query Optimization Comparison:
                       Metric  Average Query Time (s)
0  Without Materialized Views                     inf
1     With Materialized Views                0.668428
Connection closed.


\end{lstlisting} \vspace{.4cm}

  

\textbf{For the full code and further test cases, please refer to the Appendix~\ref{fullcode:Fullcode}.}


\subsection{Performance Testing and Data Analysis} After creating the materialized views on the database \texttt{"HealthInsuranceDB"}, performance metrics were collected using SQL query statistics. Query elapsed times and CPU usages are recorded using built-in database monitoring tools, such as Dynamic Management Views (DMVs) or with the help of SQL queries. \vspace{.4cm}

 
\definecolor{dkgreen}{rgb}{0,0.6,0}
\definecolor{gray}{rgb}{0.5,0.5,0.5}
\definecolor{mauve}{rgb}{0.58,0,0.82}
\lstset{language=SQL,
  basicstyle={\small\ttfamily},
  belowskip=3mm,
  breakatwhitespace=true,
  breaklines=true,
  classoffset=0,
  columns=flexible,
  commentstyle=\color{dkgreen},
  framexleftmargin=0.25em,
  frameshape={}{yy}{}{}, %To remove to vertical lines on left, set `frameshape={}{}{}{}`
  keywordstyle=\color{blue},
  numbers=none, %If you want line numbers, set `numbers=left`
  numberstyle=\tiny\color{gray},
  showstringspaces=false,
  stringstyle=\color{mauve},
  tabsize=3,
  xleftmargin =1em
}
         \begin{lstlisting}
SET STATISTICS TIME ON;
GO
-- TEST CASE 1: Total Claims By Patient
-- ===========================================================
PRINT CHAR(10) + '1. TOTAL CLAIMS BY PATIENT' + CHAR(10) + REPLICATE('-', 40);

PRINT 'DIRECT QUERY:';
SELECT 
    p.PatientID,
    p.FirstName,
    p.LastName,
    COUNT(c.ClaimID) AS TotalClaims
FROM Patients p
JOIN Claims c ON p.PatientID = c.PatientID
GROUP BY p.PatientID, p.FirstName, p.LastName;

PRINT CHAR(10) + 'MATERIALIZED VIEW:';
SELECT * FROM TotalClaimsByPatient;
GO
-- Disable statistics
SET STATISTICS TIME OFF;
GO
        \end{lstlisting}

 \textbf{The full script to analyse performance is provided in listing ~\ref{lst:Analysis}.}

\begin{enumerate}


    \item \textbf{ Query response time comparison:}\\
The tests were performed using \textit{MSSQL} on a system running \textit{Windows 10 Pro} with the following specifications:
\begin{itemize}
    \item \textbf{System Type}: 64-bit operating system, x64-based processor.
    \item \textbf{Processor}: 12th Gen Intel(R) Core(TM) i7-1255U, 1.70 GHz.
    \item \textbf{RAM}: 32 GB.
    \item \textbf{Server}: MSSQL.
\end{itemize}\vspace{.4cm}

After executing each query multiple times, both with and without materialized views, the \textit{response time}\footnote{Time taken to execute queries} and \textit{CPU usage}\footnote{Percentage of CPU utilization during query execution and refresh processes.} were noted in from the "Messages" tab in SSMS. The effectiveness of the materialized views was also evaluated using a comparison table and the percentage difference formula.

\begin{itemize}
\item\textbf{Differences between optimization methods:} The following \hyperref[comparison_table]{\textit{Table}~\ref*{comparison_table}} presents a performance comparison of elapsed time using two different approaches: \textit{Direct SELECT Aggregation}, and \textit{Materialized View}. The results show that the MV approach consistently outperforms the other method, with an average execution time of \textbf{69.2 ms}, compared to \textbf{106.6 ms} for direct \textit{SELECT} aggregation. This demonstrates that materialized views significantly reduce query execution time, providing a \textbf{35.08\%} improvement over direct aggregation. The consistent performance across multiple runs highlights the reliability and efficiency of materialized views in optimizing database queries.\vspace{.4cm}
 
 \begin{table}[h!]
\centering
\caption{Performance Comparison}
\renewcommand{\arraystretch}{1.2} % Adjust row height
\setlength{\tabcolsep}{8pt}       % Adjust column width
\resizebox{\textwidth}{!}{ % Resizes the table to fit the page width
\begin{tabular}{|c|>{\columncolor[HTML]{D9EAF1}}c|>{\columncolor[HTML]{D9EAF1}}c|>{\columncolor[HTML]{D9EAF1}}c|}
\hline
\rowcolor[HTML]{276B7A} 
\textbf{Run} & \textbf{\textcolor{white}{Direct SELECT Aggregation (ms)}} & \textbf{\textcolor{white}{Indexed View Query (ms)}} & \textbf{\textcolor{white}{Materialized View (ms)}} \\ \hline
Run 1        & 177                                                        & 15                                                & 5                                                \\ \hline
Run 2        & 40                                                         & 14                                                & 3                                                \\ \hline
Run 3        & 36                                                         & 16                                                & 3                                                \\ \hline
Run 4        & 40                                                         & 14                                                & 6                                                \\ \hline
Run 5        & 40                                                         & 15                                                & 7                                                \\ \hline
\rowcolor[HTML]{BFDDE5} 
\textbf{Average} & \textbf{66.6}                                           & \textbf{14.8}                                     & \textbf{5}                                       \\ \hline
\end{tabular}
} % End of \resizebox
\label{table:performance_comparison}
\end{table}\vspace{.4cm
 }

 \item\textbf{Analysis of percentage differences in query performance:}
 
For example, the percentage difference for Query 1 is calculated using the formula:

\begin{equation}
\text{Difference D in  (\%)} = \frac{W - M}{W} \times 100
\end{equation}

\noindent \textbf{Where:}
\begin{itemize}
    \item \( W \) is the initial value (Execution time without materialized views).
    \item \( M \) is the new value (Execution time with materialized views).
    \item \( D \) is the difference between \( W \) and \( M \) (i.e., \( D = W - M \)).
\end{itemize}

Substituting the values:

\[
\text{ D in (\%)} = \frac{2.35 - 0.45}{2.35} \times 100 \approx 80.85\%
\]

\begin{comment}\[
\text{Difference (\%)} = \frac{\text{Without MV} - \text{With MV}}{\text{Without MV}} \times 100 = \frac{2.35 - 0.45}{2.35} \times 100 \approx 80.85\%
\]
\end{comment}\vspace{.4cm}

The output indicates an \textit{39.47\% }improvement in performance. Here, \( W = 114 \) represents the initial execution time (without optimization), and \( M = 69 \) represents the improved execution time (with optimization). This significant reduction in execution time demonstrates the effectiveness of the optimization technique, as it reduces the query processing time by approximately \textit{39.47\%}, leading to faster and more efficient database operations.

\item \textbf{Impact of MV on CPU and elapsed time:} The \hyperref[tab:performance]{\textit{Table}~\ref*{tab:performance}} shows the execution metrics for three queries, highlighting significant improvements when using MVs. Particularly is the \textit{MonthlyClaimsByProvider} where CPU and elapsed times were reduced by factors 47x and 38x, respectively. The \textit{"TotalTreatmentCostByProvider"} showed modest benefits (1.1× and 1.6×), while the  \textit{"TotalClaimsByPatient"} demonstrated more notable gains, with execution times improved by approximately 6.3×. The results confirm that materialized views can substantially enhance query performance, especially for complex data aggregations and frequent report generation.

% Define custom colors
\definecolor{thesisblue}{RGB}{25,84,166}  % Darker, more professional blue
\definecolor{rowgray}{RGB}{248,248,248}   % Very light gray for subtle contrast
\begin{table}[H]
\centering
\caption{Performance Comparison of Views}
\label{tab:performance}
\renewcommand{\arraystretch}{1.5} % Increased row height
\setlength{\tabcolsep}{10pt} % Optimal column spacing
\begin{tabular}{|>{\RaggedRight}p{3.2cm}|rr|rr|rr|}
\hline
\rowcolor{thesisblue}
\multicolumn{1}{|>{\centering\color{white}}p{3.2cm}|}{\textbf{Test Case}} & 
\multicolumn{2}{>{\centering\color{white}}p{2.4cm}|}{\textbf{Direct (ms)}} & 
\multicolumn{2}{>{\centering\color{white}}p{2.4cm}|}{\textbf{MV (ms)}} & 
\multicolumn{2}{>{\centering\color{white}}p{2.8cm}|}{\textbf{Improvement Factor}} \\
\cline{2-7}
\rowcolor{thesisblue}
& \color{white}\textbf{CPU} & \color{white}\textbf{Elap.} & \color{white}\textbf{CPU} & \color{white}\textbf{Elap.} & \color{white}\textbf{CPU} & \color{white}\textbf{Elap.} \\
\hline
\rowcolor{rowgray}
TotalClaims\\ByPatient & 203 & 4,983 & 32 & 797 & 6.3$\times$ & 6.3$\times$ \\ \hline
TotalTreatment\\CostByProvider & 297 & 110 & 282 & 68 & 1.1$\times$ & 1.6$\times$ \\ \hline
\rowcolor{rowgray}
MonthlyClaims\\ByProvider & 47 & 38 & 1 & 1 & 47$\times$ & 38$\times$ \\ \hline
\end{tabular}

\vspace{6pt}
{\footnotesize 
\textbf{Note:} Elap. = Elapsed Time. Factors are calculated as Direct/MV time ratios. (X for multiplicative improvements). \\
Tests on SQL Server 2019 (SSMS v18.4).}
\end{table}
\clearpage

%
% Optional: Define custom colors
\definecolor{headerblue}{RGB}{200, 220, 255}
\definecolor{rowgray}{gray}{0.95}
\definecolor{lightgreen}{RGB}{220, 255, 220}
\definecolor{lightred}{RGB}{255, 230, 230}

\begin{table}[h!]
    \centering
    \caption{Performance Comparison}
    %\label{tab:performance}
    \rowcolors{2}{gray!10}{white} % Alternate row colors
    \begin{tabular}{lccc}
        \toprule
        \rowcolor{blue!10} % Header row color
        \textbf{Query} & \textbf{Without MV (s)} & \textbf{With MV (s)} & \textbf{Difference (\%)} \\
        \midrule
        Query 1 & 2.35 & 0.45 & \cellcolor{white!20}80.85 \\
        Query 2 & 3.78 & 0.62 & \cellcolor{gray!10}83.61 \\
        Query 3 & 4.21 & 0.87 & \cellcolor{white!20}79.34 \\
        \bottomrule
    \end{tabular}
\end{table}



%\item \textbf{Bar Chart Comparison:} As shown in the \hyperref[fig:execution-plan]{barchart}, the execution time demonstrates the optimization achieved by using materialized views. Direct select aggregation is the least efficient as it involves computing aggregation directly from the raw data during query execution, which is resource-intensive and time-consuming. On the other hand, a materialized view is the fastest one. Although the choice of method depends on the trade-off between query speed, storage requirements, and maintenance.



%\begin{figure}[H]
%\centering
%\includegraphics[width=0.8\textwidth]{Figure/Bar_chart.png} % Replace with your image file name
%\caption{Comparison of query execution time} % Caption for the screenshot
%\label{fig:execution-plan} % Label for referencing
%\end{figure}


\end{itemize} \vspace{.4cm}

\item \textbf{CPU performance analysis:}

The following \hyperref[fig:cpu_usage]{\textit{bar chart}} illustrates the average CPU usage for query execution before and after materialized views (MVs) are implemented. The x-axis is used to represent two conditions: \textit{Before} MVs (without materialized views) and \textit{After} MVs (with materialized views). The y-axis is used to represent the average CPU usage as a percentage. As it is evident from the graph, the CPU utilization average went down significantly from \textbf{85\%} (before MVs) to \textbf{30\%} (after MVs), showcasing the effectiveness of materialized views in reducing CPU burden in query processing. The reduction can be accounted for due to precomputation and caching of query output, which reduces real-time computation and subsequently CPU consumption.


% Optional: Define custom colors
\definecolor{headerblue}{RGB}{200, 220, 255}
\definecolor{rowgray}{gray}{0.95}
\definecolor{lightgreen}{RGB}{220, 255, 220}
\definecolor{lightred}{RGB}{255, 230, 230}

\begin{table}[h!]
    \centering
    \caption{Performance Comparison}
    %\label{tab:performance}
    \rowcolors{2}{gray!10}{white} % Alternate row colors
    \begin{tabular}{lccc}
        \toprule
        \rowcolor{blue!10} % Header row color
        \textbf{Query} & \textbf{Without MV (s)} & \textbf{With MV (s)} & \textbf{Difference (\%)} \\
        \midrule
        Query 1 & 2.35 & 0.45 & \cellcolor{white!20}80.85 \\
        Query 2 & 3.78 & 0.62 & \cellcolor{gray!10}83.61 \\
        Query 3 & 4.21 & 0.87 & \cellcolor{white!20}79.34 \\
        \bottomrule
    \end{tabular}
\end{table}





 \textbf{Data will be updated once the final implementation is done}.


   
\section{Future Work}

\textbf{......\textbf{Work on Progress}..........}       
   %\chapter{Conclution}
\section{Conclusion}

\textbf{......\textbf{Work on Progress}..........}       
   
\section{Appendix} % Use a descriptive title
\label{test:Test2} % Add a label for referencing


\textbf{Test 2}

\definecolor{codegreen}{rgb}{0,0.6,0}
\definecolor{codegray}{rgb}{0.5,0.5,0.5}
\definecolor{codepurple}{rgb}{0.58,0,0.82}
\definecolor{backcolour}{rgb}{0.95,0.95,0.92}

\lstdefinestyle{mystyle}{
    backgroundcolor=\color{backcolour},   
    commentstyle=\color{codegreen},
    keywordstyle=\color{magenta},
    numberstyle=\tiny\color{codegray},
    stringstyle=\color{codepurple},
    basicstyle=\ttfamily\footnotesize,
    breakatwhitespace=false,         
    breaklines=true,                 
    captionpos=b,                    
    keepspaces=true,                 
    numbers=left,                    
    numbersep=5pt,                  
    showspaces=false,                
    showstringspaces=false,
    showtabs=false,                  
    tabsize=2
}

\lstset{style=mystyle}



\subsection*{Python Code Explanation}

\subsubsection*{1. Importing Libraries}
The following Python libraries are imported for the script:
\begin{lstlisting}[language=Python]
import pyodbc  # A Python library for interacting with ODBC databases like MSSQL.
import random  # Used to generate random numbers for particle initialization.
import time    # Used to measure execution time.
import numpy as np  # Used for numerical computations.
import logging  # Used for logging information, warnings, and errors.
\end{lstlisting}\vspace{.4cm}
The \texttt{pyodbc} library is used to connect to and interact with the SQL Server database. The \texttt{random} library is used to generate random numbers for initializing particle positions and velocities in the Particle Swarm Optimization (PSO) algorithm. \texttt{Time} library is used to determine the execution time of queries. The \texttt{numpy} library is used for numerical calculations, such as computing the sigmoid function. The \texttt{logging} library is used to log messages, warnings, and errors during the execution of the script.

\subsubsection*{2. Configuring Logging}
The logging system is configured to display messages with a timestamp, log level, and message:
\begin{lstlisting}[language=Python]
logging.basicConfig(level=logging.INFO, format='%(asctime)s - %(levelname)s - %(message)s')
\end{lstlisting}\vspace{.4cm}

This configuration ensures that all log messages are displayed with a timestamp, log level (e.g., INFO, ERROR), and the message itself. This helps in tracking the execution flow and debugging issues.

\subsubsection*{3. Connection Parameters}
The connection parameters for the SQL Server database are defined:
\begin{lstlisting}[language=Python]
server = 'T915-TEST-DB'  # Server name
database = 'AccessAuditDB'  # Database name
driver_name = 'ODBC Driver 17 for SQL Server'  # ODBC driver
\end{lstlisting}\vspace{.4cm}

The \texttt{server} variable holds the name of the server, \texttt{database} specifies the database to connect to, and \texttt{driver\_name} specifies the ODBC driver to use for the connection. These parameters are used to establish a connection to the database.

\subsubsection*{4. Establishing a Database Connection}
The \texttt{create\_connection} function establishes a connection to the database:
\begin{lstlisting}[language=Python]
def create_connection():
    try:
        conn = pyodbc.connect(
            f'DRIVER={{{driver_name}}};'
            f'SERVER={server};'
            f'DATABASE={database};'
            'Trusted_Connection=yes;'  # Windows Authentication
        )
        logging.info("Connection established!")
        logging.info(f"Connected to database: {database} on server: {server}")
        return conn
    except pyodbc.Error as e:
        logging.error("Error connecting to SQL Server:", exc_info=True)
        return None
\end{lstlisting}\vspace{.4cm}

This function attempts to establish a connection to the SQL Server database using the \texttt{pyodbc.connect} method. If the connection is successful, it logs a success message and returns the connection object. If an error occurs, it logs the error and returns \texttt{None}.

\subsubsection*{5. Cost Function}
The \texttt{cost\_function} calculates the total cost of executing queries based on selected materialized views:
\begin{lstlisting}[language=Python]
def cost_function(selected_views, queries, conn):
    total_time = 0
    cpu_cost = 0

    if not any(selected_views):  # No views selected
        return float('inf'), 0, 0  # High cost for no selection

    cursor = conn.cursor()  # Create a new cursor
    for i, view in enumerate(selected_views):
        if view == 1:  # If view is selected
            start_time = time.time()
            try:
                logging.info(f"Executing query: {queries[i]}")
                cursor.execute(queries[i])
                cursor.fetchall()
                execution_time = time.time() - start_time
                total_time += execution_time

                # Estimate CPU cost
                query_complexity = queries[i].upper().count('JOIN') + 1
                cpu_cost += execution_time * query_complexity

            except pyodbc.Error as e:
                logging.error(f"Error executing query {i}: {queries[i]}", exc_info=True)
                return float('inf'), 0, 0  # High cost for query errors

    # Weighted cost function
    alpha, beta = 0.7, 0.3  # Weights for execution time and CPU cost
    total_cost = alpha * total_time + beta * cpu_cost
    return total_cost, total_time, cpu_cost
\end{lstlisting}\vspace{.4cm}

The \texttt{cost\_function} calculates the total cost of executing a set of queries based on the selected materialized views. It measures the execution time and estimates the CPU cost for each query. If no views are selected, it returns a high cost (\texttt{float('inf')}). The total cost is a weighted sum of execution time and CPU cost, where execution time contributes 70\% and CPU cost contributes 30\%.

\subsubsection*{6. Advanced PSO Algorithm}
The \texttt{pso} function implements the Particle Swarm Optimization (PSO) algorithm:
\begin{lstlisting}[language=Python]
def pso(num_particles, num_iterations, num_queries, queries, conn):
    # PSO parameters
    W_max = 0.9  # Maximum inertia weight
    W_min = 0.4  # Minimum inertia weight
    c1, c2 = 1.5, 1.5  # Cognitive and social factors
    v_max = 6.0  # Maximum velocity for clamping

    # Initialize particles
    particles = [{'position': [random.choice([0, 1]) for _ in range(num_queries)],
                 'velocity': [random.uniform(-1, 1) for _ in range(num_queries)],
                 'best_position': None,
                 'best_cost': float('inf')} for _ in range(num_particles)]

    global_best_position = None
    global_best_cost = float('inf')
    global_best_time = 0
    global_best_cpu_cost = 0

    # PSO main loop
    for iteration in range(num_iterations):
        logging.info(f"Iteration {iteration + 1} started.")
        
        # Dynamic inertia weight
        W = W_max - (W_max - W_min) * (iteration / num_iterations)

        for particle in particles:
            cost, execution_time, cpu_cost = cost_function(particle['position'], queries, conn)
            particle['cost'] = cost

            # Update personal best
            if cost < particle['best_cost']:
                particle['best_position'] = particle['position'][:]
                particle['best_cost'] = cost

            # Update global best
            if cost < global_best_cost:
                global_best_position = particle['position'][:]
                global_best_cost = cost
                global_best_time = execution_time
                global_best_cpu_cost = cpu_cost

        # Update velocity and position
        for particle in particles:
            for i in range(num_queries):
                r1, r2 = random.random(), random.random()
                # Update velocity
                particle['velocity'][i] = (W * particle['velocity'][i] +
                                          c1 * r1 * (particle['best_position'][i] - particle['position'][i]) +
                                          c2 * r2 * (global_best_position[i] - particle['position'][i]))
                # Clamp velocity
                particle['velocity'][i] = max(min(particle['velocity'][i], v_max), -v_max)
                # Update position using sigmoid function
                sigmoid = 1 / (1 + np.exp(-particle['velocity'][i]))
                particle['position'][i] = 1 if random.random() < sigmoid else 0

        # Log iteration results
        logging.info(f"Iteration {iteration + 1}: Best Cost = {global_best_cost:.4f}, Execution Time = {global_best_time:.4f}, CPU Cost = {global_best_cpu_cost:.4f}")

    return global_best_position, global_best_cost, global_best_time, global_best_cpu_cost
\end{lstlisting}\vspace{.4cm}

The \texttt{pso} function implements the Particle Swarm Optimization (PSO) algorithm. It initializes particles with random positions and velocities. The algorithm iteratively updates the particles' positions and velocities based on their personal best and the global best. The inertia weight (\texttt{W}) decreases over time to balance exploration and exploitation. The algorithm logs the best cost, execution time, and CPU cost for each iteration.

\subsubsection*{7. Main Function}
The \texttt{main} function orchestrates the execution of the script:
\begin{lstlisting}[language=Python]
def main():
    conn = create_connection()
    if not conn:
        return

    # List of queries
    queries = [
        "SELECT * FROM TotalSalesByCustomer",  
        "SELECT * FROM TotalQuantityByProduct",  
        "SELECT * FROM MonthlySales"  
    ]

    # PSO parameters
    num_particles = 5  # Number of particles
    num_iterations = 5  # Number of iterations
    num_queries = len(queries)  # Number of queries

    # Run PSO
    logging.info("Starting PSO algorithm...")
    logging.info(f"Number of particles: {num_particles}")
    logging.info(f"Number of iterations: {num_iterations}")
    best_position, best_cost, best_time, best_cpu_cost = pso(num_particles, num_iterations, num_queries, queries, conn)

    # Output optimal materialized views
    optimal_views = [queries[i] for i, view in enumerate(best_position) if view == 1]
    logging.info("Optimal Materialized Views:")
    for view in optimal_views:
        logging.info(f"- {view}")
    logging.info(f"Best Execution Time: {best_time:.4f}")
    logging.info(f"Best CPU Cost: {best_cpu_cost:.4f}")

    # Close connection
    conn.close()
    logging.info("Connection closed.")

if __name__ == "__main__":
    main()
\end{lstlisting}\vspace{.4cm}

The \texttt{main} function orchestrates the execution of the script. It establishes a database connection, defines the queries to optimize, and sets the PSO parameters. It then runs the PSO algorithm and logs the optimal materialized views, execution time, and CPU cost. Finally, it closes the database connection.

\subsubsection*{8. Execution}
The script is executed when run directly:
\begin{lstlisting}[language=Python]
if __name__ == "__main__":
    main()
\end{lstlisting}\vspace{.4cm}

This block ensures that the \texttt{main} function is executed only when the script is run directly, not when it is imported as a module.


\textbf{output}
\begin{lstlisting}[style=pythonstyle, label={lst:example} caption={Python Code Example}]

Connection established!
Particle position: [0, 1, 1], Cost: 0.1426
Particle position: [1, 0, 1], Cost: 1.0547
Particle position: [0, 0, 1], Cost: 0.0894
Particle position: [1, 1, 1], Cost: 1.0979
Particle position: [0, 0, 0], Cost: inf
Iteration 1: Best Cost = 0.0894
Particle position: [1, 0, 0], Cost: 0.9459
Particle position: [0, 1, 0], Cost: 0.0562
Particle position: [0, 0, 0], Cost: inf
Particle position: [1, 0, 0], Cost: 0.9560
Particle position: [0, 0, 1], Cost: 0.1090
Iteration 2: Best Cost = 0.0562
Particle position: [1, 1, 1], Cost: 1.0917
Particle position: [1, 0, 0], Cost: 0.9502
Particle position: [0, 1, 1], Cost: 0.1423
Particle position: [1, 0, 0], Cost: 0.9476
Particle position: [0, 1, 1], Cost: 0.1524
Iteration 3: Best Cost = 0.0562
Particle position: [1, 1, 1], Cost: 1.0997
Particle position: [1, 1, 0], Cost: 1.0002
Particle position: [0, 0, 0], Cost: inf
Particle position: [1, 1, 0], Cost: 1.0056
Particle position: [0, 0, 0], Cost: inf
Iteration 4: Best Cost = 0.0562
Particle position: [1, 1, 1], Cost: 1.0935
Particle position: [1, 0, 0], Cost: 0.9564
Particle position: [0, 1, 0], Cost: 0.0553
Particle position: [1, 0, 0], Cost: 0.9487
Particle position: [0, 1, 1], Cost: 0.1417
Iteration 5: Best Cost = 0.0553
Optimal Materialized Views: [0, 1, 0]

Query Optimization Comparison:
                       Metric  Average Query Time (s)
0  Without Materialized Views                     inf
1     With Materialized Views                0.668428
Connection closed.


\end{lstlisting}% Include the external file

   % Back matter with alphabetic page numbering
    % Start alphabetic page numbering with 'a'

    \clearpage
    
 %%bibliography
 \bibliography{other_pages/Bibliography} % print the bibliography, using the .bib-file "bib_filename.bib"
 \clearpage

%\section{Appendix} % Use a descriptive title
%\label{app:python_output} % Add a label for referencing


%\textbf{Test 2}
%
\definecolor{codegreen}{rgb}{0,0.6,0}
\definecolor{codegray}{rgb}{0.5,0.5,0.5}
\definecolor{codepurple}{rgb}{0.58,0,0.82}
\definecolor{backcolour}{rgb}{0.95,0.95,0.92}

\lstdefinestyle{mystyle}{
    backgroundcolor=\color{backcolour},   
    commentstyle=\color{codegreen},
    keywordstyle=\color{magenta},
    numberstyle=\tiny\color{codegray},
    stringstyle=\color{codepurple},
    basicstyle=\ttfamily\footnotesize,
    breakatwhitespace=false,         
    breaklines=true,                 
    captionpos=b,                    
    keepspaces=true,                 
    numbers=left,                    
    numbersep=5pt,                  
    showspaces=false,                
    showstringspaces=false,
    showtabs=false,                  
    tabsize=2
}

\lstset{style=mystyle}



\subsection*{Python Code Explanation}

\subsubsection*{1. Importing Libraries}
The following Python libraries are imported for the script:
\begin{lstlisting}[language=Python]
import pyodbc  # A Python library for interacting with ODBC databases like MSSQL.
import random  # Used to generate random numbers for particle initialization.
import time    # Used to measure execution time.
import numpy as np  # Used for numerical computations.
import logging  # Used for logging information, warnings, and errors.
\end{lstlisting}\vspace{.4cm}
The \texttt{pyodbc} library is used to connect to and interact with the SQL Server database. The \texttt{random} library is used to generate random numbers for initializing particle positions and velocities in the Particle Swarm Optimization (PSO) algorithm. \texttt{Time} library is used to determine the execution time of queries. The \texttt{numpy} library is used for numerical calculations, such as computing the sigmoid function. The \texttt{logging} library is used to log messages, warnings, and errors during the execution of the script.

\subsubsection*{2. Configuring Logging}
The logging system is configured to display messages with a timestamp, log level, and message:
\begin{lstlisting}[language=Python]
logging.basicConfig(level=logging.INFO, format='%(asctime)s - %(levelname)s - %(message)s')
\end{lstlisting}\vspace{.4cm}

This configuration ensures that all log messages are displayed with a timestamp, log level (e.g., INFO, ERROR), and the message itself. This helps in tracking the execution flow and debugging issues.

\subsubsection*{3. Connection Parameters}
The connection parameters for the SQL Server database are defined:
\begin{lstlisting}[language=Python]
server = 'T915-TEST-DB'  # Server name
database = 'AccessAuditDB'  # Database name
driver_name = 'ODBC Driver 17 for SQL Server'  # ODBC driver
\end{lstlisting}\vspace{.4cm}

The \texttt{server} variable holds the name of the server, \texttt{database} specifies the database to connect to, and \texttt{driver\_name} specifies the ODBC driver to use for the connection. These parameters are used to establish a connection to the database.

\subsubsection*{4. Establishing a Database Connection}
The \texttt{create\_connection} function establishes a connection to the database:
\begin{lstlisting}[language=Python]
def create_connection():
    try:
        conn = pyodbc.connect(
            f'DRIVER={{{driver_name}}};'
            f'SERVER={server};'
            f'DATABASE={database};'
            'Trusted_Connection=yes;'  # Windows Authentication
        )
        logging.info("Connection established!")
        logging.info(f"Connected to database: {database} on server: {server}")
        return conn
    except pyodbc.Error as e:
        logging.error("Error connecting to SQL Server:", exc_info=True)
        return None
\end{lstlisting}\vspace{.4cm}

This function attempts to establish a connection to the SQL Server database using the \texttt{pyodbc.connect} method. If the connection is successful, it logs a success message and returns the connection object. If an error occurs, it logs the error and returns \texttt{None}.

\subsubsection*{5. Cost Function}
The \texttt{cost\_function} calculates the total cost of executing queries based on selected materialized views:
\begin{lstlisting}[language=Python]
def cost_function(selected_views, queries, conn):
    total_time = 0
    cpu_cost = 0

    if not any(selected_views):  # No views selected
        return float('inf'), 0, 0  # High cost for no selection

    cursor = conn.cursor()  # Create a new cursor
    for i, view in enumerate(selected_views):
        if view == 1:  # If view is selected
            start_time = time.time()
            try:
                logging.info(f"Executing query: {queries[i]}")
                cursor.execute(queries[i])
                cursor.fetchall()
                execution_time = time.time() - start_time
                total_time += execution_time

                # Estimate CPU cost
                query_complexity = queries[i].upper().count('JOIN') + 1
                cpu_cost += execution_time * query_complexity

            except pyodbc.Error as e:
                logging.error(f"Error executing query {i}: {queries[i]}", exc_info=True)
                return float('inf'), 0, 0  # High cost for query errors

    # Weighted cost function
    alpha, beta = 0.7, 0.3  # Weights for execution time and CPU cost
    total_cost = alpha * total_time + beta * cpu_cost
    return total_cost, total_time, cpu_cost
\end{lstlisting}\vspace{.4cm}

The \texttt{cost\_function} calculates the total cost of executing a set of queries based on the selected materialized views. It measures the execution time and estimates the CPU cost for each query. If no views are selected, it returns a high cost (\texttt{float('inf')}). The total cost is a weighted sum of execution time and CPU cost, where execution time contributes 70\% and CPU cost contributes 30\%.

\subsubsection*{6. Advanced PSO Algorithm}
The \texttt{pso} function implements the Particle Swarm Optimization (PSO) algorithm:
\begin{lstlisting}[language=Python]
def pso(num_particles, num_iterations, num_queries, queries, conn):
    # PSO parameters
    W_max = 0.9  # Maximum inertia weight
    W_min = 0.4  # Minimum inertia weight
    c1, c2 = 1.5, 1.5  # Cognitive and social factors
    v_max = 6.0  # Maximum velocity for clamping

    # Initialize particles
    particles = [{'position': [random.choice([0, 1]) for _ in range(num_queries)],
                 'velocity': [random.uniform(-1, 1) for _ in range(num_queries)],
                 'best_position': None,
                 'best_cost': float('inf')} for _ in range(num_particles)]

    global_best_position = None
    global_best_cost = float('inf')
    global_best_time = 0
    global_best_cpu_cost = 0

    # PSO main loop
    for iteration in range(num_iterations):
        logging.info(f"Iteration {iteration + 1} started.")
        
        # Dynamic inertia weight
        W = W_max - (W_max - W_min) * (iteration / num_iterations)

        for particle in particles:
            cost, execution_time, cpu_cost = cost_function(particle['position'], queries, conn)
            particle['cost'] = cost

            # Update personal best
            if cost < particle['best_cost']:
                particle['best_position'] = particle['position'][:]
                particle['best_cost'] = cost

            # Update global best
            if cost < global_best_cost:
                global_best_position = particle['position'][:]
                global_best_cost = cost
                global_best_time = execution_time
                global_best_cpu_cost = cpu_cost

        # Update velocity and position
        for particle in particles:
            for i in range(num_queries):
                r1, r2 = random.random(), random.random()
                # Update velocity
                particle['velocity'][i] = (W * particle['velocity'][i] +
                                          c1 * r1 * (particle['best_position'][i] - particle['position'][i]) +
                                          c2 * r2 * (global_best_position[i] - particle['position'][i]))
                # Clamp velocity
                particle['velocity'][i] = max(min(particle['velocity'][i], v_max), -v_max)
                # Update position using sigmoid function
                sigmoid = 1 / (1 + np.exp(-particle['velocity'][i]))
                particle['position'][i] = 1 if random.random() < sigmoid else 0

        # Log iteration results
        logging.info(f"Iteration {iteration + 1}: Best Cost = {global_best_cost:.4f}, Execution Time = {global_best_time:.4f}, CPU Cost = {global_best_cpu_cost:.4f}")

    return global_best_position, global_best_cost, global_best_time, global_best_cpu_cost
\end{lstlisting}\vspace{.4cm}

The \texttt{pso} function implements the Particle Swarm Optimization (PSO) algorithm. It initializes particles with random positions and velocities. The algorithm iteratively updates the particles' positions and velocities based on their personal best and the global best. The inertia weight (\texttt{W}) decreases over time to balance exploration and exploitation. The algorithm logs the best cost, execution time, and CPU cost for each iteration.

\subsubsection*{7. Main Function}
The \texttt{main} function orchestrates the execution of the script:
\begin{lstlisting}[language=Python]
def main():
    conn = create_connection()
    if not conn:
        return

    # List of queries
    queries = [
        "SELECT * FROM TotalSalesByCustomer",  
        "SELECT * FROM TotalQuantityByProduct",  
        "SELECT * FROM MonthlySales"  
    ]

    # PSO parameters
    num_particles = 5  # Number of particles
    num_iterations = 5  # Number of iterations
    num_queries = len(queries)  # Number of queries

    # Run PSO
    logging.info("Starting PSO algorithm...")
    logging.info(f"Number of particles: {num_particles}")
    logging.info(f"Number of iterations: {num_iterations}")
    best_position, best_cost, best_time, best_cpu_cost = pso(num_particles, num_iterations, num_queries, queries, conn)

    # Output optimal materialized views
    optimal_views = [queries[i] for i, view in enumerate(best_position) if view == 1]
    logging.info("Optimal Materialized Views:")
    for view in optimal_views:
        logging.info(f"- {view}")
    logging.info(f"Best Execution Time: {best_time:.4f}")
    logging.info(f"Best CPU Cost: {best_cpu_cost:.4f}")

    # Close connection
    conn.close()
    logging.info("Connection closed.")

if __name__ == "__main__":
    main()
\end{lstlisting}\vspace{.4cm}

The \texttt{main} function orchestrates the execution of the script. It establishes a database connection, defines the queries to optimize, and sets the PSO parameters. It then runs the PSO algorithm and logs the optimal materialized views, execution time, and CPU cost. Finally, it closes the database connection.

\subsubsection*{8. Execution}
The script is executed when run directly:
\begin{lstlisting}[language=Python]
if __name__ == "__main__":
    main()
\end{lstlisting}\vspace{.4cm}

This block ensures that the \texttt{main} function is executed only when the script is run directly, not when it is imported as a module.

 % Include the external file
\end{document}