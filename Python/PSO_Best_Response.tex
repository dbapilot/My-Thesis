
\begin{lstlisting}[style=pythonstyle, caption={Python script to automate optimal view.}, label={lst:pso_query_optimization}]

import pyodbc # A python  library for interacting with ODBC databases like MSSQL that helps to manage database connection
import random #It helps to generate random numbers and choice used to initialize particle positions and velocities
import time

# Connection parameters (Establish a connection to the MSSQL server using pyodbc)
server = 'T915-TEST-DB'  # server name that used to get data 
database = 'AccessAuditDB'  # Database name
driver_name = 'ODBC Driver 17 for SQL Server'  # ODBC  driver from pyodbc.drivers()
# Uncomment and add these if using SQL Server Authentication
# username = 'm.islam'
# password = 'your_password'

try:
    # Establish connection
    conn = pyodbc.connect(
        f'DRIVER={{{driver_name}}};'
        f'SERVER={server};'
        f'DATABASE={database};'
        'Trusted_Connection=yes;'  # Indicates that Windows Authentication used for Authentication
        # Uncomment these lines for SQL Authentication
        # f'UID={username};'
        # f'PWD={password};'
    )
    cursor = conn.cursor()
    print("Connection established!")

    #  A list of Queries corresponding to materialized views
    queries = [
        "SELECT * FROM TotalSalesByCustomer;",
        "SELECT * FROM TotalQuantityByProduct;",
        "SELECT * FROM MonthlySales;"
    ]

    # Cost function: Measure total query execution time for a set of materialized views 
    def cost_function(selected_views):
        total_time = 0
        if not any(selected_views):  # No views selected
            return float('inf')  # High cost for no selection
        for i, view in enumerate(selected_views):
            if view == 1:  # If view is selected
                start_time = time.time()
                cursor.execute(queries[i]) # Executes the SQL query for the selected view
                cursor.fetchall()
                total_time += time.time() - start_time # Total query execution time 
        return total_time

    # PSO parameters
    num_particles = 5 # Number of particles in the swarm 
    num_iterations = 10 #The number of iterations the algorithm will run
    num_queries = len(queries) #The number of queries (materialized views) to optimize.
    W = 0.5  # Inertia weight Controls the impact of the previous velocity on the current velocity.

    c1, c2 = 1.5, 1.5 # Encourages particles to move toward the personal/global best position.

    # Initialize particles
    particles = [{'position': [random.choice([0, 1]) for _ in range(num_queries)], #Represents a particle's selected views (1 = selected, 0 = not selected).
                  'velocity': [random.uniform(-1, 1) for _ in range(num_queries)], #Represents the rate of change for each view selection.
                  'best_position': None,
                  'best_cost': float('inf')} for _ in range(num_particles)] #The lowest cost (execution time) encountered by the particle.

    global_best_position = None  
    global_best_cost = float('inf')

    # PSO algorithm
    for iteration in range(num_iterations):
        for particle in particles:
            # Evaluate cost
            cost = cost_function(particle['position'])
            print(f"Particle position: {particle['position']}, Cost: {cost:.4f}")

            # Update personal best
            if cost < particle['best_cost']:
                particle['best_position'] = particle['position'][:]
                particle['best_cost'] = cost

            # Update global best
            if cost < global_best_cost:
                global_best_position = particle['position'][:]
                global_best_cost = cost

            # Update velocity and position using PSO formula 
            for i in range(num_queries):
                r1, r2 = random.random(), random.random()
                particle['velocity'][i] = (W * particle['velocity'][i] +
                                           c1 * r1 * (particle['best_position'][i] - particle['position'][i]) +
                                           c2 * r2 * (global_best_position[i] - particle['position'][i]))
                particle['position'][i] = 1 if random.random() < abs(particle['velocity'][i]) else 0

        print(f"Iteration {iteration + 1}: Best Cost = {global_best_cost:.4f}")

    print("Optimal Materialized Views:", global_best_position)

except pyodbc.Error as e:
    print("Error connecting to SQL Server:", e) # Catches and displays database connection errors.


finally:
    if 'conn' in locals() and conn:
        conn.close()
        print("Connection closed.")  #Ensures the database connection is closed after the script execution.



\end{lstlisting}