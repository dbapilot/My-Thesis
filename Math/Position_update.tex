
\begin{equation}
x_i(t+1) = x_i(t) + v_i(t+1)
\end{equation}

\textbf{Where:}
\begin{itemize}
    \item \( x_i(t) \): Current position of particle \( i \) at time \( t \).
    \item \( v_i(t+1) \): Updated velocity of particle \( i \) at time \( t+1 \).
    \item \( x_i(t+1) \): New position of particle \( i \) at the next time step.
\end{itemize}

%\textbf{Description:} The position update equation states that the new position of a particle is calculated by adding its updated velocity to its current position. This movement allows the particle to explore the search space based on both its previous position and the velocity determined by its own experience and that of the swarm.

