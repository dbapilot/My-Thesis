%\chapter{Conclution}
\section{Conclusion}
This study sought to examine the issue of maximizing database performance by using advanced techniques, including Materialized Views and the PSO algorithm. It is intended to reduce query execution time and CPU load while improving database efficiency in high-traffic transactional systems. 

The study began by identifying slow-performing queries in a database system, which served as the foundation for applying optimization strategies. MV were created to store precomputed results of these queries, significantly reducing query execution time by removing redundant computations. The use of MV in MSSQL Server showed notable improvements in performance, particularly for complex queries involving large datasets.

 The algorithm effectively invests in the search space for potential execution plans by using its metaheuristics characteristics, converging on solutions that minimize query response time. The experimental findings verified that PSO is a feasible approach for improving database performance, particularly in situations where conventional optimization techniques are inadequate.

The findings of this study underline the importance of integrating database-specific improvement, such as MVs, with advanced computational methods, such as PSO. This hybrid approach not only improves query performance but also offers a scalable solution for managing complicated and data-intensive systems. The results of implementation showed that MV effectively reduced query execution time by approximately up to \textbf{\textit{40}}\% and significantly lowered CPU utilization around \textbf{\textit{22}}\% compared to conventional query execution. However, frequent MV refreshes could add extra processing overhead. Therefore, scheduled or incremental refresh policies would be required to preserve system efficiency. Future research could look at the use of monitoring tools of algorithms or extend the current framework to distributed database systems, where optimization issues are even more evident, accurate, and secure.\vspace{.4cm}

In summary, this thesis contributes to the field of database optimization by demonstrating the effectiveness of MV and algorithms in lowering query execution time and improving overall system performance. The findings highlight the capability to integrate database design with computational intelligence, resulting in more efficient and intelligent data management systems.
 