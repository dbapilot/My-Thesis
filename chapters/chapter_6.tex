%\chapter{Conclution}
\section{Conclusion}
This thesis explored the implementation and assessment of advanced methodologies, including Particle Swarm Optimisation (PSO) and Materialised Views (Indexed Views), thereby addressing the optimisation of database performance. The study aimed to reduce query execution time and CPU load while enhancing database efficiency in high-volume transactional systems. 
Beginning with slow-performing searches in a database system, the project laid the groundwork for optimisation strategies. Eliminating duplicate computations facilitated the design of materialised views that store precomputed search results, thereby significantly reducing query execution time. Particularly for complex searches involving large datasets, the implementation of Indexed Views in MSSQL Server demonstrated notable efficiency improvements.

This thesis addressed the problem of optimizing database performance by means of sophisticated approaches, including MV(Indexed Views) and PSO algorithms. The research was driven by the need to reduce query execution time, minimize CPU load, and improve database efficiency in high-volume transactional environments. The study began by identifying slow-performing queries in a database system, which served as the foundation for applying optimization strategies. Materialized Views were created to store precomputed results of these queries, significantly reducing query execution time by eliminating redundant computations. The use of Indexed Views in MSSQL Server demonstrated notable improvements in performance, particularly for complex queries involving large datasets.
Furthermore, integrating the PSO technique provided a creative way to optimise query execution plans. By applying the algorithm's heuristic features, the algorithm efficiently invests in the search space for possible execution plans, converging on solutions that minimize query response time. The experimental results confirmed that PSO is a viable method for enhancing database performance, especially in scenarios where traditional optimization techniques fall short.
The implementation results demonstrated that Materialized Views effectively reduced query execution time by up to \textbf{[XX}\% and significantly lowered CPU utilization compared to traditional query execution. However, it was observed that frequent MV refreshes could introduce extra processing overhead, necessitating the use of scheduled or incremental refresh policies to maintain system efficiency.
The results of this research emphasizes the importance of combining database-specific optimizations, such as Materialized Views, with advanced computational techniques like PSO. This hybrid technique not only improves query performance but also offers a scalable solution for managing complicated and data-intensive applications. Future work could investigate the application of other metaheuristic algorithms or extend the current framework to distributed database systems, where optimization problems are even more evident.
In conclusion, this thesis contributes to the field of database optimization by proving the effectiveness of Materialized Views and PSO in lowering query execution time and enhancing overall system performance. The results highlight the potential of integrating database engineering with computational intelligence, paving the way for more efficient and intelligent data management systems.\\
\textbf{......\textbf{Work on Progress}..........}       