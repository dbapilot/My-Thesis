%\chapter{Implementaion}

\section{Implementations}
This section describes a step-by-step process for effectively deploying materialized views, and PSO algorithm based on the earlier theoretical foundations, to reduce query execution time, minimize computational overhead, and enhance overall performance. The implementation begins with an overview of the tools and technologies utilized in this project, providing essential context before delving into the optimization process.

\subsection{Used Software and Tools}
This project uses some software tools and technologies related to database management, query optimization, and technical documentation. Collectively, these tools support database management, query optimization, data analysis, and technical documentation throughout the research and development process.

\begin{enumerate}[label=(\roman*)]
\item\textbf{SQL Server Management Studio :} SQL Server Management Studio(SSMS) is actually an integrated environment that can maintain the SQL Server infrastructure. It is used to access, manage, configure, administer, and develop all components of SQL Server. In addition, it is used to manage the schema, tables, access to SQL Server, and materialized views of the database. Also, it monitors the performance of the query using execution plans and statistics, and it helps to debug SQL queries and optimize their execution. Here, MSSQL is used to host the database, create tables, and define materialized views, enabling query execution and performance measurement.


\item\textbf{{Visual Studio Code:}} Microsoft created this open-source integrated development environment (IDE) for web browsers, Linux, macOS, and Windows. It is used to write Python scripts for automation (e.g., measuring query performance), integrate MSSQL queries with Python using extensions, and debug Python and SQL scripts. VS Code is used as the primary Integrated Development Environment (IDE) for writing, debugging, and running Python scripts that implement the PSO algorithm and interact with the SQL Server database.

\item\textbf{Overleaf:} Overleaf is an open-source online, real-time collaborative LaTeX\footnote{LaTeX is a powerful typesetting system commonly used for academic and technical documents. It includes features designed for the production of technical and scientific documentation.} editor that simplifies the process of creating, editing, and collaborating on LaTeX documents. The whole project is written with the help of Overleaf.

\item\textbf{Microsoft SQL Server:} Microsoft SQL Server is a relational database that provides a wide range of features for storing, processing, and securing data. SQL Server hosts the database, stores the tables and materialized views, and executes the SQL queries, allowing the measurement and optimization of query performance using the PSO algorithm.


\item\textbf{Python:} Python is a high-level interpreted programming language known for its simplicity and readability. It was created by Guido van Rossum and released in 1991 \cite{martin2023stam,wijanarko2020prediksi} . It has become one of the most popular programming languages worldwide. Its object-oriented approach helps programmers to write logical and clear code for small and large projects. Python libraries (packages) effectively simplify many important processes such as analyzing and visualizing data, retrieving unstructured data from the web, image processing, building machine learning models, and textual information \cite{Samira_Gholizadeh2022}. Here, it has been used to implement the PSO algorithm to optimize query performance, connect to the database using libraries like pyodbc, measures, executes, and analyzes the database programmatically. Due to its simplicity, readability, extensive libraries, ease of use, and efficiency, it is one of the best choices for implementing PSO.

\item\textbf{pyodbc:} Pyodbc is an open-source Python module that makes accessing ODBC\footnote{Open Database Connectivity (ODBC) is a standardized application programming interface (API) for accessing databases.} databases simple. It implements the DB API\footnote{An Application Programming Interface (API) is a set of protocols, tools, and definitions that allow different software applications to communicate with each other.} 2.0 specification but is packed with even more Pythonic convenience. The pyodbc library connects Python to MSSQL, allowing programmatic execution of SQL queries and retrieval of results.

\item\textbf{pandas:} The pandas constitute an open-source data manipulation and analysis tool that is fast, powerful, flexible, and easy to use. It provides data structures like DataFrames and Series, which are particularly useful for working with structured data. It is built entirely on the Python programming language. Pandas is used to analyze query performance data, such as execution times and storage costs, for better insights.

\item\textbf{matplotlib:} Matplotlib is a comprehensive library for creating static, animated, and interactive visualizations in Python. \enquote{Matplotlib makes easy things easy and hard things possible} \cite{matplotlib}. It has been used to visualize query performance metrics and PSO convergence, making results easier to interpret.

\item\textbf{GitHub:} GitHub is an open-source version control system that is used for tracking the changes made in the files and for enabling a collaborative software world. GitHub is incredibly popular due to its flexibility, speed, and ability to support almost any workflow. It has become the world's largest source code host, with around 90 per cent of developers worldwide using it to create, store, manage, and share their code. GitHub is used to track changes in Python scripts, LaTeX documents, and SQL files, making it an essential tool for collaboration and version control during thesis development. The platform’s features, such as branching, pull requests, and issue tracking, enable efficient project management and seamless integration of feedback from advisors or peers. Additionally, GitHub’s public or private repository options allow me to share my work with the academic community while retaining control over access.

\end{enumerate}

\clearpage

\subsection{Practical Implementation }

\subsubsection{Database Creation}
Choosing a database or creating one if not available is the initial step of the implementation. For the demonstration purpose of this thesis only, one database called \texttt{HealthInsuranceDB} is created here. The database was populated with tables and sample data to simulate a real-world environment. Relationships among tables are created through primary keys and foreign keys to ensure data integrity and efficient querying, enabling the testing and optimization of queries using materialized views and the PSO algorithm.

The following SQL query checks if the database \texttt{HealthInsuranceDB} exists and creates it if it does not: \vspace{.4cm}


% Define SQL style
\lstdefinestyle{sql}{
    language=SQL,
    backgroundcolor=\color{white},
    basicstyle=\ttfamily\footnotesize,
    keywordstyle=\color{blue},
    commentstyle=\color{green!40!black},
    stringstyle=\color{red},
    showstringspaces=false,
    breaklines=true,
    frame=single,
    rulecolor=\color{black},
    tabsize=2,
    captionpos=b,
    aboveskip=10pt,
    belowskip=10pt
}


The following SQL query checks if the database \texttt{AccessAuditDB} exists and creates it if it does not:

\begin{lstlisting}[style=sql, caption={SQL Query to Create Database}, label={lst:sql-create-db}]
IF NOT EXISTS (SELECT * FROM sys.databases WHERE name = 'AccessAuditDB')
BEGIN
    CREATE DATABASE AccessAuditDB;
END
GO
USE AccessAuditDB;
\end{lstlisting}




\subsubsection{Table Creation} A database schema with four tables, namely \texttt{InsuranceProviders}, \texttt{Patients}, \texttt{Claims}, and \texttt{Treatments}, are created with the help of the following SQL query. These tables are intended to store information on insurance providers, patient details, insurance claims, and associated treatments respectively. Foreign key constraints ensure data integrity as they set up relationships between the tables, making this schema suitable for demonstrating materialized views and query optimization in a health insurance context. \vspace{.4cm}

% Define colors
\definecolor{codegreen}{rgb}{0,0.6,0}  % ✅ Green for comments
\definecolor{codegray}{rgb}{0.5,0.5,0.5}  % ✅ Gray for numbers
\definecolor{codepurple}{rgb}{0.58,0,0.82}  % ✅ Purple for strings
\definecolor{backcolour}{rgb}{0.95,0.95,0.92}  % ✅ Light gray background
\definecolor{bordercolor}{rgb}{0.7,0.7,0.7}  % ✅ Left border color (gray)
\definecolor{codeblue}{rgb}{0,0,0.8}  % ✅ Blue for SQL keywords

% ✅ Define SQL language with correct comment handling
\lstdefinelanguage{MySQL}{
    keywords={SELECT, FROM, WHERE, JOIN, ON, INNER, OUTER, LEFT, RIGHT, FULL, GROUP, BY, ORDER, ASC, DESC, AS, COUNT, SUM, AVG, MAX, MIN, DISTINCT, INSERT, INTO, VALUES, UPDATE, SET, DELETE, CREATE, TABLE, PRIMARY, FOREIGN, KEY, DEFAULT, NULL, NOT, CHECK, CONSTRAINT, INDEX, VIEW, MATERIALIZED, PROCEDURE, FUNCTION, TRIGGER, DATABASE, ALTER, DROP, EXEC, IF, EXISTS, UNION, ALL, CASE, WHEN, THEN, ELSE, END, CAST, CONVERT, LIKE, IN, BETWEEN, AND, OR, HAVING, LIMIT, OFFSET},
    sensitive=false,
    morestring=[b]',  % ✅ Strings in single quotes
    morestring=[b]",  % ✅ Strings in double quotes
    morecomment=[l][\color{codegreen}]{--}  % ✅ Ensures full line comment in green
}

\lstdefinestyle{sqlstyle}{
    backgroundcolor=\color{backcolour},   
    commentstyle=\color{codegreen},  % ✅ Comments in green
    keywordstyle=\bfseries\color{codeblue},  % ✅ SQL Keywords in Blue & Bold
    numberstyle=\scriptsize\color{codegray},  % ✅ Row numbers in gray
    stringstyle=\color{codepurple},  % ✅ Strings in purple
    basicstyle=\ttfamily\footnotesize,
    breaklines=true,
    captionpos=b,
    numbers=left,      % ✅ Enables row numbers on the left
    stepnumber=1,      % ✅ Row numbers increment by 1
    firstnumber=1,     % ✅ Starts numbering at 1
    numbersep=8pt,     % ✅ Increases space between numbers and SQL code
    xleftmargin=3em,   % ✅ Ensures space inside the left border
    frame=single,      % ✅ Keeps a single border (left-aligned)
    framesep=5pt,      % ✅ Ensures space inside the frame
    rulesepcolor=\color{bordercolor},  % ✅ Matches row numbers with left border
    rulecolor=\color{bordercolor},  % ✅ Sets left border color
    language=MySQL  % ✅ Uses SQL keyword highlighting
}

\begin{lstlisting}[style=sqlstyle, caption={SQL query to Create Database}]
-- Create InsuranceProviders Table
CREATE TABLE InsuranceProviders (
    InsuranceProviderID INT PRIMARY KEY IDENTITY(1,1),
    ProviderName NVARCHAR(100) NOT NULL,
    Address NVARCHAR(255),
    City NVARCHAR(100),
    State NVARCHAR(50),
    ZipCode NVARCHAR(20)
);

-- Create Patients Table
CREATE TABLE Patients (
    PatientID INT PRIMARY KEY IDENTITY(1,1),
    FirstName NVARCHAR(50) NOT NULL,
    LastName NVARCHAR(50) NOT NULL,
    DateOfBirth DATE NOT NULL,
    Gender CHAR(1) CHECK (Gender IN ('M', 'F', 'O')), -- M: Male, F: Female, O: Other
    Address NVARCHAR(255),
    City NVARCHAR(100),
    State NVARCHAR(50),
    ZipCode NVARCHAR(20),
    InsuranceProviderID INT FOREIGN KEY REFERENCES InsuranceProviders(InsuranceProviderID)
);

-- Create Claims Table
CREATE TABLE Claims (
    ClaimID INT PRIMARY KEY IDENTITY(1,1),
    PatientID INT FOREIGN KEY REFERENCES Patients(PatientID),
    InsuranceProviderID INT FOREIGN KEY REFERENCES InsuranceProviders(InsuranceProviderID),
    ClaimDate DATE NOT NULL,
    ClaimAmount DECIMAL(18, 2) NOT NULL,
    Status NVARCHAR(50) CHECK (Status IN ('Pending', 'Approved', 'Rejected'))
);

-- Create Treatments Table
CREATE TABLE Treatments (
    TreatmentID INT PRIMARY KEY IDENTITY(1,1),
    ClaimID INT FOREIGN KEY REFERENCES Claims(ClaimID),
    TreatmentDate DATE NOT NULL,
    TreatmentType NVARCHAR(100) NOT NULL,
    Cost DECIMAL(18, 2) NOT NULL
);
GO
\end{lstlisting}

\subsubsection{Inserting Random Data in Database}
One million patient records are automatically generated through a looped SQL script~\ref{lst:inserting_data}. Random values are assigned for date of birth (ranging from 0-100 years), gender (evenly distributed among M/F/O), and location data (address, city, state, and zip code).\vspace{.4cm}

 \input{SQL/inserting_data}

\subsubsection{Identify Complex Queries} This is a very important part of identifying the frequent queries that need to be optimized. It refers to the process of analyzing a database workload to pinpoint queries that are resource-intensive, frequently executed, or critical to performance. SQL Server Management Studio or direct SQL query can be used to analyze execution logs and identify the frequently executed and resource-intensive queries. Also, identify frequently used sub-queries, aggregation (Queries with GROUP BY, SUM, AVG)s, or frequent joins between tables. For example, the following query to identify long-running queries: \vspace{.4cm}

\input{SQL/Query_identify}

As shown in Listing~\ref{lst:IdentifyComplexQueries}, query retrieves the top 10 most time-consuming queries from SQL Server by analyzing the \(\texttt{sys.dm\_exec\_query\_stats}\) Dynamic Management Views(DMV)\footnote{DMVs are system defined views for database administrators and developers to troubleshoot performance issues, identify missing indexes, analyze query performance, and monitor resource usage efficiently.}. It calculates the total elapsed time in milliseconds, counts the number of executions, and fetches the query text using \(\texttt{sys.dm\_exec\_sql\_text}\). The results are sorted by \(\texttt{total\_elapsed\_time}\) in descending order to highlight the slowest queries for performance tuning and optimization. These queries were then used to create materialized views, which were optimized using the PSO algorithm to improve overall query performance.

\subsubsection{ Materialized views Creation (Indexed Views)}\label{Query_decomposition} Once the top slow queries are sorted out, MVs(In MSSQL Server Mvs are implemented as Indexed views ) are created for each of these queries to store their precomputed results.\vspace{.4cm}

  %\input{SQL/Create_MSSQL}
  % Define colors
\definecolor{codegreen}{rgb}{0,0.6,0}  % ✅ Green for comments
\definecolor{codegray}{rgb}{0.5,0.5,0.5}  % ✅ Gray for numbers
\definecolor{codepurple}{rgb}{0.58,0,0.82}  % ✅ Purple for strings
\definecolor{backcolour}{rgb}{0.95,0.95,0.92}  % ✅ Light gray background
\definecolor{bordercolor}{rgb}{0.7,0.7,0.7}  % ✅ Left border color (gray)
\definecolor{codeblue}{rgb}{0,0,0.8}  % ✅ Blue for SQL keywords

\definecolor{commentcolor}{RGB}{0, 128, 0}  % Green for comments
\definecolor{titlecolor}{RGB}{0, 0, 255}    % Blue for titles
% ✅ Define SQL language with correct comment handling
\lstdefinelanguage{MySQL}{
    keywords={SELECT, FROM, WHERE, JOIN, ON, INNER, OUTER, LEFT, RIGHT, FULL, GROUP, BY, ORDER, ASC, DESC, AS, COUNT, SUM, AVG, MAX, MIN, DISTINCT, INSERT, INTO, VALUES, UPDATE, SET, DELETE, CREATE, TABLE, PRIMARY, FOREIGN, KEY, DEFAULT, NULL, NOT, CHECK, CONSTRAINT, INDEX, VIEW, MATERIALIZED, PROCEDURE, FUNCTION, TRIGGER, DATABASE, ALTER, DROP, EXEC, IF, EXISTS, UNION, ALL, CASE, WHEN, THEN, ELSE, END, CAST, CONVERT, LIKE, IN, BETWEEN, AND, OR, HAVING, LIMIT, OFFSET},
    sensitive=false,
    morestring=[b]',  % ✅ Strings in single quotes
    morestring=[b]",  % ✅ Strings in double quotes
    morecomment=[l][\color{codegreen}]{--}  % ✅ Ensures full line comment in green
}

\lstdefinestyle{sqlstyle}{
    backgroundcolor=\color{backcolour},   
    commentstyle=\color{codegreen},  % ✅ Comments in green
    keywordstyle=\bfseries\color{codeblue},  % ✅ SQL Keywords in Blue & Bold
    numberstyle=\scriptsize\color{codegray},  % ✅ Row numbers in gray
    stringstyle=\color{codepurple},  % ✅ Strings in purple
    basicstyle=\ttfamily\footnotesize,
    breaklines=true,
    captionpos=b,
    title=\color{titlecolor}\textbf{Table Creation Script}, % Title in blue
    numbers=left,      % ✅ Enables row numbers on the left
    stepnumber=1,      % ✅ Row numbers increment by 1
    firstnumber=1,     % ✅ Starts numbering at 1
    numbersep=8pt,     % ✅ Increases space between numbers and SQL code
    xleftmargin=3em,   % ✅ Ensures space inside the left border
    frame=single,      % ✅ Keeps a single border (left-aligned)
    framesep=5pt,      % ✅ Ensures space inside the frame
    rulesepcolor=\color{bordercolor},  % ✅ Matches row numbers with left border
    rulecolor=\color{bordercolor},  % ✅ Sets left border color
    language=MySQL  % ✅ Uses SQL keyword highlighting
}
  
\begin{lstlisting}[style=sqlstyle, caption={Materialized view creation}, label=lst:MV_creation]
--Materialized View for Query 1: Total Claims By Patient
CREATE VIEW TotalClaimsByPatient
WITH SCHEMABINDING
AS
SELECT PatientID, COUNT_BIG(*) AS TotalClaims
FROM dbo.Claims
GROUP BY PatientID;
GO

CREATE UNIQUE CLUSTERED INDEX IX_TotalClaimsByPatient
ON TotalClaimsByPatient (PatientID);
GO

-- This view calculates the total number of claims submitted by each patient. It groups the data by PatientID and uses COUNT_BIG(*) to count all related claims.



--Materialized View for Query 2: TotalTreatmentCostByProvider


CREATE VIEW TotalTreatmentCostByProvider
WITH SCHEMABINDING
AS
SELECT c.InsuranceProviderID, SUM(t.Cost) AS TotalCost
FROM dbo.Treatments t
JOIN dbo.Claims c ON t.ClaimID = c.ClaimID
GROUP BY c.InsuranceProviderID;
GO

CREATE UNIQUE CLUSTERED INDEX IX_TotalTreatmentCostByProvider
ON TotalTreatmentCostByProvider (InsuranceProviderID);
GO

-- This view computes the total treatment cost for each insurance provider. It joins the Treatments and Claims tables and sums up the Cost field, grouped by InsuranceProviderID.

--Materialized View for Query 3: MonthlyClaimsByProvider

CREATE VIEW MonthlyClaimsByProvider
WITH SCHEMABINDING
AS
SELECT 
    c.InsuranceProviderID, 
    YEAR(c.ClaimDate) AS ClaimYear, 
    MONTH(c.ClaimDate) AS ClaimMonth, 
    COUNT_BIG(*) AS TotalClaims
FROM dbo.Claims c
GROUP BY c.InsuranceProviderID, YEAR(c.ClaimDate), MONTH(c.ClaimDate);
GO

CREATE UNIQUE CLUSTERED INDEX IX_MonthlyClaimsByProvider
ON MonthlyClaimsByProvider (InsuranceProviderID, ClaimYear, ClaimMonth);
GO

-- This view summarizes the number of claims submitted per insurance provider for each month. It extracts the year and month from ClaimDate and counts claims grouped by provider, year, and month.
\end{lstlisting}\vspace{.4cm} 



These materialized/indexed views in listing ~\ref{lst:MV_creation} are designed to optimize query performance in a health insurance database by precomputing and storing aggregated results. The \texttt{TotalClaimsByPatient} view demonstrates the retrieval of claim counts per patient, while the \texttt{TotalTreatmentCostByProvider} view offers a rapid summary of treatment costs by the insurance provider. The \texttt{MonthlyClaimsByProvider} view breaks down monthly claims in detail, enabling efficient analysis of claim trends over time. By materializing these views with unique clustered indexes, the database ensures faster query execution and reduced computational costs on data with frequent access.



  \subsubsection{View Maintenance and Refresh Strategies}\label{View_maintainance} Incremental, Manual or automatic refresh strategies can be set up according to the requirements query to create a scheduled job with the help of script as in the listing ~\ref{lst:Maintenance_and_Refresh_Strategies}.\vspace{.4cm}

% Define colors
\definecolor{codegreen}{rgb}{0,0.6,0}  % ✅ Green for comments
\definecolor{codegray}{rgb}{0.5,0.5,0.5}  % ✅ Gray for numbers
\definecolor{codepurple}{rgb}{0.58,0,0.82}  % ✅ Purple for strings
\definecolor{backcolour}{rgb}{0.95,0.95,0.92}  % ✅ Light gray background
\definecolor{bordercolor}{rgb}{0.7,0.7,0.7}  % ✅ Left border color (gray)
\definecolor{codeblue}{rgb}{0,0,0.8}  % ✅ Blue for SQL keywords

% ✅ Define SQL language with correct comment handling
\lstdefinelanguage{MySQL}{
    keywords={SELECT, FROM, WHERE, JOIN, ON, INNER, OUTER, LEFT, RIGHT, FULL, GROUP, BY, ORDER, ASC, DESC, AS, COUNT, SUM, AVG, MAX, MIN, DISTINCT, INSERT, INTO, VALUES, UPDATE, SET, DELETE, CREATE, TABLE, PRIMARY, FOREIGN, KEY, DEFAULT, NULL, NOT, CHECK, CONSTRAINT, INDEX, VIEW, MATERIALIZED, PROCEDURE, FUNCTION, TRIGGER, DATABASE, ALTER, DROP, EXEC, IF, EXISTS, UNION, ALL, CASE, WHEN, THEN, ELSE, END, CAST, CONVERT, LIKE, IN, BETWEEN, AND, OR, HAVING, LIMIT, OFFSET},
    sensitive=false,
    morestring=[b]',  % ✅ Strings in single quotes
    morestring=[b]",  % ✅ Strings in double quotes
    morecomment=[l][\color{codegreen}]{--}  % ✅ Ensures full line comment in green
}

\lstdefinestyle{sqlstyle}{
    backgroundcolor=\color{backcolour},   
    commentstyle=\color{codegreen},  % ✅ Comments in green
    keywordstyle=\bfseries\color{codeblue},  % ✅ SQL Keywords in Blue & Bold
    numberstyle=\scriptsize\color{codegray},  % ✅ Row numbers in gray
    stringstyle=\color{codepurple},  % ✅ Strings in purple
    basicstyle=\ttfamily\footnotesize,
    breaklines=true,
    captionpos=b,
    numbers=left,      % ✅ Enables row numbers on the left
    stepnumber=1,      % ✅ Row numbers increment by 1
    firstnumber=1,     % ✅ Starts numbering at 1
    numbersep=8pt,     % ✅ Increases space between numbers and SQL code
    xleftmargin=3em,   % ✅ Ensures space inside the left border
    frame=single,      % ✅ Keeps a single border (left-aligned)
    framesep=5pt,      % ✅ Ensures space inside the frame
    rulesepcolor=\color{bordercolor},  % ✅ Matches row numbers with left border
    rulecolor=\color{bordercolor},  % ✅ Sets left border color
    language=MySQL  % ✅ Uses SQL keyword highlighting
}
         \begin{lstlisting}[style=sqlstyle]
-- Example of creating a job to refresh a materialized view automatically every hour

USE msdb;
GO

-- Create the job
EXEC dbo.sp_add_job
    @job_name = N'RefreshMaterializedViewJob',
    @enabled = 1,
    @description = N'Job to refresh the materialized view every hour.';

-- Add a job step
EXEC sp_add_jobstep
    @job_name = N'RefreshMaterializedViewJob',
    @step_name = N'RefreshViewStep',
    @subsystem = N'TSQL',
    @command = N'EXEC RefreshMaterializedView;',
    @retry_attempts = 3,
    @retry_interval = 5;

-- Create a schedule for the job
EXEC sp_add_schedule
    @schedule_name = N'HourlySchedule',
    @freq_type = 4, -- Daily
    @freq_interval = 1, -- Every day
    @freq_subday_type = 8, -- Hourly
    @freq_subday_interval = 1, -- Every 1 hour
    @active_start_time = 000000; -- Start time (midnight)

-- Attach the schedule to the job
EXEC sp_attach_schedule
    @job_name = N'RefreshMaterializedViewJob',
    @schedule_name = N'HourlySchedule';

-- Assign the job to the SQL Server Agent service
EXEC dbo.Sp_add_jobserver
     @job_name = N'RefreshMaterializedViewJob';

go 

        \end{lstlisting}\vspace{.4cm}

As the database is created here only for demonstration purposes and all data are inserted once manually, therefore materialized views are refreshed manually. This approach is suitable for scenarios where immediate updates are required or when automated refresh mechanisms (like triggers or scheduled jobs) are not feasible. Manual refresh ensures that the views reflect the latest changes in the underlying data, providing accurate results for query processing.\vspace{.4cm}

\subsection{Automation with the PSO algorithm:} \label{Cost_evaluation}
 The following Python code illustrates the implementation of the PSO algorithm for multiple values in the database system. The goal is to minimize query execution time and CPU cost by selecting the most beneficial views to materialize. This section provides a detailed explanation of each function in the implementation.



%\section*{\textbf{Code example to selecting the optimal view using PSO algorithm.} \vspace{.4cm}}


\definecolor{codegreen}{rgb}{0,0.6,0}
\definecolor{codegray}{rgb}{0.5,0.5,0.5}
\definecolor{codepurple}{rgb}{0.58,0,0.82}
\definecolor{backcolour}{rgb}{0.95,0.95,0.92}

\lstdefinestyle{mystyle}{
    backgroundcolor=\color{backcolour},   
    commentstyle=\color{codegreen},
    keywordstyle=\color{magenta},
    numberstyle=\tiny\color{codegray},
    stringstyle=\color{codepurple},
    basicstyle=\ttfamily\footnotesize,
    breakatwhitespace=false,         
    breaklines=true,                 
    captionpos=b,                    
    keepspaces=true,                 
    numbers=left,                    
    numbersep=5pt,                  
    showspaces=false,                
    showstringspaces=false,
    showtabs=false,                  
    tabsize=2
}

\lstset{style=mystyle}



\subsection*{Python Code Explanation}

\subsubsection*{1. Importing Libraries}
The following Python libraries are imported for the script:
\begin{lstlisting}[language=Python]
import pyodbc  # A Python library for interacting with ODBC databases like MSSQL.
import random  # Used to generate random numbers for particle initialization.
import time    # Used to measure execution time.
import numpy as np  # Used for numerical computations.
import logging  # Used for logging information, warnings, and errors.
\end{lstlisting}\vspace{.4cm}
The \texttt{pyodbc} library is used to connect to and interact with the SQL Server database. The \texttt{random} library is used to generate random numbers for initializing particle positions and velocities in the Particle Swarm Optimization (PSO) algorithm. \texttt{Time} library is used to determine the execution time of queries. The \texttt{numpy} library is used for numerical calculations, such as computing the sigmoid function. The \texttt{logging} library is used to log messages, warnings, and errors during the execution of the script.

\subsubsection*{2. Configuring Logging}
The logging system is configured to display messages with a timestamp, log level, and message:
\begin{lstlisting}[language=Python]
logging.basicConfig(level=logging.INFO, format='%(asctime)s - %(levelname)s - %(message)s')
\end{lstlisting}\vspace{.4cm}

This configuration ensures that all log messages are displayed with a timestamp, log level (e.g., INFO, ERROR), and the message itself. This helps in tracking the execution flow and debugging issues.

\subsubsection*{3. Connection Parameters}
The connection parameters for the SQL Server database are defined:
\begin{lstlisting}[language=Python]
server = 'T915-TEST-DB'  # Server name
database = 'AccessAuditDB'  # Database name
driver_name = 'ODBC Driver 17 for SQL Server'  # ODBC driver
\end{lstlisting}\vspace{.4cm}

The \texttt{server} variable holds the name of the server, \texttt{database} specifies the database to connect to, and \texttt{driver\_name} specifies the ODBC driver to use for the connection. These parameters are used to establish a connection to the database.

\subsubsection*{4. Establishing a Database Connection}
The \texttt{create\_connection} function establishes a connection to the database:
\begin{lstlisting}[language=Python]
def create_connection():
    try:
        conn = pyodbc.connect(
            f'DRIVER={{{driver_name}}};'
            f'SERVER={server};'
            f'DATABASE={database};'
            'Trusted_Connection=yes;'  # Windows Authentication
        )
        logging.info("Connection established!")
        logging.info(f"Connected to database: {database} on server: {server}")
        return conn
    except pyodbc.Error as e:
        logging.error("Error connecting to SQL Server:", exc_info=True)
        return None
\end{lstlisting}\vspace{.4cm}

This function attempts to establish a connection to the SQL Server database using the \texttt{pyodbc.connect} method. If the connection is successful, it logs a success message and returns the connection object. If an error occurs, it logs the error and returns \texttt{None}.

\subsubsection*{5. Cost Function}
The \texttt{cost\_function} calculates the total cost of executing queries based on selected materialized views:
\begin{lstlisting}[language=Python]
def cost_function(selected_views, queries, conn):
    total_time = 0
    cpu_cost = 0

    if not any(selected_views):  # No views selected
        return float('inf'), 0, 0  # High cost for no selection

    cursor = conn.cursor()  # Create a new cursor
    for i, view in enumerate(selected_views):
        if view == 1:  # If view is selected
            start_time = time.time()
            try:
                logging.info(f"Executing query: {queries[i]}")
                cursor.execute(queries[i])
                cursor.fetchall()
                execution_time = time.time() - start_time
                total_time += execution_time

                # Estimate CPU cost
                query_complexity = queries[i].upper().count('JOIN') + 1
                cpu_cost += execution_time * query_complexity

            except pyodbc.Error as e:
                logging.error(f"Error executing query {i}: {queries[i]}", exc_info=True)
                return float('inf'), 0, 0  # High cost for query errors

    # Weighted cost function
    alpha, beta = 0.7, 0.3  # Weights for execution time and CPU cost
    total_cost = alpha * total_time + beta * cpu_cost
    return total_cost, total_time, cpu_cost
\end{lstlisting}\vspace{.4cm}

The \texttt{cost\_function} calculates the total cost of executing a set of queries based on the selected materialized views. It measures the execution time and estimates the CPU cost for each query. If no views are selected, it returns a high cost (\texttt{float('inf')}). The total cost is a weighted sum of execution time and CPU cost, where execution time contributes 70\% and CPU cost contributes 30\%.

\subsubsection*{6. Advanced PSO Algorithm}
The \texttt{pso} function implements the Particle Swarm Optimization (PSO) algorithm:
\begin{lstlisting}[language=Python]
def pso(num_particles, num_iterations, num_queries, queries, conn):
    # PSO parameters
    W_max = 0.9  # Maximum inertia weight
    W_min = 0.4  # Minimum inertia weight
    c1, c2 = 1.5, 1.5  # Cognitive and social factors
    v_max = 6.0  # Maximum velocity for clamping

    # Initialize particles
    particles = [{'position': [random.choice([0, 1]) for _ in range(num_queries)],
                 'velocity': [random.uniform(-1, 1) for _ in range(num_queries)],
                 'best_position': None,
                 'best_cost': float('inf')} for _ in range(num_particles)]

    global_best_position = None
    global_best_cost = float('inf')
    global_best_time = 0
    global_best_cpu_cost = 0

    # PSO main loop
    for iteration in range(num_iterations):
        logging.info(f"Iteration {iteration + 1} started.")
        
        # Dynamic inertia weight
        W = W_max - (W_max - W_min) * (iteration / num_iterations)

        for particle in particles:
            cost, execution_time, cpu_cost = cost_function(particle['position'], queries, conn)
            particle['cost'] = cost

            # Update personal best
            if cost < particle['best_cost']:
                particle['best_position'] = particle['position'][:]
                particle['best_cost'] = cost

            # Update global best
            if cost < global_best_cost:
                global_best_position = particle['position'][:]
                global_best_cost = cost
                global_best_time = execution_time
                global_best_cpu_cost = cpu_cost

        # Update velocity and position
        for particle in particles:
            for i in range(num_queries):
                r1, r2 = random.random(), random.random()
                # Update velocity
                particle['velocity'][i] = (W * particle['velocity'][i] +
                                          c1 * r1 * (particle['best_position'][i] - particle['position'][i]) +
                                          c2 * r2 * (global_best_position[i] - particle['position'][i]))
                # Clamp velocity
                particle['velocity'][i] = max(min(particle['velocity'][i], v_max), -v_max)
                # Update position using sigmoid function
                sigmoid = 1 / (1 + np.exp(-particle['velocity'][i]))
                particle['position'][i] = 1 if random.random() < sigmoid else 0

        # Log iteration results
        logging.info(f"Iteration {iteration + 1}: Best Cost = {global_best_cost:.4f}, Execution Time = {global_best_time:.4f}, CPU Cost = {global_best_cpu_cost:.4f}")

    return global_best_position, global_best_cost, global_best_time, global_best_cpu_cost
\end{lstlisting}\vspace{.4cm}

The \texttt{pso} function implements the Particle Swarm Optimization (PSO) algorithm. It initializes particles with random positions and velocities. The algorithm iteratively updates the particles' positions and velocities based on their personal best and the global best. The inertia weight (\texttt{W}) decreases over time to balance exploration and exploitation. The algorithm logs the best cost, execution time, and CPU cost for each iteration.

\subsubsection*{7. Main Function}
The \texttt{main} function orchestrates the execution of the script:
\begin{lstlisting}[language=Python]
def main():
    conn = create_connection()
    if not conn:
        return

    # List of queries
    queries = [
        "SELECT * FROM TotalSalesByCustomer",  
        "SELECT * FROM TotalQuantityByProduct",  
        "SELECT * FROM MonthlySales"  
    ]

    # PSO parameters
    num_particles = 5  # Number of particles
    num_iterations = 5  # Number of iterations
    num_queries = len(queries)  # Number of queries

    # Run PSO
    logging.info("Starting PSO algorithm...")
    logging.info(f"Number of particles: {num_particles}")
    logging.info(f"Number of iterations: {num_iterations}")
    best_position, best_cost, best_time, best_cpu_cost = pso(num_particles, num_iterations, num_queries, queries, conn)

    # Output optimal materialized views
    optimal_views = [queries[i] for i, view in enumerate(best_position) if view == 1]
    logging.info("Optimal Materialized Views:")
    for view in optimal_views:
        logging.info(f"- {view}")
    logging.info(f"Best Execution Time: {best_time:.4f}")
    logging.info(f"Best CPU Cost: {best_cpu_cost:.4f}")

    # Close connection
    conn.close()
    logging.info("Connection closed.")

if __name__ == "__main__":
    main()
\end{lstlisting}\vspace{.4cm}

The \texttt{main} function orchestrates the execution of the script. It establishes a database connection, defines the queries to optimize, and sets the PSO parameters. It then runs the PSO algorithm and logs the optimal materialized views, execution time, and CPU cost. Finally, it closes the database connection.

\subsubsection*{8. Execution}
The script is executed when run directly:
\begin{lstlisting}[language=Python]
if __name__ == "__main__":
    main()
\end{lstlisting}\vspace{.4cm}

This block ensures that the \texttt{main} function is executed only when the script is run directly, not when it is imported as a module.

\vspace{.4cm}

A sequence diagram in figure ~\ref{fig:Sequence_diagram} is included here to help better understand the workflow and interactions between the system's components. It provides a step-by-step visualisation of how this code operates, making it more understandable.
  
 % 
\begin{lstlisting}[style=pythonstyle, caption={Python script to automate optimal view.}, label={lst:pso_query_optimization}]

# 
import pyodbc # A python  library for interacting with ODBC databases like MSSQL that helps to manage database connection
import random #It helps to generate random numbers and choice used to initialize particle positions and velocities
import time

# Connection parameters
server = 'T915-TEST-DB'  # server name that used to get data 
database = 'AccessAuditDB'  # Database name
driver_name = 'ODBC Driver 17 for SQL Server'  # ODBC  driver from pyodbc.drivers()
# Uncomment and add these if using SQL Server Authentication
# username = 'm.islam'
# password = 'your_password'

try:
    # Establish connection
    conn = pyodbc.connect(
        f'DRIVER={{{driver_name}}};'
        f'SERVER={server};'
        f'DATABASE={database};'
        'Trusted_Connection=yes;'  # Indicates that Windows Authentication used for Authentication
        # Uncomment these lines for SQL Authentication
        # f'UID={username};'
        # f'PWD={password};'
    )
    cursor = conn.cursor()
    print("Connection established!")

    #  A list of Queries corresponding to materialized views
    queries = [
        "SELECT * FROM TotalSalesByCustomer;",
        "SELECT * FROM TotalQuantityByProduct;",
        "SELECT * FROM MonthlySales;"
    ]

    # Cost function: Measure total query execution time for a set of materialized views 
    def cost_function(selected_views):
        total_time = 0
        if not any(selected_views):  # No views selected
            return float('inf')  # High cost for no selection
        for i, view in enumerate(selected_views):
            if view == 1:  # If view is selected
                start_time = time.time()
                cursor.execute(queries[i]) # Executes the SQL query for the selected view
                cursor.fetchall()
                total_time += time.time() - start_time # Total query execution time 
        return total_time

    # PSO parameters
    num_particles = 5 # Number of particles in the swarm 
    num_iterations = 5 #The number of iterations the algorithm will run
    num_queries = len(queries) #The number of queries (materialized views) to optimize.
    W = 0.5  # Inertia weight Controls the impact of the previous velocity on the current velocity.

    c1, c2 = 1.5, 1.5 # Encourages particles to move toward the personal/global best position.

    # Initialize particles
    particles = [{'position': [random.choice([0, 1]) for _ in range(num_queries)], #Represents a particle's selected views (1 = selected, 0 = not selected).
                  'velocity': [random.uniform(-1, 1) for _ in range(num_queries)], #Represents the rate of change for each view selection.
                  'best_position': None,
                  'best_cost': float('inf')} for _ in range(num_particles)] #The lowest cost (execution time) encountered by the particle.

    global_best_position = None  
    global_best_cost = float('inf')

    # PSO algorithm
    for iteration in range(num_iterations):
        for particle in particles:
            # Evaluate cost
            cost = cost_function(particle['position'])
            print(f"Particle position: {particle['position']}, Cost: {cost:.4f}")

            # Update personal best
            if cost < particle['best_cost']:
                particle['best_position'] = particle['position'][:]
                particle['best_cost'] = cost

            # Update global best
            if cost < global_best_cost:
                global_best_position = particle['position'][:]
                global_best_cost = cost

            # Update velocity and position using PSO formula 
            for i in range(num_queries):
                r1, r2 = random.random(), random.random()
                particle['velocity'][i] = (W * particle['velocity'][i] +
                                           c1 * r1 * (particle['best_position'][i] - particle['position'][i]) +
                                           c2 * r2 * (global_best_position[i] - particle['position'][i]))
                particle['position'][i] = 1 if random.random() < abs(particle['velocity'][i]) else 0

        print(f"Iteration {iteration + 1}: Best Cost = {global_best_cost:.4f}")

    print("Optimal Materialized Views:", global_best_position)

except pyodbc.Error as e:
    print("Error connecting to SQL Server:", e) # Catches and displays database connection errors.


finally:
    if 'conn' in locals() and conn:
        conn.close()
        print("Connection closed.")  #Ensures the database connection is closed after the script execution.





\end{lstlisting} \vspace{.4cm}


\clearpage


% Define block styles


% Inserting the image
\begin{figure}[h]
    \centering
    \includegraphics[width=0.5\textwidth]{Figure/seq.diagram .png} % Replace 'example-image' with your image file name
    \caption{Sequence diagram of code}
    \label{fig:Sequence_diagram}
\end{figure}






\subsubsection{Output from PSO automation }  Upon execution for a specified number of iterations, the code records the best combination of materialized views that minimize the total cost. It makes a list of the iteration-wise data, including the best execution time, CPU cost and the corresponding views for each iteration. Finally, it prints the overall best view according to their parameters. Below is the sample of the output from the python code: \vspace{.4cm}



  

\begin{lstlisting}[style=pythonstyle, caption={Output from python code }, label={lst:pso_query_optimization}]]

2025-03-17 16:58:11,620 - INFO - Connection established!
2025-03-17 16:58:11,620 - INFO - Connected to database: AccessAuditDB on server: T915-TEST-DB
2025-03-17 16:58:11,620 - INFO - Starting PSO algorithm...
2025-03-17 16:58:11,620 - INFO - Number of particles: 5
2025-03-17 16:58:11,620 - INFO - Number of iterations: 5
2025-03-17 16:58:11,620 - INFO - Iteration 1 started.
2025-03-17 16:58:11,621 - INFO - Executing query: SELECT * FROM TotalQuantityByProduct
2025-03-17 16:58:11,741 - INFO - Executing query: SELECT * FROM MonthlySales
2025-03-17 16:58:11,887 - INFO - Executing query: SELECT * FROM MonthlySales
2025-03-17 16:58:12,016 - INFO - Executing query: SELECT * FROM TotalSalesByCustomer
2025-03-17 16:58:16,463 - INFO - Executing query: SELECT * FROM TotalQuantityByProduct
2025-03-17 16:58:17,029 - INFO - Executing query: SELECT * FROM MonthlySales
2025-03-17 16:58:17,531 - INFO - Executing query: SELECT * FROM TotalSalesByCustomer
2025-03-17 16:58:22,793 - INFO - Executing query: SELECT * FROM TotalQuantityByProduct
2025-03-17 16:58:22,916 - INFO - Executing query: SELECT * FROM MonthlySales
2025-03-17 16:58:23,083 - INFO - Executing query: SELECT * FROM TotalSalesByCustomer
2025-03-17 16:58:25,112 - INFO - Executing query: SELECT * FROM TotalQuantityByProduct
2025-03-17 16:58:25,197 - INFO - Executing query: SELECT * FROM MonthlySales
2025-03-17 16:58:25,306 - INFO - Iteration 1: Best Cost = 0.1292, Execution Time = 0.1292, CPU Cost = 0.1292
2025-03-17 16:58:25,306 - INFO - Iteration 2 started.
2025-03-17 16:58:25,306 - INFO - Executing query: SELECT * FROM MonthlySales
2025-03-17 16:58:25,406 - INFO - Executing query: SELECT * FROM TotalQuantityByProduct
2025-03-17 16:58:25,487 - INFO - Executing query: SELECT * FROM TotalSalesByCustomer
2025-03-17 16:58:27,227 - INFO - Executing query: SELECT * FROM MonthlySales
2025-03-17 16:58:27,378 - INFO - Iteration 2: Best Cost = 0.0813, Execution Time = 0.0813, CPU Cost = 0.0813
2025-03-17 16:58:27,378 - INFO - Iteration 3 started.
2025-03-17 16:58:27,378 - INFO - Executing query: SELECT * FROM TotalQuantityByProduct
2025-03-17 16:58:27,494 - INFO - Executing query: SELECT * FROM MonthlySales
2025-03-17 16:58:27,618 - INFO - Executing query: SELECT * FROM TotalSalesByCustomer
2025-03-17 16:58:28,910 - INFO - Executing query: SELECT * FROM MonthlySales
2025-03-17 16:58:28,998 - INFO - Executing query: SELECT * FROM TotalSalesByCustomer
2025-03-17 16:58:29,985 - INFO - Executing query: SELECT * FROM MonthlySales
2025-03-17 16:58:30,080 - INFO - Executing query: SELECT * FROM TotalSalesByCustomer
2025-03-17 16:58:32,420 - INFO - Executing query: SELECT * FROM TotalQuantityByProduct
2025-03-17 16:58:32,484 - INFO - Executing query: SELECT * FROM TotalSalesByCustomer
2025-03-17 16:58:35,111 - INFO - Executing query: SELECT * FROM TotalQuantityByProduct
2025-03-17 16:58:35,467 - INFO - Iteration 3: Best Cost = 0.0813, Execution Time = 0.0813, CPU Cost = 0.0813
2025-03-17 16:58:35,467 - INFO - Iteration 4 started.
2025-03-17 16:58:35,467 - INFO - Executing query: SELECT * FROM TotalSalesByCustomer
2025-03-17 16:58:40,846 - INFO - Executing query: SELECT * FROM TotalQuantityByProduct
2025-03-17 16:58:41,379 - INFO - Executing query: SELECT * FROM TotalSalesByCustomer
2025-03-17 16:58:43,526 - INFO - Executing query: SELECT * FROM TotalQuantityByProduct
2025-03-17 16:58:43,596 - INFO - Executing query: SELECT * FROM TotalSalesByCustomer
2025-03-17 16:58:45,083 - INFO - Executing query: SELECT * FROM TotalQuantityByProduct
2025-03-17 16:58:45,205 - INFO - Executing query: SELECT * FROM MonthlySales
2025-03-17 16:58:45,337 - INFO - Iteration 4: Best Cost = 0.0705, Execution Time = 0.0705, CPU Cost = 0.0705
2025-03-17 16:58:45,337 - INFO - Iteration 5 started.
2025-03-17 16:58:45,337 - INFO - Executing query: SELECT * FROM MonthlySales
2025-03-17 16:58:45,439 - INFO - Executing query: SELECT * FROM TotalSalesByCustomer
2025-03-17 16:58:46,835 - INFO - Executing query: SELECT * FROM TotalQuantityByProduct
2025-03-17 16:58:46,899 - INFO - Executing query: SELECT * FROM TotalQuantityByProduct
2025-03-17 16:58:46,958 - INFO - Executing query: SELECT * FROM MonthlySales
2025-03-17 16:58:47,045 - INFO - Executing query: SELECT * FROM TotalSalesByCustomer
2025-03-17 16:58:48,672 - INFO - Executing query: SELECT * FROM TotalQuantityByProduct
2025-03-17 16:58:48,739 - INFO - Executing query: SELECT * FROM TotalQuantityByProduct
2025-03-17 16:58:48,811 - INFO - Iteration 5: Best Cost = 0.0705, Execution Time = 0.0705, CPU Cost = 0.0705
2025-03-17 16:58:48,812 - INFO - Optimal Materialized Views:
2025-03-17 16:58:48,812 - INFO - - SELECT * FROM TotalQuantityByProduct
2025-03-17 16:58:48,812 - INFO - Best Execution Time: 0.0705
2025-03-17 16:58:48,812 - INFO - Best CPU Cost: 0.0705
2025-03-17 16:58:48,916 - INFO - Connection closed.

[Done] exited with code=0 in 38.476 seconds


\end{lstlisting} \vspace{.4cm}
  

  %\begin{lstlisting}[style=pythonstyle, label={lst:example} caption={Python Code Example}]

Connection established!
Particle position: [0, 1, 1], Cost: 0.1426
Particle position: [1, 0, 1], Cost: 1.0547
Particle position: [0, 0, 1], Cost: 0.0894
Particle position: [1, 1, 1], Cost: 1.0979
Particle position: [0, 0, 0], Cost: inf
Iteration 1: Best Cost = 0.0894
Particle position: [1, 0, 0], Cost: 0.9459
Particle position: [0, 1, 0], Cost: 0.0562
Particle position: [0, 0, 0], Cost: inf
Particle position: [1, 0, 0], Cost: 0.9560
Particle position: [0, 0, 1], Cost: 0.1090
Iteration 2: Best Cost = 0.0562
Particle position: [1, 1, 1], Cost: 1.0917
Particle position: [1, 0, 0], Cost: 0.9502
Particle position: [0, 1, 1], Cost: 0.1423
Particle position: [1, 0, 0], Cost: 0.9476
Particle position: [0, 1, 1], Cost: 0.1524
Iteration 3: Best Cost = 0.0562
Particle position: [1, 1, 1], Cost: 1.0997
Particle position: [1, 1, 0], Cost: 1.0002
Particle position: [0, 0, 0], Cost: inf
Particle position: [1, 1, 0], Cost: 1.0056
Particle position: [0, 0, 0], Cost: inf
Iteration 4: Best Cost = 0.0562
Particle position: [1, 1, 1], Cost: 1.0935
Particle position: [1, 0, 0], Cost: 0.9564
Particle position: [0, 1, 0], Cost: 0.0553
Particle position: [1, 0, 0], Cost: 0.9487
Particle position: [0, 1, 1], Cost: 0.1417
Iteration 5: Best Cost = 0.0553
Optimal Materialized Views: [0, 1, 0]

Query Optimization Comparison:
                       Metric  Average Query Time (s)
0  Without Materialized Views                     inf
1     With Materialized Views                0.668428
Connection closed.


\end{lstlisting} \vspace{.4cm}

  

\textbf{For the full code and further test cases, please refer to the Appendix~\ref{fullcode:Fullcode}.}


\subsection{Performance Testing and Data Analysis} After creating the materialized views on the database \texttt{"HealthInsuranceDB"}, performance metrics were collected using SQL query statistics. Query elapsed times and CPU usages are recorded using built-in database monitoring tools, such as Dynamic Management Views (DMVs) or with the help of SQL queries. \vspace{.4cm}

 
\definecolor{dkgreen}{rgb}{0,0.6,0}
\definecolor{gray}{rgb}{0.5,0.5,0.5}
\definecolor{mauve}{rgb}{0.58,0,0.82}
\lstset{language=SQL,
  basicstyle={\small\ttfamily},
  belowskip=3mm,
  breakatwhitespace=true,
  breaklines=true,
  classoffset=0,
  columns=flexible,
  commentstyle=\color{dkgreen},
  framexleftmargin=0.25em,
  frameshape={}{yy}{}{}, %To remove to vertical lines on left, set `frameshape={}{}{}{}`
  keywordstyle=\color{blue},
  numbers=none, %If you want line numbers, set `numbers=left`
  numberstyle=\tiny\color{gray},
  showstringspaces=false,
  stringstyle=\color{mauve},
  tabsize=3,
  xleftmargin =1em
}
         \begin{lstlisting}
SET STATISTICS TIME ON;
GO
-- TEST CASE 1: Total Claims By Patient
-- ===========================================================
PRINT CHAR(10) + '1. TOTAL CLAIMS BY PATIENT' + CHAR(10) + REPLICATE('-', 40);

PRINT 'DIRECT QUERY:';
SELECT 
    p.PatientID,
    p.FirstName,
    p.LastName,
    COUNT(c.ClaimID) AS TotalClaims
FROM Patients p
JOIN Claims c ON p.PatientID = c.PatientID
GROUP BY p.PatientID, p.FirstName, p.LastName;

PRINT CHAR(10) + 'MATERIALIZED VIEW:';
SELECT * FROM TotalClaimsByPatient;
GO
-- Disable statistics
SET STATISTICS TIME OFF;
GO
        \end{lstlisting}

 \textbf{The full script to analyse performance is provided in listing ~\ref{lst:Analysis}.}

\begin{enumerate}


    \item \textbf{ Query response time comparison:}\\
The tests were performed using \textit{MSSQL} on a system running \textit{Windows 10 Pro} with the following specifications:
\begin{itemize}
    \item \textbf{System Type}: 64-bit operating system, x64-based processor.
    \item \textbf{Processor}: 12th Gen Intel(R) Core(TM) i7-1255U, 1.70 GHz.
    \item \textbf{RAM}: 32 GB.
    \item \textbf{Server}: MSSQL.
\end{itemize}\vspace{.4cm}

After executing each query multiple times, both with and without materialized views, the \textit{response time}\footnote{Time taken to execute queries} and \textit{CPU usage}\footnote{Percentage of CPU utilization during query execution and refresh processes.} were noted in from the "Messages" tab in SSMS. The effectiveness of the materialized views was also evaluated using a comparison table and the percentage difference formula.

\begin{itemize}
\item\textbf{Differences between optimization methods:} The following \hyperref[comparison_table]{\textit{Table}~\ref*{comparison_table}} presents a performance comparison of elapsed time using two different approaches: \textit{Direct SELECT Aggregation}, and \textit{Materialized View}. The results show that the MV approach consistently outperforms the other method, with an average execution time of \textbf{69.2 ms}, compared to \textbf{106.6 ms} for direct \textit{SELECT} aggregation. This demonstrates that materialized views significantly reduce query execution time, providing a \textbf{35.08\%} improvement over direct aggregation. The consistent performance across multiple runs highlights the reliability and efficiency of materialized views in optimizing database queries.\vspace{.4cm}
 
 \begin{table}[h!]
\centering
\caption{Performance Comparison}
\renewcommand{\arraystretch}{1.2} % Adjust row height
\setlength{\tabcolsep}{8pt}       % Adjust column width
\resizebox{\textwidth}{!}{ % Resizes the table to fit the page width
\begin{tabular}{|c|>{\columncolor[HTML]{D9EAF1}}c|>{\columncolor[HTML]{D9EAF1}}c|>{\columncolor[HTML]{D9EAF1}}c|}
\hline
\rowcolor[HTML]{276B7A} 
\textbf{Run} & \textbf{\textcolor{white}{Direct SELECT Aggregation (ms)}} & \textbf{\textcolor{white}{Indexed View Query (ms)}} & \textbf{\textcolor{white}{Materialized View (ms)}} \\ \hline
Run 1        & 177                                                        & 15                                                & 5                                                \\ \hline
Run 2        & 40                                                         & 14                                                & 3                                                \\ \hline
Run 3        & 36                                                         & 16                                                & 3                                                \\ \hline
Run 4        & 40                                                         & 14                                                & 6                                                \\ \hline
Run 5        & 40                                                         & 15                                                & 7                                                \\ \hline
\rowcolor[HTML]{BFDDE5} 
\textbf{Average} & \textbf{66.6}                                           & \textbf{14.8}                                     & \textbf{5}                                       \\ \hline
\end{tabular}
} % End of \resizebox
\label{table:performance_comparison}
\end{table}\vspace{.4cm
 }

 \item\textbf{Analysis of percentage differences in query performance:}
 
For example, the percentage difference for Query 1 is calculated using the formula:

\begin{equation}
\text{Difference D in  (\%)} = \frac{W - M}{W} \times 100
\end{equation}

\noindent \textbf{Where:}
\begin{itemize}
    \item \( W \) is the initial value (Execution time without materialized views).
    \item \( M \) is the new value (Execution time with materialized views).
    \item \( D \) is the difference between \( W \) and \( M \) (i.e., \( D = W - M \)).
\end{itemize}

Substituting the values:

\[
\text{ D in (\%)} = \frac{2.35 - 0.45}{2.35} \times 100 \approx 80.85\%
\]

\begin{comment}\[
\text{Difference (\%)} = \frac{\text{Without MV} - \text{With MV}}{\text{Without MV}} \times 100 = \frac{2.35 - 0.45}{2.35} \times 100 \approx 80.85\%
\]
\end{comment}\vspace{.4cm}

The output indicates an \textit{39.47\% }improvement in performance. Here, \( W = 114 \) represents the initial execution time (without optimization), and \( M = 69 \) represents the improved execution time (with optimization). This significant reduction in execution time demonstrates the effectiveness of the optimization technique, as it reduces the query processing time by approximately \textit{39.47\%}, leading to faster and more efficient database operations.

\item \textbf{Impact of MV on CPU and elapsed time:} The \hyperref[tab:performance]{\textit{Table}~\ref*{tab:performance}} shows the execution metrics for three queries, highlighting significant improvements when using MVs. Particularly is the \textit{MonthlyClaimsByProvider} where CPU and elapsed times were reduced by factors 47x and 38x, respectively. The \textit{"TotalTreatmentCostByProvider"} showed modest benefits (1.1× and 1.6×), while the  \textit{"TotalClaimsByPatient"} demonstrated more notable gains, with execution times improved by approximately 6.3×. The results confirm that materialized views can substantially enhance query performance, especially for complex data aggregations and frequent report generation.

% Define custom colors
\definecolor{thesisblue}{RGB}{25,84,166}  % Darker, more professional blue
\definecolor{rowgray}{RGB}{248,248,248}   % Very light gray for subtle contrast
\begin{table}[H]
\centering
\caption{Performance Comparison of Views}
\label{tab:performance}
\renewcommand{\arraystretch}{1.5} % Increased row height
\setlength{\tabcolsep}{10pt} % Optimal column spacing
\begin{tabular}{|>{\RaggedRight}p{3.2cm}|rr|rr|rr|}
\hline
\rowcolor{thesisblue}
\multicolumn{1}{|>{\centering\color{white}}p{3.2cm}|}{\textbf{Test Case}} & 
\multicolumn{2}{>{\centering\color{white}}p{2.4cm}|}{\textbf{Direct (ms)}} & 
\multicolumn{2}{>{\centering\color{white}}p{2.4cm}|}{\textbf{MV (ms)}} & 
\multicolumn{2}{>{\centering\color{white}}p{2.8cm}|}{\textbf{Improvement Factor}} \\
\cline{2-7}
\rowcolor{thesisblue}
& \color{white}\textbf{CPU} & \color{white}\textbf{Elap.} & \color{white}\textbf{CPU} & \color{white}\textbf{Elap.} & \color{white}\textbf{CPU} & \color{white}\textbf{Elap.} \\
\hline
\rowcolor{rowgray}
TotalClaims\\ByPatient & 203 & 4,983 & 32 & 797 & 6.3$\times$ & 6.3$\times$ \\ \hline
TotalTreatment\\CostByProvider & 297 & 110 & 282 & 68 & 1.1$\times$ & 1.6$\times$ \\ \hline
\rowcolor{rowgray}
MonthlyClaims\\ByProvider & 47 & 38 & 1 & 1 & 47$\times$ & 38$\times$ \\ \hline
\end{tabular}

\vspace{6pt}
{\footnotesize 
\textbf{Note:} Elap. = Elapsed Time. Factors are calculated as Direct/MV time ratios. (X for multiplicative improvements). \\
Tests on SQL Server 2019 (SSMS v18.4).}
\end{table}
\clearpage

%
% Optional: Define custom colors
\definecolor{headerblue}{RGB}{200, 220, 255}
\definecolor{rowgray}{gray}{0.95}
\definecolor{lightgreen}{RGB}{220, 255, 220}
\definecolor{lightred}{RGB}{255, 230, 230}

\begin{table}[h!]
    \centering
    \caption{Performance Comparison}
    %\label{tab:performance}
    \rowcolors{2}{gray!10}{white} % Alternate row colors
    \begin{tabular}{lccc}
        \toprule
        \rowcolor{blue!10} % Header row color
        \textbf{Query} & \textbf{Without MV (s)} & \textbf{With MV (s)} & \textbf{Difference (\%)} \\
        \midrule
        Query 1 & 2.35 & 0.45 & \cellcolor{white!20}80.85 \\
        Query 2 & 3.78 & 0.62 & \cellcolor{gray!10}83.61 \\
        Query 3 & 4.21 & 0.87 & \cellcolor{white!20}79.34 \\
        \bottomrule
    \end{tabular}
\end{table}



%\item \textbf{Bar Chart Comparison:} As shown in the \hyperref[fig:execution-plan]{barchart}, the execution time demonstrates the optimization achieved by using materialized views. Direct select aggregation is the least efficient as it involves computing aggregation directly from the raw data during query execution, which is resource-intensive and time-consuming. On the other hand, a materialized view is the fastest one. Although the choice of method depends on the trade-off between query speed, storage requirements, and maintenance.



%\begin{figure}[H]
%\centering
%\includegraphics[width=0.8\textwidth]{Figure/Bar_chart.png} % Replace with your image file name
%\caption{Comparison of query execution time} % Caption for the screenshot
%\label{fig:execution-plan} % Label for referencing
%\end{figure}


\end{itemize} \vspace{.4cm}

\item \textbf{CPU performance analysis:}

The following \hyperref[fig:cpu_usage]{\textit{bar chart}} illustrates the average CPU usage for query execution before and after materialized views (MVs) are implemented. The x-axis is used to represent two conditions: \textit{Before} MVs (without materialized views) and \textit{After} MVs (with materialized views). The y-axis is used to represent the average CPU usage as a percentage. As it is evident from the graph, the CPU utilization average went down significantly from \textbf{85\%} (before MVs) to \textbf{30\%} (after MVs), showcasing the effectiveness of materialized views in reducing CPU burden in query processing. The reduction can be accounted for due to precomputation and caching of query output, which reduces real-time computation and subsequently CPU consumption.


% Optional: Define custom colors
\definecolor{headerblue}{RGB}{200, 220, 255}
\definecolor{rowgray}{gray}{0.95}
\definecolor{lightgreen}{RGB}{220, 255, 220}
\definecolor{lightred}{RGB}{255, 230, 230}

\begin{table}[h!]
    \centering
    \caption{Performance Comparison}
    %\label{tab:performance}
    \rowcolors{2}{gray!10}{white} % Alternate row colors
    \begin{tabular}{lccc}
        \toprule
        \rowcolor{blue!10} % Header row color
        \textbf{Query} & \textbf{Without MV (s)} & \textbf{With MV (s)} & \textbf{Difference (\%)} \\
        \midrule
        Query 1 & 2.35 & 0.45 & \cellcolor{white!20}80.85 \\
        Query 2 & 3.78 & 0.62 & \cellcolor{gray!10}83.61 \\
        Query 3 & 4.21 & 0.87 & \cellcolor{white!20}79.34 \\
        \bottomrule
    \end{tabular}
\end{table}





 \textbf{Data will be updated once the final implementation is done}.

