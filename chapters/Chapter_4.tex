%\chapter{Implementaion}
\section{Implementation}
\subsection{Experimental Setup}
\subsection{Database System Selection}
%\subsection{Test Environment Setup }



 \begin{center}
     \textbf{.................Test Implementation ... Running ...........}
 \end{center}

\subsection{Practical Implementation}
\subsubsection{Test Environment Setup }
\subsubsection{Understand/ Design Query Design} Analyzing the query workload to identify frequently used sub-queries, aggregation ( Queries with GROUP BY, SUM, AVG)s, or frequently joins between tables. 
\subsubsection{Implement the PSO algorithm:} It is important to prepare the input data to get the best outcome. The following factors should consider for the preparation. 

  \begin{itemize}
      \item \textbf{Candidate views}
      \item \textbf{Storage budget}
      \item \textbf{Query logs}
  \end{itemize}

  The following Python code demonstrates the implementation of the PSO algorithm for MV.

  \begin{lstlisting}[style=python, caption={Python Code Example}]
import numpy as np
import random

# Define parameters
NUM_PARTICLES = 30
NUM_ITERATIONS = 50
W = 0.5  # Inertia weight
C1 = 1.5  # Cognitive coefficient
C2 = 1.5  # Social coefficient
STORAGE_BUDGET = 100  # Maximum storage allowed in MB

# Sample input data: candidate materialized views
views = [
    {"id": 1, "response_time_savings": 50, "storage_cost": 20, "maintenance_cost": 10},
    {"id": 2, "response_time_savings": 70, "storage_cost": 30, "maintenance_cost": 20},
    {"id": 3, "response_time_savings": 40, "storage_cost": 15, "maintenance_cost": 5},
    {"id": 4, "response_time_savings": 60, "storage_cost": 25, "maintenance_cost": 15},
    # Add more views as needed
]
NUM_VIEWS = len(views)

# Initialize particles
particles = []
velocities = []
pbest_positions = []
pbest_scores = []
gbest_position = None
gbest_score = float('-inf')

# Initialize particles randomly
for _ in range(NUM_PARTICLES):
    particle = np.random.randint(0, 2, size=NUM_VIEWS)  # Binary vector (0 or 1)
    velocity = np.random.uniform(-1, 1, size=NUM_VIEWS)
    particles.append(particle)
    velocities.append(velocity)
    pbest_positions.append(particle)
    pbest_scores.append(float('-inf'))

# Define fitness function
def fitness_function(particle):
    total_savings = 0
    total_storage_cost = 0
    total_maintenance_cost = 0

    for i, selected in enumerate(particle):
        if selected == 1:
            total_savings += views[i]["response_time_savings"]
            total_storage_cost += views[i]["storage_cost"]
            total_maintenance_cost += views[i]["maintenance_cost"]

    # Penalize solutions exceeding storage budget
    if total_storage_cost > STORAGE_BUDGET:
        return float('-inf')  # Invalid solution

    return total_savings - (total_storage_cost + total_maintenance_cost)

# PSO loop
for iteration in range(NUM_ITERATIONS):
    for i, particle in enumerate(particles):
        # Calculate fitness
        fitness = fitness_function(particle)

        # Update personal best
        if fitness > pbest_scores[i]:
            pbest_scores[i] = fitness
            pbest_positions[i] = particle

        # Update global best
        if fitness > gbest_score:
            gbest_score = fitness
            gbest_position = particle

    # Update velocities and positions
    for i, particle in enumerate(particles):
        r1 = np.random.uniform(0, 1, size=NUM_VIEWS)
        r2 = np.random.uniform(0, 1, size=NUM_VIEWS)
        velocities[i] = (
            W * velocities[i]
            + C1 * r1 * (pbest_positions[i] - particle)
            + C2 * r2 * (gbest_position - particle)
        )
        # Update positions using sigmoid activation
        particles[i] = np.where(np.random.uniform(0, 1, size=NUM_VIEWS) < sigmoid(velocities[i]), 1, 0)

# Sigmoid function for position update
def sigmoid(x):
    return 1 / (1 + np.exp(-x))

# Output results
print("Optimal Materialized Views:", gbest_position)
print("Maximum Fitness:", gbest_score)


\end{lstlisting}
  
\subsubsection{Apply the Selected  Materialized Views in MSSQL:}

\begin{enumerate}
    \item \textbf{Identify optimal views:} After running the PSO algorithm, we have to select the views to create MV based on the output.
    \item \textbf{ Create MV on MSSQL :} Create MV in MSSQL using the following SQL commands.
   
\definecolor{dkgreen}{rgb}{0,0.6,0}
\definecolor{gray}{rgb}{0.5,0.5,0.5}
\definecolor{mauve}{rgb}{0.58,0,0.82}
\lstset{language=SQL,
  basicstyle={\small\ttfamily},
  belowskip=3mm,
  breakatwhitespace=true,
  breaklines=true,
  classoffset=0,
  columns=flexible,
  commentstyle=\color{dkgreen},
  framexleftmargin=0.25em,
  frameshape={}{yy}{}{}, %To remove to vertical lines on left, set `frameshape={}{}{}{}`
  keywordstyle=\color{blue},
  numbers=none, %If you want line numbers, set `numbers=left`
  numberstyle=\tiny\color{gray},
  showstringspaces=false,
  stringstyle=\color{mauve},
  tabsize=3,
  xleftmargin =1em
}
         \begin{lstlisting}

CREATE VIEW SelectedView1
WITH SCHEMABINDING
AS
SELECT ProductID, SUM(SalesAmount) AS TotalSales
FROM Sales
GROUP BY ProductID;

CREATE UNIQUE CLUSTERED INDEX idx_SelectedView1
ON SelectedView1 (ProductID);


        \end{lstlisting}
    
\end{enumerate}


\subsubsection{Materialized View Management \& Selection Approach}
\subsubsection{View Maintenance and Refresh Strategies:} Manual or automatic refresh strategies can be set up according to the requirements query to create a scheduled job:

% Define colors
\definecolor{codegreen}{rgb}{0,0.6,0}  % ✅ Green for comments
\definecolor{codegray}{rgb}{0.5,0.5,0.5}  % ✅ Gray for numbers
\definecolor{codepurple}{rgb}{0.58,0,0.82}  % ✅ Purple for strings
\definecolor{backcolour}{rgb}{0.95,0.95,0.92}  % ✅ Light gray background
\definecolor{bordercolor}{rgb}{0.7,0.7,0.7}  % ✅ Left border color (gray)
\definecolor{codeblue}{rgb}{0,0,0.8}  % ✅ Blue for SQL keywords

% ✅ Define SQL language with correct comment handling
\lstdefinelanguage{MySQL}{
    keywords={SELECT, FROM, WHERE, JOIN, ON, INNER, OUTER, LEFT, RIGHT, FULL, GROUP, BY, ORDER, ASC, DESC, AS, COUNT, SUM, AVG, MAX, MIN, DISTINCT, INSERT, INTO, VALUES, UPDATE, SET, DELETE, CREATE, TABLE, PRIMARY, FOREIGN, KEY, DEFAULT, NULL, NOT, CHECK, CONSTRAINT, INDEX, VIEW, MATERIALIZED, PROCEDURE, FUNCTION, TRIGGER, DATABASE, ALTER, DROP, EXEC, IF, EXISTS, UNION, ALL, CASE, WHEN, THEN, ELSE, END, CAST, CONVERT, LIKE, IN, BETWEEN, AND, OR, HAVING, LIMIT, OFFSET},
    sensitive=false,
    morestring=[b]',  % ✅ Strings in single quotes
    morestring=[b]",  % ✅ Strings in double quotes
    morecomment=[l][\color{codegreen}]{--}  % ✅ Ensures full line comment in green
}

\lstdefinestyle{sqlstyle}{
    backgroundcolor=\color{backcolour},   
    commentstyle=\color{codegreen},  % ✅ Comments in green
    keywordstyle=\bfseries\color{codeblue},  % ✅ SQL Keywords in Blue & Bold
    numberstyle=\scriptsize\color{codegray},  % ✅ Row numbers in gray
    stringstyle=\color{codepurple},  % ✅ Strings in purple
    basicstyle=\ttfamily\footnotesize,
    breaklines=true,
    captionpos=b,
    numbers=left,      % ✅ Enables row numbers on the left
    stepnumber=1,      % ✅ Row numbers increment by 1
    firstnumber=1,     % ✅ Starts numbering at 1
    numbersep=8pt,     % ✅ Increases space between numbers and SQL code
    xleftmargin=3em,   % ✅ Ensures space inside the left border
    frame=single,      % ✅ Keeps a single border (left-aligned)
    framesep=5pt,      % ✅ Ensures space inside the frame
    rulesepcolor=\color{bordercolor},  % ✅ Matches row numbers with left border
    rulecolor=\color{bordercolor},  % ✅ Sets left border color
    language=MySQL  % ✅ Uses SQL keyword highlighting
}
         \begin{lstlisting}[style=sqlstyle, caption={Maintenance and Refresh Strategies}, label=lst:Maintenance_and_Refresh_Strategies]

     ---1. Incremental refresh 
     
         CREATE TRIGGER trg_UpdateClaims
ON Claims
AFTER INSERT, UPDATE, DELETE
AS
BEGIN
    -- Refresh materialized view logic here
    EXEC sp_refreshview 'TotalClaimsByPatient';
END;
GO

--This trigger ensures that the TotalClaimsByPatient materialized view stays up to date. It automatically refreshes the view whenever records in the Claims table are inserted, updated, or deleted by calling sp_refreshview.



     ---2. Example of creating a job to refresh a materialized view automatically every hour

USE msdb;
GO

-- Create the job
EXEC dbo.sp_add_job
    @job_name = N'RefreshMaterializedViewJob',
    @enabled = 1,
    @description = N'Job to refresh the materialized view every hour.';

-- Add a job step
EXEC sp_add_jobstep
    @job_name = N'RefreshMaterializedViewJob',
    @step_name = N'RefreshViews',
    @subsystem = N'TSQL',
    @command = 'EXEC sp_refreshview ''TotalClaimsByPatient'';
               EXEC sp_refreshview ''TotalTreatmentCostByProvider'';
               EXEC sp_refreshview ''MonthlyClaimsByProvider'';',
    @retry_attempts = 3,
    @retry_interval = 5;

-- Create a schedule for the job
EXEC sp_add_schedule
    @schedule_name = N'HourlySchedule',
    @freq_type = 4, -- Daily
    @freq_interval = 1, -- Every day
    @freq_subday_type = 8, -- Hourly
    @freq_subday_interval = 1, -- Every 1 hour
    @active_start_time = 000000; -- Start time (midnight)

-- Attach the schedule to the job
EXEC sp_attach_schedule
    @job_name = N'RefreshMaterializedViewJob',
    @schedule_name = N'HourlySchedule';

-- Assign the job to the SQL Server Agent service
EXEC dbo.Sp_add_jobserver
     @job_name = N'RefreshMaterializedViewJob';

go 



      ---3. Manual Refresh
EXEC sp_refreshview 'TotalClaimsByPatient';
EXEC sp_refreshview 'TotalTreatmentCostByProvider';
EXEC sp_refreshview 'MonthlyClaimsByProvider';

--The EXEC sp_refreshview statements refresh the metadata for each specified view. This ensures that the views are up to date with any changes made to the underlying tables (like added columns or altered data types), helping to prevent potential schema-related errors during future queries or operations.
        \end{lstlisting}

\subsubsection{Performance Testing and Data Analysis}

\textbf{......\textbf{Work on Progress}..........}       