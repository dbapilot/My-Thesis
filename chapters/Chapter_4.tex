%\chapter{Implementaion}
\section{Implementation}
This section describes a step-by-step process for effectively deploying materialized views, based on the earlier theoretical foundations, to reduce query execution time, minimize computational overhead, and enhance overall performance.

\subsection{Used Software and Tools}
\subsubsection{SQL Server Management Studio:}SQL Server Management Studio is actually an integrated environment that can maintain the SQL server infrastructure. It is used to manage the schema, tables, and materialized views of the database. In addition, it monitors the performance of the query using execution plans and statistics, and it helps to debug SQL queries and optimize their execution.

\subsubsection{Visual Studio Code:} Microsoft created this open-source integrated development environment (IDE) for web browsers, Linux, macOS, and Windows. It is used to write Python scripts for automation (e.g., measuring query performance), integrate MSSQL queries with Python using extensions, and debug Python and SQL scripts.

\subsubsection{Overleaf:} Overleaf is an open-source online, real-time collaborative LaTeX editor that simplifies the process of creating, editing, and collaborating on LaTeX documents.

\subsubsection{Microsoft SQL Server:} Microsoft SQL server is a relational database that provides a wide range of features for storing, processing, and securing data.


\subsubsection{Python:} Python is a high-level interpreted programming language known for its simplicity and readability. It was created by Guido van Rossum and released in 1991 \cite{martin2023stam,wijanarko2020prediksi} . It has become one of the most popular programming languages worldwide. The structure of the language and its object-oriented approach help programmers to write logical and clear code for small and large projects. Python libraries (packages) effectively simplify many important processes such as analysing and visualizing data, retrieving unstructured data from the web, image processing, building machine learning models, and textual information \cite{Samira_Gholizadeh2022}. Here, it has been used to implement the PSO algorithm to optimize query performance and execute and analyze the database programmatically.

\subsubsection{pyodbc:} Pyodbc is an open-source Python module that makes accessing ODBC databases simple. It implements the DB API 2.0 specification but is packed with even more Pythonic convenience.

\subsubsection{pandas:} The pandas constitute an open-source data manipulation and analysis tool that is fast, powerful, flexible, and easy to use. It is built entirely on the Python programming language. Pandas organizes and processes query performance data for further analysis. It is built entirely on the basis of the Python programming language. Panda Organizes and processes query performance data for further analysis.

\subsubsection{matplotlib:} Matplotlib is a comprehensive library for creating static, animated, and interactive visualizations in Python. Matplotlib makes easy things easy and hard things possible \cite{matplotlib}. It has been used to visualize performance improvements in different charts and tables. 
\subsubsection{Git:} Git is a version control system that is used for tracking the changes made in the files and for enabling a collaborative software world. Flexsibility, speed, and enablement of almost any applied workflow contribute to the great popularity of Git. It is more like a standard for version control - about 90 per cent of developers' primary system-in-use around the world fall into this category. This enables one to have version control over changes in Python and latex scripts and even multiple collaboration by users.
\subsection{Experimental Setup}
\subsection{Database System Selection}
%\subsection{Test Environment Setup }



 \begin{center}
     \textbf{.................Test Implementation ... Running ...........}
 \end{center}

\subsection{Practical Implementation}
\subsubsection{Test Environment Setup:}
\subsubsection{Identify Complex Queries} Analyzing the query workload to identify frequently used sub-queries, aggregation (Queries with GROUP BY, SUM, AVG)s, or frequently joins between tables. This is a very important part of identifying the frequent queries that need to be optimized. SQL server management studio can be used to analyze execution logs and identify the frequently executed and resource-intensive queries. For example, the following query to identify long-running queries:

% Define colors
\definecolor{codegreen}{rgb}{0,0.6,0}  % ✅ Green for comments
\definecolor{codegray}{rgb}{0.5,0.5,0.5}  % ✅ Gray for numbers
\definecolor{codepurple}{rgb}{0.58,0,0.82}  % ✅ Purple for strings
\definecolor{backcolour}{rgb}{0.95,0.95,0.92}  % ✅ Light gray background
\definecolor{bordercolor}{rgb}{0.7,0.7,0.7}  % ✅ Left border color (gray)
\definecolor{codeblue}{rgb}{0,0,0.8}  % ✅ Blue for SQL keywords

% ✅ Define SQL language with correct comment handling
\lstdefinelanguage{MySQL}{
    keywords={SELECT, FROM, WHERE, JOIN, ON, INNER, OUTER, LEFT, RIGHT, FULL, GROUP, BY, ORDER, ASC, DESC, AS, COUNT, SUM, AVG, MAX, MIN, DISTINCT, INSERT, INTO, VALUES, UPDATE, SET, DELETE, CREATE, TABLE, PRIMARY, FOREIGN, KEY, DEFAULT, NULL, NOT, CHECK, CONSTRAINT, INDEX, VIEW, MATERIALIZED, PROCEDURE, FUNCTION, TRIGGER, DATABASE, ALTER, DROP, EXEC, IF, EXISTS, UNION, ALL, CASE, WHEN, THEN, ELSE, END, CAST, CONVERT, LIKE, IN, BETWEEN, AND, OR, HAVING, LIMIT, OFFSET},
    sensitive=false,
    morestring=[b]',  % ✅ Strings in single quotes
    morestring=[b]",  % ✅ Strings in double quotes
    morecomment=[l][\color{codegreen}]{--}  % ✅ Ensures full line comment in green
}

\lstdefinestyle{sqlstyle}{
    backgroundcolor=\color{backcolour},   
    commentstyle=\color{codegreen},  % ✅ Comments in green
    keywordstyle=\bfseries\color{codeblue},  % ✅ SQL Keywords in Blue & Bold
    numberstyle=\scriptsize\color{codegray},  % ✅ Row numbers in gray
    stringstyle=\color{codepurple},  % ✅ Strings in purple
    basicstyle=\ttfamily\footnotesize,
    breaklines=true,
    captionpos=b,
    numbers=left,      % ✅ Enables row numbers on the left
    stepnumber=1,      % ✅ Row numbers increment by 1
    firstnumber=1,     % ✅ Starts numbering at 1
    numbersep=8pt,     % ✅ Increases space between numbers and SQL code
    xleftmargin=3em,   % ✅ Ensures space inside the left border
    frame=single,      % ✅ Keeps a single border (left-aligned)
    framesep=5pt,      % ✅ Ensures space inside the frame
    rulesepcolor=\color{bordercolor},  % ✅ Matches row numbers with left border
    rulecolor=\color{bordercolor},  % ✅ Sets left border color
    language=MySQL  % ✅ Uses SQL keyword highlighting
}


\begin{lstlisting}[style=sqlstyle, caption={SQL Query to  Identify Complex Queries},label=lst:IdentifyComplexQueries]
SELECT TOP 10 total_elapsed_time / 1000 AS ExecutionTime_ms,
              execution_count,
              query_hash,
              st.text                   AS sql_text
FROM   sys.dm_exec_query_stats
       CROSS apply sys.Dm_exec_sql_text(sql_handle) st
ORDER  BY total_elapsed_time DESC; 

\end{lstlisting}

\subsubsection{ Create Candidate Materialized Views:} It is important to prepare the input data to get the best outcome. The following factors should be considered for the preparation. 

  \begin{itemize}
      \item \textbf{Create views}
      \item \textbf{Storage budget}
      \item \textbf{Query logs}
  \end{itemize}

  % Define colors
\definecolor{codegreen}{rgb}{0,0.6,0}  % ✅ Green for comments
\definecolor{codegray}{rgb}{0.5,0.5,0.5}  % ✅ Gray for numbers
\definecolor{codepurple}{rgb}{0.58,0,0.82}  % ✅ Purple for strings
\definecolor{backcolour}{rgb}{0.95,0.95,0.92}  % ✅ Light gray background
\definecolor{bordercolor}{rgb}{0.7,0.7,0.7}  % ✅ Left border color (gray)
\definecolor{codeblue}{rgb}{0,0,0.8}  % ✅ Blue for SQL keywords

\definecolor{commentcolor}{RGB}{0, 128, 0}  % Green for comments
\definecolor{titlecolor}{RGB}{0, 0, 255}    % Blue for titles
% ✅ Define SQL language with correct comment handling
\lstdefinelanguage{MySQL}{
    keywords={SELECT, FROM, WHERE, JOIN, ON, INNER, OUTER, LEFT, RIGHT, FULL, GROUP, BY, ORDER, ASC, DESC, AS, COUNT, SUM, AVG, MAX, MIN, DISTINCT, INSERT, INTO, VALUES, UPDATE, SET, DELETE, CREATE, TABLE, PRIMARY, FOREIGN, KEY, DEFAULT, NULL, NOT, CHECK, CONSTRAINT, INDEX, VIEW, MATERIALIZED, PROCEDURE, FUNCTION, TRIGGER, DATABASE, ALTER, DROP, EXEC, IF, EXISTS, UNION, ALL, CASE, WHEN, THEN, ELSE, END, CAST, CONVERT, LIKE, IN, BETWEEN, AND, OR, HAVING, LIMIT, OFFSET},
    sensitive=false,
    morestring=[b]',  % ✅ Strings in single quotes
    morestring=[b]",  % ✅ Strings in double quotes
    morecomment=[l][\color{codegreen}]{--}  % ✅ Ensures full line comment in green
}

\lstdefinestyle{sqlstyle}{
    backgroundcolor=\color{backcolour},   
    commentstyle=\color{codegreen},  % ✅ Comments in green
    keywordstyle=\bfseries\color{codeblue},  % ✅ SQL Keywords in Blue & Bold
    numberstyle=\scriptsize\color{codegray},  % ✅ Row numbers in gray
    stringstyle=\color{codepurple},  % ✅ Strings in purple
    basicstyle=\ttfamily\footnotesize,
    breaklines=true,
    captionpos=b,
    title=\color{titlecolor}\textbf{Table Creation Script}, % Title in blue
    numbers=left,      % ✅ Enables row numbers on the left
    stepnumber=1,      % ✅ Row numbers increment by 1
    firstnumber=1,     % ✅ Starts numbering at 1
    numbersep=8pt,     % ✅ Increases space between numbers and SQL code
    xleftmargin=3em,   % ✅ Ensures space inside the left border
    frame=single,      % ✅ Keeps a single border (left-aligned)
    framesep=5pt,      % ✅ Ensures space inside the frame
    rulesepcolor=\color{bordercolor},  % ✅ Matches row numbers with left border
    rulecolor=\color{bordercolor},  % ✅ Sets left border color
    language=MySQL  % ✅ Uses SQL keyword highlighting
}
  
\begin{lstlisting}[style=sqlstyle, caption={Materialized view creation}, label=lst:MV_creation]
--Materialized View for Query 1: Total Claims By Patient
CREATE VIEW TotalClaimsByPatient
WITH SCHEMABINDING
AS
SELECT PatientID, COUNT_BIG(*) AS TotalClaims
FROM dbo.Claims
GROUP BY PatientID;
GO

CREATE UNIQUE CLUSTERED INDEX IX_TotalClaimsByPatient
ON TotalClaimsByPatient (PatientID);
GO

-- This view calculates the total number of claims submitted by each patient. It groups the data by PatientID and uses COUNT_BIG(*) to count all related claims.



--Materialized View for Query 2: TotalTreatmentCostByProvider


CREATE VIEW TotalTreatmentCostByProvider
WITH SCHEMABINDING
AS
SELECT c.InsuranceProviderID, SUM(t.Cost) AS TotalCost
FROM dbo.Treatments t
JOIN dbo.Claims c ON t.ClaimID = c.ClaimID
GROUP BY c.InsuranceProviderID;
GO

CREATE UNIQUE CLUSTERED INDEX IX_TotalTreatmentCostByProvider
ON TotalTreatmentCostByProvider (InsuranceProviderID);
GO

-- This view computes the total treatment cost for each insurance provider. It joins the Treatments and Claims tables and sums up the Cost field, grouped by InsuranceProviderID.

--Materialized View for Query 3: MonthlyClaimsByProvider

CREATE VIEW MonthlyClaimsByProvider
WITH SCHEMABINDING
AS
SELECT 
    c.InsuranceProviderID, 
    YEAR(c.ClaimDate) AS ClaimYear, 
    MONTH(c.ClaimDate) AS ClaimMonth, 
    COUNT_BIG(*) AS TotalClaims
FROM dbo.Claims c
GROUP BY c.InsuranceProviderID, YEAR(c.ClaimDate), MONTH(c.ClaimDate);
GO

CREATE UNIQUE CLUSTERED INDEX IX_MonthlyClaimsByProvider
ON MonthlyClaimsByProvider (InsuranceProviderID, ClaimYear, ClaimMonth);
GO

-- This view summarizes the number of claims submitted per insurance provider for each month. It extracts the year and month from ClaimDate and counts claims grouped by provider, year, and month.
\end{lstlisting}

  \input{ta}
  
\subsubsection{Implement the PSO algorithm:}
 The following Python code illustrates the implementation of the PSO algorithm for multivariate analysis. It evaluates various combinations of these parameters and selects the one that yields the best query response time.\vspace{.4cm}

\textbf{Example 1}

  \begin{lstlisting}[style=pythonstyle, label={lst:example} caption={Python Code Example}]

import pyodbc # A python  library for interacting with ODBC databases like MSSQL that helps to manage database connection
import random #It helps to generate random numbers and choice used to initialize particle positions and velocities
import time

# Connection parameters
server = 'T915-TEST-DB'  # server name that used to get data 
database = 'AccessAuditDB'  # Database name
driver_name = 'ODBC Driver 17 for SQL Server'  # ODBC  driver from pyodbc.drivers()
# Uncomment and add these if using SQL Server Authentication
# username = 'm.islam'
# password = 'your_password'

try:
    # Establish connection
    conn = pyodbc.connect(
        f'DRIVER={{{driver_name}}};'
        f'SERVER={server};'
        f'DATABASE={database};'
        'Trusted_Connection=yes;'  # Indicates that Windows Authentication used for Authentication
        # Uncomment these lines for SQL Authentication
        # f'UID={username};'
        # f'PWD={password};'
    )
    cursor = conn.cursor()
    print("Connection established!")

    #  A list of Queries corresponding to materialized views
    queries = [
        "SELECT * FROM TotalSalesByCustomer;",
        "SELECT * FROM TotalQuantityByProduct;",
        "SELECT * FROM MonthlySales;"
    ]

    # Cost function: Measure total query execution time for a set of materialized views 
    def cost_function(selected_views):
        total_time = 0
        if not any(selected_views):  # No views selected
            return float('inf')  # High cost for no selection
        for i, view in enumerate(selected_views):
            if view == 1:  # If view is selected
                start_time = time.time()
                cursor.execute(queries[i]) # Executes the SQL query for the selected view
                cursor.fetchall()
                total_time += time.time() - start_time # Total query execution time 
        return total_time

    # PSO parameters
    num_particles = 5 # Number of particles in the swarm 
    num_iterations = 10 #The number of iterations the algorithm will run
    num_queries = len(queries) #The number of queries (materialized views) to optimize.
    W = 0.5  # Inertia weight Controls the impact of the previous velocity on the current velocity.

    c1, c2 = 1.5, 1.5 # Encourages particles to move toward the personal/global best position.

    # Initialize particles
    particles = [{'position': [random.choice([0, 1]) for _ in range(num_queries)], #Represents a particle's selected views (1 = selected, 0 = not selected).
                  'velocity': [random.uniform(-1, 1) for _ in range(num_queries)], #Represents the rate of change for each view selection.
                  'best_position': None,
                  'best_cost': float('inf')} for _ in range(num_particles)] #The lowest cost (execution time) encountered by the particle.

    global_best_position = None  
    global_best_cost = float('inf')

    # PSO algorithm
    for iteration in range(num_iterations):
        for particle in particles:
            # Evaluate cost
            cost = cost_function(particle['position'])
            print(f"Particle position: {particle['position']}, Cost: {cost:.4f}")

            # Update personal best
            if cost < particle['best_cost']:
                particle['best_position'] = particle['position'][:]
                particle['best_cost'] = cost

            # Update global best
            if cost < global_best_cost:
                global_best_position = particle['position'][:]
                global_best_cost = cost

            # Update velocity and position using PSO formula 
            for i in range(num_queries):
                r1, r2 = random.random(), random.random()
                particle['velocity'][i] = (W * particle['velocity'][i] +
                                           c1 * r1 * (particle['best_position'][i] - particle['position'][i]) +
                                           c2 * r2 * (global_best_position[i] - particle['position'][i]))
                particle['position'][i] = 1 if random.random() < abs(particle['velocity'][i]) else 0

        print(f"Iteration {iteration + 1}: Best Cost = {global_best_cost:.4f}")

    print("Optimal Materialized Views:", global_best_position)

except pyodbc.Error as e:
    print("Error connecting to SQL Server:", e) # Catches and displays database connection errors.


finally:
    if 'conn' in locals() and conn:
        conn.close()
        print("Connection closed.")  #Ensures the database connection is closed after the script execution.



\end{lstlisting}\vspace{.4cm}

\textbf{Example 2}
  \begin{lstlisting}[style=python, caption={Python Code Example}]
import numpy as np
import random

# Define parameters
NUM_PARTICLES = 30
NUM_ITERATIONS = 50
W = 0.5  # Inertia weight
C1 = 1.5  # Cognitive coefficient
C2 = 1.5  # Social coefficient
STORAGE_BUDGET = 100  # Maximum storage allowed in MB

# Sample input data: candidate materialized views
views = [
    {"id": 1, "response_time_savings": 50, "storage_cost": 20, "maintenance_cost": 10},
    {"id": 2, "response_time_savings": 70, "storage_cost": 30, "maintenance_cost": 20},
    {"id": 3, "response_time_savings": 40, "storage_cost": 15, "maintenance_cost": 5},
    {"id": 4, "response_time_savings": 60, "storage_cost": 25, "maintenance_cost": 15},
    # Add more views as needed
]
NUM_VIEWS = len(views)

# Initialize particles
particles = []
velocities = []
pbest_positions = []
pbest_scores = []
gbest_position = None
gbest_score = float('-inf')

# Initialize particles randomly
for _ in range(NUM_PARTICLES):
    particle = np.random.randint(0, 2, size=NUM_VIEWS)  # Binary vector (0 or 1)
    velocity = np.random.uniform(-1, 1, size=NUM_VIEWS)
    particles.append(particle)
    velocities.append(velocity)
    pbest_positions.append(particle)
    pbest_scores.append(float('-inf'))

# Define fitness function
def fitness_function(particle):
    total_savings = 0
    total_storage_cost = 0
    total_maintenance_cost = 0

    for i, selected in enumerate(particle):
        if selected == 1:
            total_savings += views[i]["response_time_savings"]
            total_storage_cost += views[i]["storage_cost"]
            total_maintenance_cost += views[i]["maintenance_cost"]

    # Penalize solutions exceeding storage budget
    if total_storage_cost > STORAGE_BUDGET:
        return float('-inf')  # Invalid solution

    return total_savings - (total_storage_cost + total_maintenance_cost)

# PSO loop
for iteration in range(NUM_ITERATIONS):
    for i, particle in enumerate(particles):
        # Calculate fitness
        fitness = fitness_function(particle)

        # Update personal best
        if fitness > pbest_scores[i]:
            pbest_scores[i] = fitness
            pbest_positions[i] = particle

        # Update global best
        if fitness > gbest_score:
            gbest_score = fitness
            gbest_position = particle

    # Update velocities and positions
    for i, particle in enumerate(particles):
        r1 = np.random.uniform(0, 1, size=NUM_VIEWS)
        r2 = np.random.uniform(0, 1, size=NUM_VIEWS)
        velocities[i] = (
            W * velocities[i]
            + C1 * r1 * (pbest_positions[i] - particle)
            + C2 * r2 * (gbest_position - particle)
        )
        # Update positions using sigmoid activation
        particles[i] = np.where(np.random.uniform(0, 1, size=NUM_VIEWS) < sigmoid(velocities[i]), 1, 0)

# Sigmoid function for position update
def sigmoid(x):
    return 1 / (1 + np.exp(-x))

# Output results
print("Optimal Materialized Views:", gbest_position)
print("Maximum Fitness:", gbest_score)


\end{lstlisting}\vspace{.4cm}
  
\subsubsection{Apply the Selected  Materialized Views in MSSQL:}

\begin{enumerate}
    \item \textbf{Identify optimal views:} After running the PSO algorithm, we have to select the views to create MV based on the output.
    \item \textbf{ Create MV on MSSQL :} Create MV in MSSQL using the following SQL commands.
    
   
\definecolor{dkgreen}{rgb}{0,0.6,0}
\definecolor{gray}{rgb}{0.5,0.5,0.5}
\definecolor{mauve}{rgb}{0.58,0,0.82}
\lstset{language=SQL,
  basicstyle={\small\ttfamily},
  belowskip=3mm,
  breakatwhitespace=true,
  breaklines=true,
  classoffset=0,
  columns=flexible,
  commentstyle=\color{dkgreen},
  framexleftmargin=0.25em,
  frameshape={}{yy}{}{}, %To remove to vertical lines on left, set `frameshape={}{}{}{}`
  keywordstyle=\color{blue},
  numbers=none, %If you want line numbers, set `numbers=left`
  numberstyle=\tiny\color{gray},
  showstringspaces=false,
  stringstyle=\color{mauve},
  tabsize=3,
  xleftmargin =1em
}
         \begin{lstlisting}

CREATE VIEW SelectedView1
WITH SCHEMABINDING
AS
SELECT ProductID, SUM(SalesAmount) AS TotalSales
FROM Sales
GROUP BY ProductID;

CREATE UNIQUE CLUSTERED INDEX idx_SelectedView1
ON SelectedView1 (ProductID);


        \end{lstlisting}
    
\end{enumerate}


\subsubsection{Materialized View Management \& Selection Approach}
\subsubsection{View Maintenance and Refresh Strategies:} Manual or automatic refresh strategies can be set up according to the requirements query to create a scheduled job:

% Define colors
\definecolor{codegreen}{rgb}{0,0.6,0}  % ✅ Green for comments
\definecolor{codegray}{rgb}{0.5,0.5,0.5}  % ✅ Gray for numbers
\definecolor{codepurple}{rgb}{0.58,0,0.82}  % ✅ Purple for strings
\definecolor{backcolour}{rgb}{0.95,0.95,0.92}  % ✅ Light gray background
\definecolor{bordercolor}{rgb}{0.7,0.7,0.7}  % ✅ Left border color (gray)
\definecolor{codeblue}{rgb}{0,0,0.8}  % ✅ Blue for SQL keywords

% ✅ Define SQL language with correct comment handling
\lstdefinelanguage{MySQL}{
    keywords={SELECT, FROM, WHERE, JOIN, ON, INNER, OUTER, LEFT, RIGHT, FULL, GROUP, BY, ORDER, ASC, DESC, AS, COUNT, SUM, AVG, MAX, MIN, DISTINCT, INSERT, INTO, VALUES, UPDATE, SET, DELETE, CREATE, TABLE, PRIMARY, FOREIGN, KEY, DEFAULT, NULL, NOT, CHECK, CONSTRAINT, INDEX, VIEW, MATERIALIZED, PROCEDURE, FUNCTION, TRIGGER, DATABASE, ALTER, DROP, EXEC, IF, EXISTS, UNION, ALL, CASE, WHEN, THEN, ELSE, END, CAST, CONVERT, LIKE, IN, BETWEEN, AND, OR, HAVING, LIMIT, OFFSET},
    sensitive=false,
    morestring=[b]',  % ✅ Strings in single quotes
    morestring=[b]",  % ✅ Strings in double quotes
    morecomment=[l][\color{codegreen}]{--}  % ✅ Ensures full line comment in green
}

\lstdefinestyle{sqlstyle}{
    backgroundcolor=\color{backcolour},   
    commentstyle=\color{codegreen},  % ✅ Comments in green
    keywordstyle=\bfseries\color{codeblue},  % ✅ SQL Keywords in Blue & Bold
    numberstyle=\scriptsize\color{codegray},  % ✅ Row numbers in gray
    stringstyle=\color{codepurple},  % ✅ Strings in purple
    basicstyle=\ttfamily\footnotesize,
    breaklines=true,
    captionpos=b,
    numbers=left,      % ✅ Enables row numbers on the left
    stepnumber=1,      % ✅ Row numbers increment by 1
    firstnumber=1,     % ✅ Starts numbering at 1
    numbersep=8pt,     % ✅ Increases space between numbers and SQL code
    xleftmargin=3em,   % ✅ Ensures space inside the left border
    frame=single,      % ✅ Keeps a single border (left-aligned)
    framesep=5pt,      % ✅ Ensures space inside the frame
    rulesepcolor=\color{bordercolor},  % ✅ Matches row numbers with left border
    rulecolor=\color{bordercolor},  % ✅ Sets left border color
    language=MySQL  % ✅ Uses SQL keyword highlighting
}
         \begin{lstlisting}[style=sqlstyle, caption={Maintenance and Refresh Strategies}, label=lst:Maintenance_and_Refresh_Strategies]

     ---1. Incremental refresh 
     
         CREATE TRIGGER trg_UpdateClaims
ON Claims
AFTER INSERT, UPDATE, DELETE
AS
BEGIN
    -- Refresh materialized view logic here
    EXEC sp_refreshview 'TotalClaimsByPatient';
END;
GO

--This trigger ensures that the TotalClaimsByPatient materialized view stays up to date. It automatically refreshes the view whenever records in the Claims table are inserted, updated, or deleted by calling sp_refreshview.



     ---2. Example of creating a job to refresh a materialized view automatically every hour

USE msdb;
GO

-- Create the job
EXEC dbo.sp_add_job
    @job_name = N'RefreshMaterializedViewJob',
    @enabled = 1,
    @description = N'Job to refresh the materialized view every hour.';

-- Add a job step
EXEC sp_add_jobstep
    @job_name = N'RefreshMaterializedViewJob',
    @step_name = N'RefreshViews',
    @subsystem = N'TSQL',
    @command = 'EXEC sp_refreshview ''TotalClaimsByPatient'';
               EXEC sp_refreshview ''TotalTreatmentCostByProvider'';
               EXEC sp_refreshview ''MonthlyClaimsByProvider'';',
    @retry_attempts = 3,
    @retry_interval = 5;

-- Create a schedule for the job
EXEC sp_add_schedule
    @schedule_name = N'HourlySchedule',
    @freq_type = 4, -- Daily
    @freq_interval = 1, -- Every day
    @freq_subday_type = 8, -- Hourly
    @freq_subday_interval = 1, -- Every 1 hour
    @active_start_time = 000000; -- Start time (midnight)

-- Attach the schedule to the job
EXEC sp_attach_schedule
    @job_name = N'RefreshMaterializedViewJob',
    @schedule_name = N'HourlySchedule';

-- Assign the job to the SQL Server Agent service
EXEC dbo.Sp_add_jobserver
     @job_name = N'RefreshMaterializedViewJob';

go 



      ---3. Manual Refresh
EXEC sp_refreshview 'TotalClaimsByPatient';
EXEC sp_refreshview 'TotalTreatmentCostByProvider';
EXEC sp_refreshview 'MonthlyClaimsByProvider';

--The EXEC sp_refreshview statements refresh the metadata for each specified view. This ensures that the views are up to date with any changes made to the underlying tables (like added columns or altered data types), helping to prevent potential schema-related errors during future queries or operations.
        \end{lstlisting}

\subsubsection{Performance Testing and Data Analysis}

\textbf{......\textbf{Work on Progress}..........}       

% Test Test Test Test---------------------------------Test 
