\section{Future Work}
While the proposed approach demonstrated significant improvements in query performance, several areas for future work have been identified:

\begin{itemize}
    \item \textbf{Dynamic Query Workload Analysis}: Implement a dynamic system to analyze query workloads in real-time and automatically create or update materialized views based on changing query patterns.
    
    \item \textbf{Hybrid Optimization Techniques}: Explore the integration of PSO with other optimization algorithms, such as Genetic Algorithms (GA) or Simulated Annealing (SA), to further improve convergence and avoid local optima.
    
    \item \textbf{Cost-Based Optimization}: Incorporate storage and maintenance costs into the optimization process to ensure that materialized views remain cost-effective in large-scale databases.
    
    \item \textbf{Cloud Integration}: Extend the solution to cloud-based database systems, such as Azure SQL Database or Amazon RDS, to evaluate its scalability and performance in distributed environments.
    
    \item \textbf{Machine Learning for Adaptive PSO}: Use machine learning techniques to adaptively adjust PSO parameters based on historical performance data, improving the efficiency of the optimization process.
    
    \item \textbf{Benchmarking with Standard Datasets}: Evaluate the proposed approach using standard benchmarking datasets, such as TPC-H or TPC-DS, to compare its performance with existing methods.
\end{itemize}

These future directions aim to enhance the scalability, efficiency, and applicability of the proposed solution in real-world database systems.
\textbf{......\textbf{Work on Progress}..........}       