\thispagestyle{empty}
\begin{center}
    \fancyhead[]{}\Large\textbf{Abstract}
\end{center}

\normalsize
This proposal emphasizes the improvement of query effectiveness through the discussion of an array of metrics that reflect different aspects of performance, which is essential for improving execution speed and flexibility in large-scale information management, particularly in databases systems. Collaborating with opta data Finance GmbH, an IT consulting and system integration company based in Essen. This company offers a range of services and products for healthcare companies as well as medium-sized firms outside the healthcare sector. This research paper points to boosting framework query throughput (which refers to the number of queries or transactions processed by the database in each period) and decreasing execution times (query latency). Additionally, the study emphasizes resource utilization,  including CPU,\footnote{CPU, a central processing unit, also called a central processor, main processor, is the most important processor in a given computer.} Memory and I/O during query execution time.\vspace{.4cm}

This thesis also offers query optimization techniques like \hyperref[term:materialized_views]{Materialized views \footnote{Materialized view explain later part on this paper chapter 2.6.}} that can be used to reduce the time to select optimized queries related to views in database systems. Materialized views have been found to be very effective at speeding up queries and are increasingly effective and being supported by commercial databases. Materialized views can provide massive improvements in query processing time, especially for aggregation queries over large tables. To realize this potential, the query optimizer must know how and when to exploit materialized views. This paper presents a fast and scalable algorithm for determining whether part or all of a query can be computed from materialized views and describes how it can be incorporated in transformation-based optimizers.\vspace{.4cm}

The selection of materialized views is one of the most important decisions in designing a data warehouse for optimal efficiency. A suitable set of materialized views minimizes the total cost associated with their maintenance and storage, making it the key component in data warehousing. To solve this problem, the use of the Particle Swarm Optimization (PSO) algorithm is proposed, which helps to select MVs to accelerate workloads efficiently. This paper gives the results of a PSO-based selection algorithm, indicating its effectiveness in improving query performance.\vspace{.4cm}



%The current version handles views composed of selections, joins and a final group-by. Optimization remains fully cost based, that is, a single “best” rewrite is not selected by heuristic rules but multiple rewrites are generated and the optimizer chooses the best alternative in the normal way. Experimental results based on an implementation in Microsoft SQL Server show outstanding performance and scalability. Optimization time increases slowly with the number of views but remains low even up to a thousand.\\
%We focuses on the integration of Prometheus, Grafana, a powerful open source visualization tools (AI Tools EverSQL, SQL AI,Apache Spark,ApexSQl Plan) to monitor,improve query performance within these system and the challenges associated with query performance,including network latency,data consistency and load distribution.\\
%This thesis conduct a series of experimental methodology involves settings up a simulated distributed database environment where various query performance issues are artificially created.Grafana/Prometheus AI Tools  are  integrated to monitor various metrics ( such as query response time,system throughput and resource usage).We will create Grafana/ Prometheus dashboard to visualize these metrics,enabling clear and actionable insights into query performance.In addition we will discuss the application of specific optimization strategies guided by Grafanas/ AI Tools  analytics,such as query turning,indexing and implementation of effective caching mechanisms..\cite{1,3,2,Materialized-View}.\\

\noindent \textbf{Keywords:} Database, Database system, Query Optimization, Materialized views, View Matching, DML, DDL, SPJ, ACID, API, ETL SQL, Indexing, Latency, Throughput, Resource utilization, System Efficiency, Performance Enhancement. Particle swarm algorithm(PSO).

 

