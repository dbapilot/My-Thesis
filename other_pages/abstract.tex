
\begin{center}
    \fancyhead[]{}\Large\textbf{Abstract}
    \cfoot{\thepage} % add page number in center footer
\end{center}

\normalsize
This thesis emphasizes the improvement of query effectiveness through the discussion of an array of metrics that reflect different aspects of performance, which is essential for improving execution speed and flexibility in large-scale information management, particularly in database systems. This study points to boosting framework query throughput (which refers to the number of queries or transactions processed by the database in each period) and decreasing execution times (query latency). Additionally, the study emphasizes resource utilization,  including CPU,\footnote{CPU, a central processing unit, also called a central processor, main processor, is the most important processor in a given computer.} Memory and I/O during query execution time.\vspace{.4cm}

This thesis also offers query optimization techniques like \hyperref[term:materialized_views]{Materialized views \footnote{Materialized view explain later part on this paper chapter 2.6.}} that can be used to reduce the time to select optimized queries related to views in database systems. Materialized views have been found to be very effective at speeding up queries and are increasingly becoming effective and being supported by commercial databases. Materialized views can provide massive improvements in query processing time, especially for aggregate queries over large tables. To take advantage of this potential, the query optimizer must know how and when to exploit materialized views. This paper presents a fast and scalable algorithm for determining whether part or all of a query can be computed from materialized views and describes how it can be incorporated in transformation-based optimizers.\vspace{.4cm}

The selection of materialized views is one of the most important decisions in designing a data warehouse for optimal efficiency. A suitable set of materialized views minimizes the total cost associated with their maintenance and storage, making it the key component in data warehousing. To solve this problem, the use of the Particle Swarm Optimization (PSO) algorithm is proposed, which helps to select MVs to accelerate workloads efficiently. This paper gives the output of a PSO-based selection algorithm, indicating its effectiveness in improving query performance.\vspace{.4cm}



\noindent \textbf{Keywords:} Database, SQL, ETL, Materialized views, DML, API, Indexing, Latency, Throughput, DDL, Resource utilization, ACID, Performance Enhancement, SPJ, Particle swarm algorithm(PSO).

 

