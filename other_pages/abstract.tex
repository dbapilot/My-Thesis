\begin{center}
    \Large\textbf{Abstract}
\end{center}

\normalsize
This proposal underscores the improvement of query  effectiveness through  the  discussion of an array of metrics  that reflect different aspects of performance, which is  essential for progressing execution and flexibility in large-scale information administration, with a center on databases. Collaborating with opta data Finance GmbH, is a company in the IT consulting and system integration industry based in Essen. It offers a variety of services and products for healthcare companies as well as for medium-sized companies outside the healthcare sector. This research paper  points to boost framework Query  throughput(refers to the number of queries or transactions processed by the database in each time period) and decrease execution times (Query Latency). we also highlight the resource utilization  includes CPU, Memory and I/O during query execution time.\\
This  thesis also offers some  mechanism or query optimization technique like materialized views that can be used to reduce the time required to perform many queries in database systems.\\
Materialized views have been found to be very effective at speeding up quarries and are increasingly effective and being supported by commercial database. we introduce a algorithm that can also be used to efficient select materialized views to speed up the  workloads containing quarries and updates.\\
Materialized views can provide massive improvements in query processing time, especially for aggregation queries over large tables. To realize this potential, the query optimizer must know how and when to exploit materialized views. This paper presents a fast and scalable algorithm for determining whether part or all of a query can be computed from materialized views and describes how it can be incorporated in transformation-based optimizers.\\

Selection of materialized views is one of the most important decisions in designing a data warehouse for optimal efficiency. Selecting a suitable set of views that minimizes the total cost associated with the materialized views and is the key component in data warehousing. Materialized views are found to be very useful for fast query
processing. This paper gives the results of proposed particle swarm optimization(PSO) based materialized view selection algorithm for
query processing.\\



%The current version handles views composed of selections, joins and a final group-by. Optimization remains fully cost based, that is, a single “best” rewrite is not selected by heuristic rules but multiple rewrites are generated and the optimizer chooses the best alternative in the normal way. Experimental results based on an implementation in Microsoft SQL Server show outstanding performance and scalability. Optimization time increases slowly with the number of views but remains low even up to a thousand.\\
%We focuses on the integration of Prometheus, Grafana, a powerful open source visualization tools (AI Tools EverSQL, SQL AI,Apache Spark,ApexSQl Plan) to monitor,improve query performance within these system and the challenges associated with query performance,including network latency,data consistency and load distribution.\\
%This thesis conduct a series of experimental methodology involves settings up a simulated distributed database environment where various query performance issues are artificially created.Grafana/Prometheus AI Tools  are  integrated to monitor various metrics ( such as query response time,system throughput and resource usage).We will create Grafana/ Prometheus dashboard to visualize these metrics,enabling clear and actionable insights into query performance.In addition we will discuss the application of specific optimization strategies guided by Grafanas/ AI Tools  analytics,such as query turning,indexing and implementation of effective caching mechanisms..\cite{1,3,2,Materialized-View}.\\

\noindent \textbf{Keywords:} Database, Distributed Databases, Query Optimization, materialized views, view Matching, AI Tools, Prometheus, Ever SQL, Indexing, Latency, Throughput, resource utilization Machine Learning, System Efficiency, Performance Enhancement. Particle swarm algorithm.

 

